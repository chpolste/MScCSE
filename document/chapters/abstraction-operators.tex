In order to define states and actions for the game abstraction of the LSS, meaningful operators on the state- and control space have to be introduced that reflect different aspects of the system dynamics.
The now operators require knowledge of the parameters of the LSS, i.e.\ $\MatA$, $\MatB$, $\StateSpace$, $\RandomSpace$, $\ControlSpace$.
The association of operator and LSS is not written out explicitly.
It will always be clear from context which system an operator belongs to.

The following sections introduce the operators together with bried descriptions of how to compute them in the polytopic setting assumed here.
%TODO give own section: Decomposition into disjunct $\IndexedSet{\State{i}}{i \in I}$, $I \subseteq \naturalnumbers$ where $\State{i} \subseteq \ExtendedStateSpace$ for all $i \in I$.


\startsubsection[title={Posterior}]

    The posterior ($\Post$) is a forward looking operator, yielding the set of states that can be reached from an origin region $\StateRegion \subseteq \StateSpace$ under control inputs $\ControlSpace' \subseteq \ControlSpace$ with non-zero probability, i.e.\ the one-step reachable set:

    \startformula
        \Posterior{\StateRegion}{\ControlRegion} := \Set{ \VecState \in \reals^n \mid \exists \VecState' \in \StateRegion\MidComma \exists \VecControl' \in \ControlRegion\MidComma \exists \VecRandom \in \RandomSpace : \VecState = \MatA \VecState' + \MatB \VecControl' + \VecRandom } \EndPeriod
    \stopformula

    The posterior allows computation of $\ExtendedStateSpace = \StateSpace \,\cup\, \Posterior{\StateSpace}{\ControlSpace}$, required for the outer states of the system decomposition in \in[sec:abstraction-decomposition].

    While the above definition takes sets as arguments, the notation of the posterior and, where sensible, all following operators is overloaded for vector-valued state-space arguments as well as sets of parts of the state space decomposition.
    Let $\VecState' \in \StateSpace$ and $\State{j} \in \StateSpace$ for all $j \in J$, then

    \startformula
        \startalign[n=2,align={right,left}]
            \NC \Posterior{\VecState'}{\ControlRegion} :=
            \NC \Posterior{\Set{\VecState'}}{\ControlRegion} \EndAnd
            \NR
            \NC \BigPosterior{\IndexedStates{j}{J}}{\ControlRegion} :=
            \NC \BigPosterior{\bigcup_{j \in J}\State{j}}{\ControlRegion} \EndPeriod
            \NR
        \stopalign
    \stopformula

    For polytopic inputs, $\Post$ can be computed with the Minkowski sum as

    \startformula
        \Posterior{\StateRegion}{\ControlSpace'} = \MatA \StateRegion \oplus \MatB \ControlRegion \oplus \RandomSpace \EndPeriod
    \stopformula

\stopsection


\startsubsection[title={Predecessors}]

    As the name suggests, the predecessor operators are backward looking, yielding different kinds of origin regions for a given target and control input.
    First, the predecessor ($\Pre$) and robust predecessor ($\PreR$) are defined as

    \startformula
        \startalign[n=2,align={right,left}]
            \NC \Predecessor{\StateRegion}{\ControlSpace'}{\StateTarget} :=
            \NC \Set{ \VecState' \in \StateRegion \mid \exists \VecControl' \in \ControlSpace' : \Posterior{\VecState'}{\VecControl'} \cap \StateTarget \neq \emptyset } \EndAnd
            \NR
            \NC \RobustPredecessor{\StateRegion}{\ControlSpace'}{\StateTarget} :=
            \NC \Set{ \VecState' \in \StateRegion \mid \exists \VecControl' \in \ControlSpace' : \Posterior{\VecState'}{\VecControl'} \subseteq \StateTarget } \EndComma
            \NR
        \stopalign
    \stopformula

    where the second argument $\ControlRegion \subseteq \ControlSpace$ is the considered control space region, the third parameter $\StateRegion \subseteq \StateSpace$ is a target region in the extended state space and the first argument $\StateTarget \subseteq \ExtendedStateSpace$ is a region of the state space to which the returned predecessor is restricted and exists mainly for convenience.
    From any state in the computed origin region of $\Pre$, the target region can be reached with non-zero probability in the next step for some control input in the given set.
    It should be noted that the control input that enables the transition can (and likely will) be different for every state in the predecessor.

    The probability of reaching the target region in case of $\PreR$ is 1, meaning that the robust predecessor is \quotation{robust} in the sense that irrespective of the stochastic dynamics, the target can be reached from the computed origin region.
    Trivially, it holds that

    \startformula
        \RobustPredecessor{\StateRegion}{\ControlSpace'}{\StateTarget} \subseteq \Predecessor{\StateRegion}{\ControlSpace'}{\StateTarget} \EndPeriod
    \stopformula

    For target regions where $\StateTarget \ominus \RandomSpace = \emptyset$, the $\PreR$ is always empty.
    This can easily be seen from the computations for polytopic regions

    \startformula
        \startalign[n=2,align={right,left}]
            \NC \Predecessor{\StateRegion}{\ControlSpace'}{\StateTarget} =
            \NC \StateRegion \cap \Big( \StateTarget \oplus -(\MatB \ControlRegion \oplus \RandomSpace) \Big) \MatA \EndAnd
            \NR
            \NC \RobustPredecessor{\StateRegion}{\ControlSpace'}{\StateTarget} =
            \NC \StateRegion \cap \Big( (\StateTarget \ominus \RandomSpace) \oplus -(\MatB \ControlRegion \oplus \RandomSpace) \Big) \MatA \EndPeriod
            \NR
        \stopalign
    \stopformula

    Finally, the precise predecessor is defined, an operator that only makes sense with a state space partition:

    \startformula
        \startalign[n=2,align={right,left}]
            \NC \PrecisePredecessor{\StateSpace'}{\ControlSpace'}{\IndexedSet{\State{j}}{j \in J}} := \Big\{ \VecState \in \StateSpace' \Bigmid \exists \VecControl' \in \ControlSpace' :
            \NC \Posterior{\VecState}{\VecControl'} \subseteq \bigcup_{j \in J} \State{j} \;\text{and}
            \NR
            \NC \empty
            \NC ~\forall j \in J : \Posterior{\VecState}{\VecControl'} \cap \bigcup_{j \in J} \State{j} \neq \emptyset \Big\} \EndPeriod
            \NR
        \stopalign
    \stopformula

    Every state in the resulting origin set fulfills the robust predecessor property with respect to the entire target region $\bigcup_{j \in J} \State{j}$ for some $\VecControl' \in \ControlRegion$, while simultaneously fulfilling the predecessor property with respect to every individual part of the target region for the same $\VecControl'$.
    That means for any state in a $\PreP$ set a control input in $\ControlRegion$ exists such that a state in $\bigcup_{j \in J} \State{j}$ is reached almost surely after one step and the proability of ending up in any one of the target region parts $\IndexedStates{j}{J}$ is non-zero.
    For single-member target sets, $\PreP = \PreR$

\stopsection


\startsubsection[title={Attractors}]

    Attractor ($\Attr$) and Robust Attractor ($\AttrR$), where

    \startformula
        \startalign[n=2,align={right,left}]
            \NC \Attractor{\StateRegion}{\ControlSpace'}{\StateTarget} :=
            \NC \BigSet{ \VecState \in \StateRegion \Bigmid \forall \VecControl' \in \ControlSpace' : \Posterior{\VecState}{\VecControl'} \cap \StateTarget \neq \emptyset } \EndPeriod
            \NR
            \NC \RobustAttractor{\StateRegion}{\ControlSpace'}{\StateTarget} :=
            \NC \BigSet{ \VecState \in \StateRegion \Bigmid \forall \VecControl' \in \ControlSpace' : \Posterior{\VecState}{\VecControl'} \subseteq \StateTarget } \EndPeriod
            \NR
        \stopalign
    \stopformula

    The following relations hold:

    \startformula
        \startalign[n=2,align={right,left}]
            \NC \RobustAttractor{\StateRegion}{\ControlSpace'}{\StateTarget} \subseteq
            \NC \Attractor{\StateRegion}{\ControlSpace'}{\StateTarget} \EndComma
            \NR
            \NC \RobustAttractor{\StateRegion}{\ControlSpace'}{\StateTarget} \subseteq
            \NC \RobustPredecessor{\StateRegion}{\ControlSpace'}{\StateTarget}
            \NR
            \NC \Attractor{\StateRegion}{\ControlSpace'}{\StateTarget} \subseteq
            \NC \Predecessor{\StateRegion}{\ControlSpace'}{\StateTarget} \EndComma
            \NR
        \stopalign
    \stopformula

    The attractors can be computed from the predecessor operators as

    \startformula
        \startalign[n=2,align={right,left}]
            \NC \Attractor{\StateRegion}{\ControlSpace'}{\StateTarget} =
            \NC \StateRegion \setminus \RobustPredecessor{\StateRegion}{\ControlSpace'}{\ExtendedStateSpace \setminus \StateTarget} \EndComma
            \NR
            \NC \RobustAttractor{\StateRegion}{\ControlSpace'}{\StateTarget} =
            \NC \StateRegion \setminus \Predecessor{\StateRegion}{\ControlSpace'}{\ExtendedStateSpace \setminus \StateTarget} \EndPeriod
            \NR
        \stopalign
    \stopformula

\stopsection


\startsubsection[title={Actions}]

    Finally, define two operators in the control space: Action Polytope ($\Act$) and Robust Action Polytope ($\ActR$)

    \startformula
        \startalign[n=2,align={right,left}]
            \NC \Action{\StateRegion}{\StateTarget} =
            \NC \BigSet{ \VecControl \in \ControlSpace \Bigmid \Posterior{\StateRegion}{\VecControl} \cap \StateTarget \neq \emptyset } \EndComma
            \NR
            \NC \RobustAction{\StateRegion}{\StateTarget} =
            \NC \BigSet{ \VecControl \in \ControlSpace \Bigmid \Posterior{\StateRegion}{\VecControl} \subseteq \StateTarget } \EndComma
            \NR
        \stopalign
    \stopformula

    where $\StateRegion$ is the origin region and $\IndexedSet{\State{j}}{j \in J}$ is the set of target states.
    Both are adapted from the non-probabilistic setting of \cite[Yordanov2009], who defined the operators $U^{X \rightarrow Y}$ and $U^{X \Rightarrow Y}$, corresponding to $\Action{X}{Y}$ and $\RobustAction{X}{Y}$, respectively.

    For a single origin polytope $\State{i}$ and a target state set containing only one state polytope $\State{j}$, the operators can be computed using the Minkowski sum and Pontryagin difference:

    \startformula
        \startalign[n=2,align={right,left}]
            \NC \Action{\State{i}}{\State{j}} =
            \NC \Big( ( \State{j} \oplus - (\MatA \State{i} \oplus \RandomSpace) ) \MatB \Big) \cap \ControlSpace \EndComma
            \NR
            \NC \RobustAction{\State{i}}{\State{j}} =
            \NC \Big( ( \State{j} \ominus (\MatA \State{i} \oplus \RandomSpace) ) \MatB \Big) \cap \ControlSpace \EndPeriod
            \NR
        \stopalign
    \stopformula

    This method can be extended to sets of convex polytopes using the Minkowski sum and Pontryagin difference variants for such sets from section \in[sec:theory-geometry-operations].

    Finally the concrete action operator for the control space:

    \startformula
        \startalign[n=2,align={right,left}]
            \NC \ConcreteAction{\StateRegion}{\IndexedSet{\State{j}}{j \in J}} := \Big\{ \VecControl \in \ControlSpace \Bigmid
            \NC \Posterior{\StateRegion}{\VecControl} \subseteq \bigcup_{j \in J} \State{j} \;\text{and}
            \NR
            \NC \empty
            \NC ~\forall j \in J : \Posterior{\StateRegion}{\VecControl} \cap \State{j} \neq \emptyset \Big\} \EndPeriod
            \NR
        \stopalign
    \stopformula

    Only fully dimensional polytopic regions of the control space considered here (implementation purposes).
    Decision not to include single-point actions.

    \startformula
        \startalign[n=2,align={right,left}]
            \NC \ConcreteAction{\StateRegion}{\IndexedSet{\State{j}}{j \in J}} =
            \NC \bigcap_{j \in J} \Action{\StateRegion}{\Set{\State{j}}} \setminus \bigcup_{i \in I \setminus J} \Action{\StateRegion}{\Set{\State{i}}}\EndPeriod
            \NR
        \stopalign
    \stopformula

    Elaborate on difference between $\Act$ and $\PreR$: Precise Predecessor requires one $\VecControl$ for which both subset and intersection condition is fulfilled.
    Action requires one $\VecState$ for which intersection condition is fulfilled, but subset condition must be fulfilled for all $\VecX \in \StateRegion$.

    Therefore,

    \startformula
        \ConcreteAction{\StateRegion}{\IndexedSet{\State{j}}{j \in J}} \,\cap\, \ConcreteAction{\StateRegion}{\IndexedSet{\State{j}}{j \in J'}} = \emptyset
    \stopformula

    for all $J \neq J'$.
    The same does not hold for the precise predecessor, whose sets can overlap for different target sets.

\stopsection

