% Convention: CamelCase for custom commands

\define\DocTitleFooter{Refinement for game-based abstractions of continuous-space LSSs} % TODO
\define\Author{Christopher Polster}
\define\YearOfCompletion{2019}
\define\DateOfCompletion{\YearOfCompletion-??-??} % TODO


% Math
% ----

% Tall version of \mid
\definemathcommand[Bigmid][rel]{\mathrel{\Big|}}

% Cartesian product with flattening
\definemathcommand[ftimes][op]{\mathrel{\check\times}}

% Equation endings
\define\EndComma{\;\text{,}}
\define\EndPeriod{\;\text{.}}
\define\EndAnd{\;\text{and}}
% Inline connectors
\define\MidComma{\text{,}\;}
\define\MidAnd{\;\text{and}\;}
\define\StartIf{\text{if}\;}

% Function definition
\define[3]\Function{#1:#2\rightarrow#3}

% 2-Norm
\define[1]\TwoNorm{\lVert{#1}\rVert}

% Vector variables with bolditalic font
\define[1]\Vec{{\bi #1}}
\define\VecC{\Vec{c}}
\define\VecU{\Vec{u}}
\define\VecV{\Vec{v}}
\define\VecW{\Vec{w}}
\define\VecX{\Vec{x}}
\define\VecY{\Vec{y}}
\define\VecZ{\Vec{z}}

% Matrix variables with bolditalic font
\define[1]\Mat{{\bi{#1}}}
\define\MatA{\Mat{A}}
\define\MatB{\Mat{B}}
\define\MatU{\Mat{U}}

% Vectors and matrices with square brackets
\definemathmatrix[sqmatrix][left={\left\lbrack\,},right={\,\right\rbrack},strut=0.8,distance=0.8em]
\define[4]\TwoByTwo{ \startsqmatrix[n=2,align={middle,middle}] \NC #1 \NC #2 \NR \NC #3 \NC #4 \NR \stopsqmatrix }
\define[2]\TwoByOne{ \startsqmatrix[n=1,align={middle}] \NC #1 \NR \NC #2 \NR \stopsqmatrix }
\define[2]\OneByTwo{ \startsqmatrix[n=2,align={middle,middle}] \NC #1 \NC #2 \NR \stopsqmatrix }
\define[1]\OneByOne{ \startsqmatrix[n=1,align={middle}] \NC #1 \NR \stopsqmatrix }

% Inline vectors and matrices with square brackets (https://www.mail-archive.com/ntg-context@ntg.nl/msg78899.html)
\definemathmatrix[ssqmatrix][left=\left\lbrack,right=\right\rbrack,style=\scriptstyle,strut=0.5,distance=0.5em]
\define[2]\TwoByOneSmall{ \startssqmatrix[n=1,align={middle}] \NC #1 \NR \NC #2 \NR \stopssqmatrix }

% Sets and Intervals
\define[1]\Set{\{ {#1} \}}
\define[1]\BigSet{\Big\{ {#1} \Big\}}
\define[2]\IndexedSet{\Set{#1}_{#2}}
\define[2]\ClosedInterval{\left[\, #1,\, #2 \,\right]}

% N-Tuples
\define[2]\Tuple{({#1},\,{#2})}
\define[2]\BigTuple{\Big({#1},\,{#2}\Big)}
\define[3]\Triple{({#1},\,{#2},\,{#3})}
\define[3]\BigTriple{\Big({#1},\,{#2},\,{#3}\Big)}

% Inline multiline aligned blocks (http://dl.contextgarden.net/myway/mathalign.pdf)
\definemathmatrix[gathered][n=1,align=left,style=\displaystyle]

% Polytopic Computations
% ----------------------

% Polytope components
\definemathcommand[Hull][nolop]{\mfunction{hull}}
\definemathcommand[Vertices][nolop]{\mfunction{vert}}
% Posterior
\definemathcommand[Post][nolop]{\mfunction{Post}}
% Predecessors
\definemathcommand[Pre][nolop]{\mfunction{Pre}}
\definemathcommand[PreR][nolop]{\mfunction{PreR}}
\definemathcommand[PreP][nolop]{\mfunction{PreP}}
% Attractors
\definemathcommand[Attr][nolop]{\mfunction{Attr}}
\definemathcommand[AttrR][nolop]{\mfunction{AttrR}}
% Action Polytopes
\definemathcommand[Act][nolop]{\mfunction{Act}}
\definemathcommand[ActR][nolop]{\mfunction{ActR}}
\definemathcommand[ActC][nolop]{\mfunction{ActC}}

% LSS
\define\LSS{{\mathcal{T}}}

% States
\define\StateSpace{X}
\define\StateRegion{\StateSpace'}
\define\StateTarget{\tilde\StateSpace}
\define\ExtendedStateSpace{\StateSpace_{\mathrm ext}}
\define\InitialStates{\StateSpace_{\mathrm init}}
\define\VecState{\VecX}
\define[1]\State{X_{#1}}
\define[2]\IndexedStates{\IndexedSet{\State{#1}}{{#1} \in {#2}}}
\define\StateIndices{I}
\define\StateIndicesInner{I_{\mathrm in}}
\define\StateIndicesOuter{I_{\mathrm out}}

% Linear Predicates
\definemathcommand[FulfilledPredicates][nolop]{\mfunction{\pi}}
\define\Predicates{\Pi}
\define\Predicate{\pi}
\define[1]\PredicatesOf{\FulfilledPredicates({#1})}

% Control
\define\ControlSpace{U}
\define\ControlRegion{\ControlSpace'}
\define\VecControl{\VecU}
\define[2]\Action{\Act({#1},\, {#2})}

% Random
\define\RandomSpace{W}
\define\VecRandom{\VecW}

% Polytopic Operators
\define[2]\Posterior{\Post({#1},\, {#2})}
\define[3]\Predecessor{\Pre({#1},\, {#2},\, {#3})}
\define[3]\RobustPredecessor{\PreR({#1},\, {#2},\, {#3})}
\define[3]\PrecisePredecessor{\PreP({#1},\, {#2},\, {#3})}
\define[3]\Attractor{\Attr({#1},\, {#2},\, {#3})}
\define[3]\RobustAttractor{\AttrR({#1},\, {#2},\, {#3})}
\define[2]\Action{\Act({#1},\, {#2})}
\define[2]\RobustAction{\ActR({#1},\, {#2})}
\define[2]\ConcreteAction{\ActC({#1},\, {#2})}
% And variants with big parenthesis (I don't like how \left and \right handle some situations
\define[2]\BigPosterior{\Post\Big({#1},\, {#2}\Big)}
\define[3]\BigPredecessor{\Pre\Big({#1},\, {#2},\, {#3}\Big)}
\define[3]\BigRobustPredecessor{\PreR\Big({#1},\, {#2},\, {#3}\Big)}
\define[3]\BigPrecisePredecessor{\PreP\Big({#1},\, {#2},\, {#3}\Big)}
\define[3]\BigAttractor{\Attr\Big({#1},\, {#2},\, {#3}\Big)}
\define[3]\BigRobustAttractor{\AttrR\Big({#1},\, {#2},\, {#3}\Big)}
\define[2]\BigAction{\Act\Big({#1},\, {#2}\Big)}
\define[2]\BigRobustAction{\ActR\Big({#1},\, {#2}\Big)}
\define[2]\BigConcreteAction{\ActC\Big({#1},\, {#2}\Big)}



% Logic and Automata
% ------------------

\definemathcommand[Inf][nolop]{\mfunction{Inf}}
\define[1]\InfinitelyOften{\Inf(#1)}

% Words
\define\Automaton{{\mathcal A}}
\define\Language{{\mathcal L}}
\define\Condition{{\mathcal C}}
\define\Transition{\delta}
\define[1]\RepeatFinitely{{#1}^{\ast}}
\define[1]\RepeatInfinitely{{#1}^{\omega}}

\definemathcommand[True]{{\mathss true}}
\definemathcommand[Next]{{\mathss X}}
\definemathcommand[Until]{\,{\mathss U}\,}
\definemathcommand[Finally]{{\mathss F}}
\definemathcommand[Globally]{{\mathss G}}


% Games
% -----

\definemathcommand[AlmostSure][nolop]{\mfunction{Almost}}

\define\GameGraph{{\mathcal G}}
\define\Coop{{\mathrm coop}}
\define[2]\Strategy{S_{#2}^{#1}}
\define[1]\Almost{\AlmostSure_{#1}}
\define[1]\AlmostCoop{\AlmostSure_{#1}^{\Coop}}
\define[1]\ProbDist{\mathcal{D}(#1)}

\define\ProductGame{{\mathcal P}}
\define[2]\PlayerOneAction{{#1}\rightarrow{#2}}
\define[3]\PlayerTwoAction{\Tuple{#1}{#2}\rightarrow{#3}}


% Physics
% -------

\define\DDt{\frac{{\mathrm d}}{{\mathrm d}t}}

\define\Angle{\alpha}
\define\AngularMomentum{L}
\define\Torque{\tau}
\define\Force{F}
\define\MaxForce{\tilde\Force}
\define\MomentOfInertia{I}
\define\Mass{m}
\define\Length{l}
\define\Gravity{g}
\define\Friction{\mu}
\define\Thrust{u}
\define\Deltat{\Delta t}


% Additional commands for algorithmic

\unprotect

\starttexdefinition algorithmic_function #1#2
  \par
  \algorithmic_label{function} {\sc #1}({#2})
  \algorithmic_begsub
\stoptexdefinition

\starttexdefinition algorithmic_endfunction
  \algorithmic_endsub
  \algorithmic_label{endfunction}
\stoptexdefinition

\starttexdefinition algorithmic_do
  \par
  \algorithmic_label{do}
  \algorithmic_begsub
\stoptexdefinition

\starttexdefinition algorithmic_enddowhile #1
  \algorithmic_endsub
  \algorithmic_label{while} #1
\stoptexdefinition

\algorithmic_define[FUNCTION][\algorithmic_function]
\algorithmic_define[ENDFUNCTION][\algorithmic_endfunction]
\algorithmic_define[DO][\algorithmic_do]
\algorithmic_define[ENDDOWHILE][\algorithmic_enddowhile]

\setupalgorithmictext[\s!en][function=function]
\setupalgorithmictext[\s!en][endfunction=end function]

\protect

