Operators that reflect certain aspects of dynamics and are the foundation of the abstraction construction and for refinement.
Operators carry implicit knowlegde of LSS ($A$, $B$, $\RandomSpace$, $\ControlSpace$).
Association is not written out explicitly, should always be clear from context.

Important operators used for game construction and refinement presented together with ways to compute them.

TODO Decomposition into disjunct $\IndexedSet{\State{i}}{i \in I}$, $I \subseteq \naturalnumbers$ where $\State{i} \subseteq \ExtendedStateSpace$ for all $i \in I$.


\startsubsection[title={Posterior}]

    Let $\StateRegion \subseteq \StateSpace$ and $\ControlSpace' \subseteq \ControlSpace$.
    The first operator introduced is the posterior

    \startformula
        \Posterior{\StateRegion}{\ControlSpace'} := \Set{ \VecState \in \reals^n \mid \exists \VecState' \in \StateRegion\MidComma \exists \VecControl' \in \ControlSpace'\MidComma \exists \VecRandom \in \RandomSpace : \VecState = \MatA \VecState' + \MatB \VecControl' + \VecRandom } \EndComma
    \stopformula

    with overloaded notation for vector-valued arguments, such that

    \startformula
        \startalign[n=2,align={right,left}]
            \NC \Posterior{\VecState'}{\ControlSpace'} = \NC \Posterior{\Set{\VecState'}}{\ControlSpace'} \EndComma \NR
            \NC \Posterior{\StateRegion}{\VecControl'} = \NC \Posterior{\StateRegion}{\Set{\VecControl'}} \EndAnd \NR
            \NC \Posterior{\VecState'}{\VecControl'} = \NC \Posterior{\Set{\VecState'}}{\Set{\VecControl'}} \EndPeriod \NR
        \stopalign
    \stopformula

    One-step reachable set.

    Let $\ExtendedStateSpace = \StateSpace \,\cup\, \Posterior{\StateSpace}{\ControlSpace}$.

\stopsection


\startsubsection[title={Predecessors}]

    Next operators are the predecessor ($\Pre$) and attractor ($\Attr$):

    \startformula
        \startalign[n=2,align={right,left}]
            \NC \Predecessor{\StateRegion}{\ControlSpace'}{\IndexedSet{\State{j}}{j \in J}} :=
            \NC \BigSet{ \VecState \in \StateRegion \Bigmid \exists \VecControl' \in \ControlSpace' : \Posterior{\VecState}{\VecControl'} \cap \bigcup_{j \in J} \State{j} \neq \emptyset } \EndComma
            \NR
            \NC \RobustPredecessor{\StateRegion}{\ControlSpace'}{\IndexedSet{\State{j}}{j \in J}} :=
            \NC \BigSet{ \VecState \in \StateRegion \Bigmid \exists \VecControl' \in \ControlSpace' : \Posterior{\VecState}{\VecControl'} \subseteq \bigcup_{j \in J} \State{j} } \EndComma
            \NR
        \stopalign
    \stopformula

    First argument is origin, last argument is state-space target region where $J \subseteq I$.
    Interpretation: From where in $\StateRegion$ is it possible to go to enter the target region with non-zero probability in the next step for at least one ($\Pre$) or all ($\Attr$) control inputs in $\ControlSpace'$.
    Robust variants of predecessor and attractor

    Subsets of predecessor and attractor, respectively.
    Cannot go to anywhere outside the target region, irrespective of stochastic outcome.

    The following relations hold:

    \startformula
        \startalign[n=2,align={right,left}]
            \NC \Predecessor{\StateRegion}{\ControlSpace'}{\IndexedSet{\State{j}}{j \in J}} =
            \NC \bigcup_{j \in J} \Predecessor{\StateRegion}{\ControlSpace'}{\Set{\State{j}}} \EndComma
            \NR
            \NC \RobustPredecessor{\StateRegion}{\ControlSpace'}{\IndexedSet{\State{j}}{j \in J}} \subseteq
            \NC \Predecessor{\StateRegion}{\ControlSpace'}{\IndexedSet{\State{j}}{j \in J}} \EndComma
            \NR
        \stopalign
    \stopformula

    Again, overloading applies, here also for single-element sets of states.

    Finally, the precise predecessor

    \startformula
        \startalign[n=2,align={right,left}]
            \NC \PrecisePredecessor{\StateSpace'}{\ControlSpace'}{\IndexedSet{\State{j}}{j \in J}} := \Big\{ \VecState \in \StateSpace' \Bigmid \exists \VecControl' \in \ControlSpace' :
            \NC \Posterior{\VecState}{\VecControl'} \subseteq \bigcup_{j \in J} \State{j} \;\text{and}
            \NR
            \NC \empty
            \NC ~\forall j \in J : \Posterior{\VecState}{\VecControl'} \cap \bigcup_{j \in J} \State{j} \neq \emptyset \Big\}
            \NR
        \stopalign
    \stopformula

    requires that the robust predecessor property holds for the entire target region, while the predecessor property additionally holds for each element of the target region individually.

\stopsection


\startsubsection[title={Attractors}]

    Attractor ($\Attr$) and Robust Attractor ($\AttrR$), where

    \startformula
        \startalign[n=2,align={right,left}]
            \NC \Attractor{\StateRegion}{\ControlSpace'}{\IndexedSet{\State{j}}{j \in J}} :=
            \NC \BigSet{ \VecState \in \StateRegion \Bigmid \forall \VecControl' \in \ControlSpace' : \Posterior{\VecState}{\VecControl'} \cap \bigcup_{j \in J} \State{j} \neq \emptyset } \EndPeriod
            \NR
            \NC \RobustAttractor{\StateRegion}{\ControlSpace'}{\IndexedSet{\State{j}}{j \in J}} :=
            \NC \BigSet{ \VecState \in \StateRegion \Bigmid \forall \VecControl' \in \ControlSpace' : \Posterior{\VecState}{\VecControl'} \subseteq \bigcup_{j \in J} \State{j} } \EndPeriod
            \NR
        \stopalign
    \stopformula

    The following relations hold:

    \startformula
        \startalign[n=2,align={right,left}]
            \NC \RobustAttractor{\StateRegion}{\ControlSpace'}{\IndexedSet{\State{j}}{j \in J}} \subseteq
            \NC \Attractor{\StateRegion}{\ControlSpace'}{\IndexedSet{\State{j}}{j \in J}} \EndComma
            \NR
            \NC \RobustAttractor{\StateRegion}{\ControlSpace'}{\IndexedSet{\State{j}}{j \in J}} \subseteq
            \NC \RobustPredecessor{\StateRegion}{\ControlSpace'}{\IndexedSet{\State{j}}{j \in J}}
            \NR
            \NC \Attractor{\StateRegion}{\ControlSpace'}{\IndexedSet{\State{j}}{j \in J}} \subseteq
            \NC \Predecessor{\StateRegion}{\ControlSpace'}{\IndexedSet{\State{j}}{j \in J}} \EndComma
            \NR
        \stopalign
    \stopformula

    The attractors can be computed from the predecessor operators as

    \startformula
        \startalign[n=2,align={right,left}]
            \NC \Attractor{\StateRegion}{\ControlSpace'}{\IndexedSet{\State{j}}{j \in J}} =
            \NC \StateRegion \setminus \RobustPredecessor{\StateRegion}{\ControlSpace'}{\ExtendedStateSpace \setminus \IndexedSet{\State{j}}{j \in J}} \EndComma
            \NR
            \NC \RobustAttractor{\StateRegion}{\ControlSpace'}{\IndexedSet{\State{j}}{j \in J}} =
            \NC \StateRegion \setminus \Predecessor{\StateRegion}{\ControlSpace'}{\ExtendedStateSpace \setminus \IndexedSet{\State{j}}{j \in J}} \EndPeriod
            \NR
        \stopalign
    \stopformula

\stopsection


\startsubsection[title={Actions}]

    Finally, define two operators in the control space: Action Polytope ($\Act$) and Robust Action Polytope ($\ActR$)

    \startformula
        \startalign[n=2,align={right,left}]
            \NC \Action{\State{i}}{\IndexedSet{\State{j}}{j \in J}} =
            \NC \BigSet{ \VecControl \in \ControlSpace \Bigmid \Posterior{\State{i}}{\VecControl} \cap \bigcup_{j \in J} \State{j} \neq \emptyset } \EndComma
            \NR
            \NC \RobustAction{\State{i}}{\IndexedSet{\State{j}}{j \in J}} =
            \NC \BigSet{ \VecControl \in \ControlSpace \Bigmid \Posterior{\State{i}}{\VecControl} \subseteq \bigcup_{j \in J} \State{j} } \EndComma
            \NR
        \stopalign
    \stopformula

    where $\StateRegion$ is the origin region and $\IndexedSet{\State{j}}{j \in J}$ is the set of target states.
    Both are adapted from the non-probabilistic setting of \cite[Yordanov2009], who defined the operators $U^{X \rightarrow Y}$ and $U^{X \Rightarrow Y}$, corresponding to $\Action{X}{Y}$ and $\RobustAction{X}{Y}$, respectively.

    For a single origin polytope $\State{i}$ and a target state set containing only one state polytope $\State{j}$, the operators can be computed using the Minkowski sum and Pontryagin difference:

    \startformula
        \startalign[n=2,align={right,left}]
            \NC \Action{\State{i}}{\State{j}} =
            \NC \Big( ( \State{j} \oplus - (\MatA \State{i} \oplus \RandomSpace) ) \MatB \Big) \cap \ControlSpace \EndComma
            \NR
            \NC \RobustAction{\State{i}}{\State{j}} =
            \NC \Big( ( \State{j} \ominus (\MatA \State{i} \oplus \RandomSpace) ) \MatB \Big) \cap \ControlSpace \EndPeriod
            \NR
        \stopalign
    \stopformula

    This method can be extended to sets of convex polytopes using the Minkowski sum and Pontryagin difference variants for such sets from section \in[sec:theory-geometry-operations].

    Finally the concrete action operator for the control space:

    \startformula
        \startalign[n=2,align={right,left}]
            \NC \ConcreteAction{\StateRegion}{\IndexedSet{\State{j}}{j \in J}} := \Big\{ \VecControl \in \ControlSpace \Bigmid
            \NC \Posterior{\StateRegion}{\VecControl} \subseteq \bigcup_{j \in J} \State{j} \;\text{and}
            \NR
            \NC \empty
            \NC ~\forall j \in J : \Posterior{\StateRegion}{\VecControl} \cap \State{j} \neq \emptyset \Big\} \EndPeriod
            \NR
        \stopalign
    \stopformula

    Only fully dimensional polytopic regions of the control space considered here (implementation purposes).
    Decision not to include single-point actions.

    \startformula
        \startalign[n=2,align={right,left}]
            \NC \ConcreteAction{\StateRegion}{\IndexedSet{\State{j}}{j \in J}} =
            \NC \bigcap_{j \in J} \Action{\StateRegion}{\Set{\State{j}}} \setminus \bigcup_{i \in I \setminus J} \Action{\StateRegion}{\Set{\State{i}}}\EndPeriod
            \NR
        \stopalign
    \stopformula

    Elaborate on difference between $\Act$ and $\PreR$: Precise Predecessor requires one $\VecControl$ for which both subset and intersection condition is fulfilled.
    Action requires one $\VecState$ for which intersection condition is fulfilled, but subset condition must be fulfilled for all $\VecX \in \StateRegion$.

    Therefore,

    \startformula
        \ConcreteAction{\StateRegion}{\IndexedSet{\State{j}}{j \in J}} \,\cap\, \ConcreteAction{\StateRegion}{\IndexedSet{\State{j}}{j \in J'}} = \emptyset
    \stopformula

    for all $J \neq J'$.
    The same does not hold for the precise predecessor, whose sets can overlap for different target sets.

\stopsection

