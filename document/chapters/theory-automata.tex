Automata are a means to represent a language. % TODO phrasing, maybe introduce languages first?
A Language contains words made up from symbols originating from an alphabet.
An automaton is a transition system constructed such that any given word induces a run, which is either accepted or rejected. % TODO non-det automata can have multiple valid runs for a given word
All words inducing accepted runs are part of the language of the automaton, while rejected words are not.
Many different types of automata exist for representing many different types of languages. % TODO phrasing
In this work, deterministic finite-state automata (DFAs), in particular \omega-automata, are used to define languages of infinite words.

The language of an automaton ${\mathcal A}$ is denoted as $\Language({\mathcal A})$.
A word is written as a sequence of symbols, e.g. $w_0 w_1 w_2$.
The set of all finite words over an alphabet $\Sigma$ is written as $\RepeatFinitely{\Sigma}$, while the set of all infinite words is written as $\RepeatInfinitely{\Sigma}$. % TODO some of this has been introduced before, do a "recall ..."
This notation is more generally used for sets of sequences and for (\omega-)regular expressions over an alphabet, where it represents finite ($\RepeatFinitely{\,}$) or infinite ($\RepeatInfinitely{\,}$) recurrence of a symbol (consult chapter 4 in the book of \cite[Baier2008] for a more thorough introduction to regular expressions).
% TODO sort out the notation introductions


\startsubsection[title={\omega-Automata},reference=sec:theory-automata-omega]

    A deterministic \omega-automaton is type of DFA, specified by a 5-tuple 

    \startformula
        \mathcal{A} = (Q, \Sigma, \delta, q_0, \mathcal{C}) \EndComma
    \stopformula

    where
    $Q \neq \emptyset$ is a finite set of states,
    $\Sigma$ is a finite alphabet of symbols,
    $\delta: Q \times \Sigma \rightarrow Q$ is a deterministic transition function and
    $q_0 \in Q$ is the initial state of the automaton.
    If the transition function is defined for every combination of state from $Q$ and symbol from $\Sigma$, the automaton is called complete otherwise it is called incomplete.

    A run is an infinite sequence of states $q_0 q_1 ... \in \RepeatInfinitely{Q}$ such that $q_{i+1} = \delta(q_i, w_i)$, $w_i \in \Sigma$ for all $i \in \naturalnumbers_0$.
    Every run is associated with a word $w = w_0 w_1 ... \in \RepeatInfinitely{\Sigma}$, induced by the symbols that label the transitions via $\delta$.
    The purpose of the acceptance condition $\mathcal{C}$ is to specify which words are accepted by the automaton and therefore part of $\Language({\mathcal A})$ and which words are rejected.

\stopsubsection


\startreusableMPgraphic{theory-automaton-example}
    beginfig(0);
        with spacing((30,10)) matrix.a(3,8);
        node_dash.a[1][1](btex $q_0$ etex);
        node.a[1][4](btex $q_1$ etex);
        node_double.a[1][7](btex $q_2$ etex);
        incoming(0, "") (a[1][1]) 180;
        loop.rt(.4, btex \;$a$ etex) (a[1][1]) 90;
        arrow.top(.5, btex $b$ etex) (a[1][1],a[1][4]) a[1][1].c..a[1][4].c;
        arrow.top(.5, btex $c$ etex) (a[1][4],a[1][7]) a[1][1].c..a[1][7].c;
        loop.rt(.4, btex \;$d$ etex) (a[1][7]) 90;
    endfig;
\stopreusableMPgraphic

\startsubsection[title={Acceptance Conditions}]

    For a run to be accepted by the automaton it has to start in the initial state $q_0$ and all transitions required to realize the run have to be defined by the transition function.
    In case of an incomplete automaton, all words that require a transition not defined by $\delta$ in their corresponding run are immediately rejected.
    In addition to these basic requirements of accepted runs, the acceptance condition specifies further restrictions.

    Formally, the acceptance condition ${\mathcal C}$ divides the set of all runs into a set containing all accepted runs and a set containing all not accepted, i.e. rejected, runs.
    A word is called accepted if the corresponding run is accepted.
    The set of all accepted words is the language of the automaton.
    Since there are usually infinitely many accepted words, the acceptance condition is expressed by other means than a set of words in practice.
    % TODO acceptance ... accept .. acceptance ...

    A Büchi condition is expressed by a set of states and requires that at least one state from this set is visited infinitely often in every accepted run.
    A generalization of the Büchi acceptance condition is the Büchi-implication condition, also known as the one-pair Streett condition. % TODO reference
    It is specified by a tuple of sets $\mathcal{C} = (E, F) \subseteq Q \times Q$ and accepts all runs $r$ satisfying

    \startformula
        (\InfinitelyOften{r} \cap E \ne \emptyset) \Longrightarrow (\InfinitelyOften{r} \cap F \ne \emptyset) \EndComma
    \stopformula

    where the set of states which occur infinitely often in the run $r$ is denoted by $\InfinitelyOften{r}$.
    I.e., in every accepted run an infinite occurrence of at least one state from $E$ implies that at least one state from $F$ is also visited infinitely often or else the word would have been rejected. % TODO phrasing
    % TODO empty sets E/F are never fulfilled, see FGx

    \placefigure[top][fig:theory-automaton-example]{
        An automaton with the one-pair Streett acceptance condition $(\Set{q_0}, \Set{q_2})$.
    }{
        % Put in wide box so that figure caption has proper width
        \framed[width=\textwidth,frame=off]{\reuseMPgraphic{theory-automaton-example}}
    }

    For example, consider the automaton with one-pair Streett acceptance condition

    \startformula
        (\Set{q_0, q_1, q_2}, \Set{a, b, c, d}, \delta, q_0, (\Set{q_0}, \Set{q_2})) \EndComma
    \stopformula

    where the transition function defines $\delta(q_0, a) = q_0$, $\delta(q_0, b) = q_1$, $\delta(q_1, c) = q_2$ and $\delta(q_2, d) = q_2$.
    This (incomplete) automaton accepts all words starting with a finite number of $a$s, followed by a single $b$, a single $c$ and then $d$s forever, i.e. all words of the form $\RepeatFinitely{a} b c \RepeatInfinitely{d}$.
    It is shown in Figure \in[fig:theory-automaton-example], which demonstrates how states from one-pair Streett acceptance sets $E$ (dashed border) and $F$ (double border) are highlighted in automaton depictions here.

\stopsubsection

