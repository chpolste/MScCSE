In this chapter, the \quotation{Abstraction} and \quotation{Analysis} phases of the procedure are described in detail.
The chapter also includes a brief discussion of optimality in controller synthesis in the last section.

\cite[Svorenova2017] introduce procedure for analysis and controller synthesis of (discrete-time, continuous-space) LSS, GR(1) problems.
Procedure is foundation of this work and therefore reviewed in detail here.

TODO
Suitable abstraction for LSS is 2 1/2 player game.
Fully captures controllable, nondeterministic and probabilistic aspects of system evoulution.
Restriction to convex geometry for convenience.

Introduce operators that capture important aspects of the dynamics.
Create player 1 states by partitioning the state space, and associated (deterministic) actions from the control space.
Introduce second player to compensate for non-determinism introduced in partitioning, and associated probabilistic actions.
Finally, introduce objective by constructing the product of the game abstraction and an objective automaton (here: one-pair Streett) to obtain a 2 1/2 player game with one-pair Streett objective.


Example: 1D variant of illustrative example and recurrence
System $\LSS$ with evolution equation

\placeformula[fml:abstraction-example]
\startformula
    x_{t+1} = x_t + u_t + w_t
\stopformula

where $\StateSpace = \ClosedInterval{0}{4}$, $\RandomSpace = \ClosedInterval{-0.1}{0.1}$ and $\ControlSpace = \ClosedInterval{-1}{1}$.

