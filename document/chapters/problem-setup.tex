The definition of linear stochastic systems from Section \in[sec:theory-hybrids-lss] is recalled.
An LSS is a discrete-time dynamical system, in which traces $\VecState = \VecState_0 \VecState_1 ...$ evolve according to the difference equation

\placeformula[fml:problem-setup-lsseq]
\startformula
    \VecState_{t+1} = \MatA \VecState_t + \MatB \VecControl_t + \VecRandom_t \EndComma
\stopformula

where $\MatA \in \reals^{n \times n}$, $\MatB \in \reals^{n \times m}$.
$\VecState_t \in \ControlSpace \subset \reals^n$ is the state of the trace at time $t$, $\VecControl_t \in \ControlSpace \subset \reals^m$ is a control input and $\VecRandom \in \RandomSpace \subset \reals^n$ is a random perturbation.
For convenience, it is assumed that $\StateSpace$, $\ControlSpace$ and $\RandomSpace$ are convex polytopes.
The random vector distribution is uniform and non-zero everywhere in $\RandomSpace$.
A trace is controlled by a strategy $\Function{\Strategy{}{\LSS}}{\StateSpace^+}{\ControlSpace}$ that maps finite prefixes of traces to control inputs.
Note that such an LSS can be interpreted as a Markov decision process with infinitely many states $\StateSpace$ and infinitely many actions $\ControlSpace$.

A set of linear predicates $\Predicates = \IndexedSet{\Predicate_k}{k \in K}$ is defined over the state space.
Each linear predicate $\Predicate_k$ is associated with a halfspace $H_k = \Set{ \Vec{x} \in \reals^n \mid \VecU_k \cdot \VecX \leq c_k }$.
The function $\Function{\Predicate}{\StateSpace}{2^\Predicates}$ returns the set of fulfilled predicates for a given state vector $\VecState$.
When applied element-wise to a trace, $\Predicate$ generates a word over the alphabet $2^\Predicates$ which can be interpreted by a temporal logic specification.

