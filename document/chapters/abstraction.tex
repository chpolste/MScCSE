The abstraction and analysis phases of the solution procedure outlined in section \in[sec:problem-approach] are described in this chapter, followed by a brief discussion of the controller synthesis problem.
The presented solution is the work of \cite[Svorenova2017], who attribute their approach to \cite[Yordanov2012].
It is the foundation this work is built on and therefore reviewed in detail.

First, a set of polytopic operators capturing important aspects of the dynamics is introduced in section \in[sec:abstraction-operators].
A 2½-player game graph is then build as an abstraction of the linear stochastic system in \in[sec:abstraction-graph].
This game graph is combined with an \omega-automaton obtained from the GR(1) specification in a synchronous product in \in[sec:abstraction-product].
Section \in[sec:abstraction-analysis] demonstrates how to obtain a solution to the analysis problem from the product game.
Based on this solution, an almost-sure winning control strategy for player 1 is synthesized in \in[sec:abstraction-synthesis].

Throughout this chapter the simple, 1-dimensional LSS 

\placeformula[fml:abstraction-example]
\startformula
    \VecState_{t+1} = \VecState_t + \VecControl_t + \VecRandom_t \EndComma
\stopformula

where $\VecState_t \in \StateSpace = \ClosedInterval{0}{4}$, $\VecRandom_t \in \RandomSpace = \ClosedInterval{-0.1}{0.1}$ and $\VecControl_t \in \ControlSpace = \ClosedInterval{-1}{1}$ for all $t$, is used to illustrate the construction of the solution.

