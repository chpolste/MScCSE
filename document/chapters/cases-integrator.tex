Consider the double integrator dynamics

\startformula
    \VecX_{t+1} = \TwoByTwo{1}{1}{0}{1} \VecState_{t} + \TwoByOne{0.5}{1} \VecControl_{t} + \VecRandom_{t} \EndComma
\stopformula

where $\VecState_{t} \in \StateSpace = \ClosedInterval{-5}{5} \times \ClosedInterval{-3}{3}$, $\VecControl_{t} \in \ControlSpace = \ClosedInterval{-1}{1}$ and $\VecRandom_{t} \in \RandomSpace = \ClosedInterval{-0.1}{0.1}^2$.
The objective is to reach $\ClosedInterval{-1}{1}^2$, therefore the linear predicates $\Predicate_{1}$, $\Predicate_{2}$, $\Predicate_{3}$ and $\Predicate_{4}$ are introduced, corresponding to halfspaces governed by the inequalities $x \leq 1$, $-x \leq 1$, $y \leq 1$ and $-y \leq 1$.
The LTL formula expressing this reachability objective is then

\startformula
    \Finally ( p_1 \wedge p_2 \wedge p_3 \wedge p_4 )
\stopformula

with a co-safe interpretation.
\cite[Svorenova2017] used this exact system for their case studies and it is used here again as a baseline for comparison.

\placefigure[top][fig:cases-integrator-initial]{
    The double integrator test system and its initial partition.
    Polytopes $\State{1}$ to $\State{4}$ (grey) are outer states from the decomposition of $\ExtendedStateSpace \setminus \StateSpace$.
    The reachability target is $\State{12}$ (green).
}{
    \externalfigure[cases-integrator-initial][width=\textwidth]
}


\startsubsection[title={Negative Refinement},reference=sec:cases-integrator-negative]

    \placetable[top][tab:cases-integrator-negative]{
        TODO \par TODO
    }{
        \RefinementTable{
            \RefinementTableRow[iteration=1,polys=9,onestates=28,oneactions=44,twostates=26,twoactions=216,
                                total={0:00},refinement={-},gamegraph={0:00},analysis={0:00},
                                yes=6.7,no=0.0,maybe=93.3,figure=cases-integrator-iteration1]
            \RefinementTableRow[iteration=2,polys=13,onestates=36,oneactions=60,twostates=38,twoactions=362,
                                total={0:00},refinement={0:00},gamegraph={0:00},analysis={0:00},
                                yes=6.7,no=11.3,maybe=82.1,figure=cases-integrator-iteration2]
            \RefinementTableRow[iteration=3,polys=17,onestates=44,oneactions=80,twostates=50,twoactions=592,
                                total={0:01},refinement={0:00},gamegraph={0:00},analysis={0:00},
                                yes=6.7,no=16.1,maybe=77.3,figure=cases-integrator-iteration3]
            \RefinementTableRow[iteration=4,polys=21,onestates=52,oneactions=104,twostates=66,twoactions=794,
                                total={0:01},refinement={0:00},gamegraph={0:00},analysis={0:00},
                                yes=6.7,no=17.1,maybe=76.2,figure=cases-integrator-iteration4]
        }
    }

    Negative attractor refinement.
    Iterate until convergence.
    Introduction into refinement progress reporting.

    Note that due to progress guarantee analysis is technically not necessary and procedure could be speed up.
    Analysis status of other states could be affected by refinement which can only be determined by an analysis.

    Show that simplification with multiple iterations of negative attractor does not generally work because of self-loops.

\stopsubsection



\startsubsection[title={Positive Robust Refinement},reference=sec:cases-integrator-positive]

    \placetable[top][tab:cases-integrator-positive-single]{
        TODO \par TODO
    }{
        \RefinementTable{
            \RefinementTableRow[iteration=5,polys=48,onestates=106,oneactions=476,twostates=407,twoactions=6188,
                                total={0:04},refinement={0:00},gamegraph={0:03},analysis={0:00},
                                yes=13.9,no=17.1,maybe=68.9,figure=cases-integrator-iteration5-prs]
            \RefinementTableRow[iteration=6,polys=166,onestates=342,oneactions=3723,twostates=3519,twoactions=79332,
                                total={2:07},refinement={0:02},gamegraph={1:59},analysis={0:03},
                                yes=36.5,no=17.1,maybe=46.4,figure=cases-integrator-iteration6-prs]
            \RefinementTableRow[iteration=7,polys=301,onestates=612,oneactions=5902,twostates=5459,twoactions=111705,
                                total={5:59},refinement={0:09},gamegraph={3:40},analysis={0:03},
                                yes=54.9,no=17.1,maybe=28.0,figure=cases-integrator-iteration7-prs]
            \RefinementTableRow[iteration=8,polys=362,onestates=734,oneactions=3096,twostates=2458,twoactions=54578,
                                total={9:41},refinement={0:07},gamegraph={3:33},analysis={0:02},
                                yes=67.0,no=17.1,maybe=15.9,figure=cases-integrator-iteration8-prs]}
    }

    \placetable[top][tab:cases-integrator-positive-double]{
        TODO \par TODO
    }{
        \RefinementTable{
            \RefinementTableRow[iteration=5,polys=68,onestates=146,oneactions=885,twostates=796,twoactions=13251,
                                total={0:11},refinement={0:00},gamegraph={0:10},analysis={0:00},
                                yes=18.4,no=17.1,maybe=64.5,figure=cases-integrator-iteration5-prm]
            \RefinementTableRow[iteration=6,polys=236,onestates=482,oneactions=6293,twostates=6003,twoactions=141782,
                                total={4:32},refinement={0:06},gamegraph={4:08},analysis={0:06},
                                yes=53.9,no=17.1,maybe=29.0,figure=cases-integrator-iteration6-prm]
            \RefinementTableRow[iteration=7,polys=333,onestates=676,oneactions=4114,twostates=3567,twoactions=53441,
                                total={6:09},refinement={0:06},gamegraph={1:29},analysis={0:02},
                                yes=77.0,no=17.1,maybe=5.9,figure=cases-integrator-iteration7-prm]
            \RefinementTableRow[iteration=8,polys=361,onestates=732,oneactions=1367,twostates=682,twoactions=7436,
                                total={6:18},refinement={0:01},gamegraph={0:08},analysis={0:00},
                                yes=82.5,no=17.1,maybe=0.3,figure=cases-integrator-iteration8-prm]
        }
    }

    \placetable[top][tab:cases-integrator-positive-suppressed]{
        TODO \par TODO
    }{
        \RefinementTable{
            \RefinementTableRow[iteration=5,polys=51,onestates=112,oneactions=507,twostates=435,twoactions=6147,
                                total={0:05},refinement={0:00},gamegraph={0:04},analysis={0:00},
                                yes=18.5,no=17.1,maybe=64.3,figure=cases-integrator-iteration5-prms]
            \RefinementTableRow[iteration=6,polys=99,onestates=208,oneactions=1360,twostates=1221,twoactions=21859,
                                total={0:26},refinement={0:02},gamegraph={0:18},analysis={0:01},
                                yes=42.4,no=17.1,maybe=40.4,figure=cases-integrator-iteration6-prms]
            \RefinementTableRow[iteration=7,polys=134,onestates=278,oneactions=1291,twostates=1069,twoactions=16174,
                                total={0:41},refinement={0:02},gamegraph={0:13},analysis={0:00},
                                yes=65.6,no=17.1,maybe=17.3,figure=cases-integrator-iteration7-prms]
            \RefinementTableRow[iteration=8,polys=161,onestates=332,oneactions=962,twostates=671,twoactions=8157,
                                total={0:48},refinement={0:01},gamegraph={0:05},analysis={0:00},
                                yes=80.1,no=17.1,maybe=2.7,figure=cases-integrator-iteration8-prms]
        }
    }

    Recreation of Svorenova positive refinement procedure, who refined wrt 3 random actions.
    Control selection for $\RefinePos$ is generally better than action-based, therefore, two single-step applications here.
    Also use PreR?

    Problem is already reachability and co-safe, therefore extraction of robust reachability problem trivial (transition $q_0$ to $q_1$).
    Positive refinement, two iterations per step, with target expansion.
    More progress per iteration.
    Should be a bit faster than 2x single-step, but similar state counts.
    Progress guarantee could be used to speed up, but this has not been implemented.
    It would only work if the target region is recognized as part of the yes states too, not in general.

    Positive refinement, two iterations per step, small state suppression.
    Additional iteration required but faster progress, much fewer states.
    Show comparison (super-small states).
    Similar result if 8 iterations are carried out directly.
    Again, for this reachability problem analysis time could be reduced by using progress guarantee, which small state-suppression delivers.
    Compare to convex hull overapproximation which gets stuck.

\stopsubsection


\startsubsection[title={Positive Robust Refinement with Layer Decomposition},reference=sec:cases-integrator-layered]

    \placetable[top][tab:cases-integrator-layered-original]{
        TODO \par TODO
    }{
        \RefinementTable{
            \RefinementTableRow[iteration=5,polys=166,onestates=342,oneactions=4364,twostates=4177,twoactions=79056,
                                total={1:24},refinement={0:03},gamegraph={1:16},analysis={0:05},
                                yes=76.3,no=17.1,maybe=6.6,figure=cases-integrator-iteration5-lprms]
        }
    }

    \placetable[top][tab:cases-integrator-layered-shrunk]{
        TODO \par TODO
    }{
        \RefinementTable{
            \RefinementTableRow[iteration=5,polys=129,onestates=268,oneactions=2565,twostates=2415,twoactions=36246,
                                total={0:29},refinement={0:01},gamegraph={0:25},analysis={0:02},
                                yes=81.8,no=17.1,maybe=1.0,figure=cases-integrator-iteration5-slprms]
        }
    }

    Positive robust refinement with layer decomposition, target expansion, small state suppression and PreR, 4 iterations.
    Show how limit behaviour causes small states to appear despite suppression.
    Instead, show with 95\% shrunk U ($\ClosedInterval{-0.95}{0.95}$).
    Limit behaviour gone, performance comparable to 9 iterations direct, but less costly refinement due to simpler robust sub-problems.
    Structure of layered refinement prescribes minimum number of states.

    Show unintuitive behaviour: only 2 layers at a time, faster time to solution despite more states.
    Reason: game simplification.
    Conclusion: progress guarantee should be used if it exists.

\stopsubsection


\startbuffer[buf:cases-integrator-results-statistics]
    \setupTABLE[frame=off,rightframe=on]
    \setupTABLE[c][first][rightframe=off]
    \setupTABLE[r][last][bottomframe=on]
    \bTABLE[align={middle,lohi}]
        \bTR[topframe=off]
            \bTH[width=0.06\textheight] \eTH
            \bTH[width=0.09\textheight] \eTH
            \bTH[width=0.17\textheight] single-step \times 2 \eTH
            \bTH[width=0.17\textheight] two-step \eTH
            \bTH[width=0.17\textheight] two-step \par no small \eTH
            \bTH[width=0.17\textheight] layers \par four-step \par no small \eTH
            \bTH[width=0.17\textheight] shrunk layers \par four-step \par no small \eTH
        \eTR
        \bTR[topframe=off]
            \bTH Iter. \eTH
            \bTH \eTH
            \bTD min / avg / max \eTD
            \bTD min / avg / max \eTD
            \bTD min / avg / max \eTD
            \bTD min / avg / max \eTD
            \bTD min / avg / max \eTD
        \eTR
        \bTR[topframe=on]
            \bTH[nr=3] 1-4 \par neg. \eTH
            \bTD polytopes \eTD
            \bTD[nc=5] 21 / 21 / 21 \eTD
        \eTR
        \bTR
            \bTD elapsed \eTD
            \bTD[nc=5] 0:01 / 0:01 / 0:01 \eTD
        \eTR
        \bTR
            \bTD \% decided \eTD
            \bTD[nc=5] 23.8 / 23.8 / 23.8 \eTD
        \eTR
        \bTR[topframe=on]
            \bTH[nr=3] 5 \eTH
            \bTD polytopes \eTD
            \bTD 48 / 50 / 53 \eTD
            \bTD 68 / 85 / 97 \eTD
            \bTD 49 / 54 / 60 \eTD
            \bTD 165 / 168 / 171 \eTD
            \bTD 128 / 130 / 133 \eTD
        \eTR
        \bTR
            \bTD elapsed \eTD
            \bTD 0:04 / 0:04 / 0:05 \eTD
            \bTD 0:11 / 0:21 / 0:36 \eTD
            \bTD 0:04 / 0:05 / 0:07 \eTD
            \bTD 1:19 / 1:24 / 1:27 \eTD
            \bTD 0:28 / 0:29 / 0:31 \eTD
        \eTR
        \bTR
            \bTD \% decided \eTD
            \bTD 30.7 / 31.0 / 31.4 \eTD
            \bTD 35.5 / 38.4 / 39.6 \eTD
            \bTD 35.5 / 36.7 / 39.2 \eTD
            \bTD 93.4 / 95.0 / 96.7 \eTD
            \bTD 99.0 / 99.0 / 99.0 \eTD
        \eTR
        \bTR[topframe=on]
            \bTH[nr=3] 6 \eTH
            \bTD polytopes \eTD
            \bTD 166 / 194 / 239 \eTD
            \bTD 227 / 277 / 317 \eTD
            \bTD 83 / 99 / 114 \eTD
            \bTD  \eTD
            \bTD  \eTD
        \eTR
        \bTR
            \bTD elapsed \eTD
            \bTD 1:55 / 4:00 / 11:56 \eTD
            \bTD 2:16 / 6:06 / 16:36 \eTD
            \bTD 0:18 / 0:25 / 0:36 \eTD
            \bTD - \eTD
            \bTD - \eTD
        \eTR
        \bTR
            \bTD \% decided \eTD
            \bTD 53.6 / 55.8 / 58.5 \eTD
            \bTD 70.7 / 74.8 / 79.4 \eTD
            \bTD 53.6 / 61.8 / 71.0 \eTD
            \bTD  \eTD
            \bTD  \eTD
        \eTR
        \bTR[topframe=on]
            \bTH[nr=3] 7 \eTH
            \bTD polytopes \eTD
            \bTD 239 / 280 / 321 \eTD
            \bTD 300 / 353 / 394 \eTD
            \bTD 124 / 132 / 137 \eTD
            \bTD  \eTD
            \bTD  \eTD
        \eTR
        \bTR
            \bTD elapsed \eTD
            \bTD 4:04 / 7:14 / 17:13 \eTD
            \bTD 3:10 / 7:24 / 17:59 \eTD
            \bTD 0:26 / 0:41 / 0:50 \eTD
            \bTD - \eTD
            \bTD - \eTD
        \eTR
        \bTR
            \bTD \% decided \eTD
            \bTD 69.3 / 72.0 / 73.3 \eTD
            \bTD 87.8 / 95.6 / 99.5 \eTD
            \bTD 76.2 / 83.8 / 89.8 \eTD
            \bTD  \eTD
            \bTD  \eTD
        \eTR
        \bTR[topframe=on]
            \bTH[nr=3,bottomframe=on] 8 \eTH
            \bTD polytopes \eTD
            \bTD 296 / 351 / 408 \eTD
            \bTD 333 / 389 / 435 \eTD
            \bTD 141 / 151 / 163 \eTD
            \bTD  \eTD
            \bTD  \eTD
        \eTR
        \bTR
            \bTD elapsed \eTD
            \bTD 6:15 / 12:35 / 27:24 \eTD
            \bTD 3:26 / 8:53 / 18:07 \eTD
            \bTD 0:31 / 0:47 / 0:58 \eTD
            \bTD - \eTD
            \bTD - \eTD
        \eTR
        \bTR
            \bTD \% decided \eTD
            \bTD 81.6 / 84.3 / 85.6 \eTD
            \bTD 95.5 / 99.2 / 99.8 \eTD
            \bTD 89.9 / 96.5 / 99.3 \eTD
            \bTD  \eTD
            \bTD  \eTD
        \eTR
    \eTABLE
\stopbuffer

\startsubsection[title={Results},reference=sec:cases-integrator-results]

    % Rotation value has to be flipped for some reason (maybe because it refers
    % to the page that includes the figure, not the figure itself?
    % % TODO check rotation in final document
    \placetable[here,\doifoddpageelse{270}{90}][tab:cases-integrator-results-statistics]{
        TODO
    }{
        \startframedtext[width=\textheight,offset=0mm,frame=off,topframe=off,bottomframe=off]
            \getbuffer[buf:cases-integrator-results-statistics]
        \stopframedtext
    }

    Show table with average numbers for all problems.
    Talk about randomization.
    Give numbers as percent progress per second?
    Layered refinement performs best, importance of small-state suppression (combats over-refinement) and problems with jaggedness (which just causes more jaggedness).
    Plot number of state space partition elements vs. game size for full analysis.
    Show how product game simplification can save time.
    Illustrated also by 2-layers at a time experiment.

\stopsubsection

