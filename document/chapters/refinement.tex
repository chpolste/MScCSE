The game-based abstraction requires a state space partition from which to derive player 1 states and actions.
The initial partition described in \in{Section}[sec:abstraction-graph-playerone] is governed by the set of linear predicates $\Predicates$.
This partition is convenient for the synchronous product construction because it creates a unique association between its parts and the linear predicates satisfied by them.
However, this partition will generally be too coarse to fully solve the analysis problem for the given objective, with product game states remaining in the undecided set $P_\Maybe$ after the first analysis.

In this chapter, heuristic procedures are presented that refine the state space partition.
Ideally, these procedures should enable a solution or at least a partial solution after some iterations of the abstraction-analysis-refinement cycle (\in{Figure}[fig:problem-approach-flowchart]).

