In order to define states and actions for the game abstraction of the LSS, meaningful operators on the state- and control space have to be introduced that reflect different aspects of the system dynamics.
The now operators require knowledge of the parameters of the LSS, i.e.\ $\MatA$, $\MatB$, $\StateSpace$, $\RandomSpace$, $\ControlSpace$.
The association of operator and LSS is not written out explicitly.
It will always be clear from context which system an operator belongs to.

The following sections introduce the operators together with bried descriptions of how to compute them in the polytopic setting assumed here.


\startsubsection[title={Posterior}]

    The posterior ($\Post$) is a forward looking operator, yielding the set of states that can be reached from an origin region $\StateRegion \subseteq \StateSpace$ under control inputs $\ControlSpace' \subseteq \ControlSpace$ with non-zero probability, i.e.\ the one-step reachable set:

    \startformula
        \Posterior{\StateRegion}{\ControlRegion} := \Set{ \VecState \in \reals^n \mid \exists \VecState' \in \StateRegion\MidComma \exists \VecControl' \in \ControlRegion\MidComma \exists \VecRandom \in \RandomSpace : \VecState = \MatA \VecState' + \MatB \VecControl' + \VecRandom } \EndPeriod
    \stopformula

    The posterior allows the computation of the extended state space $\ExtendedStateSpace = \StateSpace \,\cup\, \Posterior{\StateSpace}{\ControlSpace}$.

    While the above definition takes sets as arguments, the notation of the posterior and, where sensible, all following operators is overloaded for vector-valued state- and control space arguments as well as sets of elements of a state space decomposition.
    Let $\VecState' \in \StateSpace$ and $\State{j} \in \StateSpace$ for all $j \in J$, then

    \startformula
        \startalign[n=2,align={right,left}]
            \NC \Posterior{\VecState'}{\ControlRegion} :=
            \NC \Posterior{\Set{\VecState'}}{\ControlRegion} \EndComma
            \NR
            \NC \Posterior{\StateRegion}{\VecControl'} :=
            \NC \Posterior{\StateRegion}{\Set{\VecControl'}} \EndAnd
            \NR
            \NC \Posterior{\IndexedStates{j}{J}}{\ControlRegion} :=
            \NC \BigPosterior{\bigcup_{j \in J}\State{j}}{\ControlRegion} \EndPeriod
            \NR
        \stopalign
    \stopformula

    For polytopic inputs, $\Post$ can be computed with the Minkowski sum as

    \startformula
        \Posterior{\StateRegion}{\ControlSpace'} = \MatA \StateRegion \oplus \MatB \ControlRegion \oplus \RandomSpace \EndPeriod
    \stopformula

\stopsection


\startsubsection[title={Predecessors}]

    As the name suggests, the predecessor operators are backward looking, yielding different kinds of origin regions for a given target and control input.
    First, the predecessor ($\Pre$) and robust predecessor ($\PreR$) are defined as

    \startformula
        \startalign[n=2,align={right,left}]
            \NC \Predecessor{\StateRegion}{\ControlSpace'}{\StateTarget} :=
            \NC \Set{ \VecState' \in \StateRegion \mid \exists \VecControl' \in \ControlSpace' : \Posterior{\VecState'}{\VecControl'} \cap \StateTarget \neq \emptyset } \EndAnd
            \NR
            \NC \RobustPredecessor{\StateRegion}{\ControlSpace'}{\StateTarget} :=
            \NC \Set{ \VecState' \in \StateRegion \mid \exists \VecControl' \in \ControlSpace' : \Posterior{\VecState'}{\VecControl'} \subseteq \StateTarget } \EndComma
            \NR
        \stopalign
    \stopformula

    where the second argument $\ControlRegion \subseteq \ControlSpace$ is the considered control space region, the third parameter $\StateRegion \subseteq \StateSpace$ is a target region in the extended state space and the first argument $\StateTarget \subseteq \ExtendedStateSpace$ is a region of the state space to which the returned predecessor is restricted and exists mainly for convenience.
    From any state in the computed origin region of $\Pre$, the target region can be reached with non-zero probability in the next step for some control input in the given set.
    It should be noted that the control input that enables the transition can (and likely will) be different for every state in the predecessor.

    The probability of reaching the target region in case of $\PreR$ is 1, meaning that the robust predecessor is \quotation{robust} in the sense that irrespective of the stochastic dynamics, the target can be reached from the computed origin region.
    Trivially, it holds that

    \startformula
        \RobustPredecessor{\StateRegion}{\ControlSpace'}{\StateTarget} \subseteq \Predecessor{\StateRegion}{\ControlSpace'}{\StateTarget} \EndPeriod
    \stopformula

    For target regions where $\StateTarget \ominus \RandomSpace = \emptyset$, the $\PreR$ is always empty.
    This can easily be seen from the computations for polytopic regions

    \startformula
        \startalign[n=2,align={right,left}]
            \NC \Predecessor{\StateRegion}{\ControlSpace'}{\StateTarget} =
            \NC \StateRegion \cap \Big( \StateTarget \oplus -(\MatB \ControlRegion \oplus \RandomSpace) \Big) \MatA \EndAnd
            \NR
            \NC \RobustPredecessor{\StateRegion}{\ControlSpace'}{\StateTarget} =
            \NC \StateRegion \cap \Big( (\StateTarget \ominus \RandomSpace) \oplus -\MatB \ControlRegion \Big) \MatA \EndPeriod
            \NR
        \stopalign
    \stopformula

    Finally, the precise predecessor is defined, an operator that only makes sense with a state space partition:

    \startformula
        \startalign[n=2,align={right,left}]
            \NC \PrecisePredecessor{\StateRegion}{\ControlSpace'}{\IndexedSet{\State{j}}{j \in J}} := \Big\{ \VecState \in \StateRegion \Bigmid \exists \VecControl' \in \ControlSpace' :
            \NC \Posterior{\VecState}{\VecControl'} \subseteq \bigcup_{j \in J} \State{j} \;\text{and}
            \NR
            \NC \empty
            \NC ~\forall j \in J : \Posterior{\VecState}{\VecControl'} \cap \bigcup_{j \in J} \State{j} \neq \emptyset \Big\} \EndPeriod
            \NR
        \stopalign
    \stopformula

    Every state in the resulting origin set fulfills the robust predecessor property with respect to the entire target region $\bigcup_{j \in J} \State{j}$ for some $\VecControl' \in \ControlRegion$, while simultaneously fulfilling the predecessor property with respect to every individual part of the target region for the same $\VecControl'$.
    In other words, for any state in a $\PreP$ set a control input in $\ControlRegion$ exists such that a state in $\bigcup_{j \in J} \State{j}$ is reached almost surely after one step and the proability of ending up in any one of the target region parts $\IndexedStates{j}{J}$ is non-zero.
    The precise predecessor can be computed as

    \startformula
        \RobustPredecessor{\StateRegion}{\ControlSpace'}{\IndexedSet{\State{j}}{j \in J}} =
        \Big( \Big( \bigcap_{j \in J} (\State{j} \oplus - \RandomSpace) \setminus \bigcup_{j \in I \setminus J} ( \State{j} \oplus - \RandomSpace ) \Big) \oplus -\MatB \ControlSpace' \Big) \MatA \EndPeriod
    \stopformula

\stopsection


\startsubsection[title={Attractors}]

    Analogous to the predecessor and robust predecessor operators, the attractor ($\Attr$) and robust attractor $\AttrR$ are defined as

    \startformula
        \startalign[n=2,align={right,left}]
            \NC \Attractor{\StateRegion}{\ControlSpace'}{\StateTarget} :=
            \NC \Set{ \VecState \in \StateRegion \mid \forall \VecControl' \in \ControlSpace' : \Posterior{\VecState}{\VecControl'} \cap \StateTarget \neq \emptyset } \EndAnd
            \NR
            \NC \RobustAttractor{\StateRegion}{\ControlSpace'}{\StateTarget} :=
            \NC \Set{ \VecState \in \StateRegion \mid \forall \VecControl' \in \ControlSpace' : \Posterior{\VecState}{\VecControl'} \subseteq \StateTarget } \EndComma
            \NR
        \stopalign
    \stopformula

    i.e. the predecessor properties hold for all control inputs in $\ControlRegion$ for the attractors, not just some.
    Therefore, it holds that

    \startformula
        \startalign[n=2,align={right,left}]
            \NC \Attractor{\StateRegion}{\ControlSpace'}{\StateTarget} \subseteq
            \NC \Predecessor{\StateRegion}{\ControlSpace'}{\StateTarget} \EndComma
            \NR
            \NC \RobustAttractor{\StateRegion}{\ControlSpace'}{\StateTarget} \subseteq
            \NC \RobustPredecessor{\StateRegion}{\ControlSpace'}{\StateTarget} \EndAnd
            \NR
            \NC \RobustAttractor{\StateRegion}{\ControlSpace'}{\StateTarget} \subseteq
            \NC \Attractor{\StateRegion}{\ControlSpace'}{\StateTarget} \EndPeriod
            \NR
        \stopalign
    \stopformula

    The attractors can be computed from the predecessor operators as

    \startformula
        \startalign[n=2,align={right,left}]
            \NC \Attractor{\StateRegion}{\ControlSpace'}{\StateTarget} =
            \NC \StateRegion \setminus \RobustPredecessor{\StateRegion}{\ControlSpace'}{\ExtendedStateSpace \setminus \StateTarget} \EndComma
            \NR
            \NC \RobustAttractor{\StateRegion}{\ControlSpace'}{\StateTarget} =
            \NC \StateRegion \setminus \Predecessor{\StateRegion}{\ControlSpace'}{\ExtendedStateSpace \setminus \StateTarget} \EndPeriod
            \NR
        \stopalign
    \stopformula

\stopsection


\startsubsection[title={Actions}]

    The last operators defined are control space operators, i.e.\ the resulting sets are subsets of $\ControlSpace$.
    The action ($\Act$) and robust action ($\ActR$), given by

    \startformula
        \startalign[n=2,align={right,left}]
            \NC \Action{\StateRegion}{\StateTarget} =
            \NC \Set{ \VecControl \in \ControlSpace \mid \Posterior{\StateRegion}{\VecControl} \cap \StateTarget \neq \emptyset } \EndAnd
            \NR
            \NC \RobustAction{\StateRegion}{\StateTarget} =
            \NC \Set{ \VecControl \in \ControlSpace \mid \Posterior{\StateRegion}{\VecControl} \subseteq \StateTarget } \EndComma
            \NR
        \stopalign
    \stopformula

    return the sets of control inputs with which the probability of transitioning to the target region $\StateTarget$ from origin region $\StateRegion$ in one step is non-zero and 1, respectively.
    They are adapted from the non-probabilistic setting of \cite[Yordanov2009], who defined the operators $U^{X \rightarrow Y}$ corresponding to $\Action{X}{Y}$ and $U^{X \Rightarrow Y}$ corresponding to $\RobustAction{X}{Y}$.
    Both operators can be computed using Minkowski sum and Pontryagin difference as

    \startformula
        \startalign[n=2,align={right,left}]
            \NC \Action{\State{i}}{\State{j}} =
            \NC \Big( ( \State{j} \oplus - (\MatA \State{i} \oplus \RandomSpace) ) \MatB \Big) \cap \ControlSpace \EndAnd
            \NR
            \NC \RobustAction{\State{i}}{\State{j}} =
            \NC \Big( ( \State{j} \ominus (\MatA \State{i} \oplus \RandomSpace) ) \MatB \Big) \cap \ControlSpace \EndPeriod
            \NR
        \stopalign
    \stopformula

    Due to implementation constraints, the results of the operators can only be non-empty full-dimensional polytopic regions.
    Lower-dimensional control space regions are therefore not computable by the action operators.
    This means e.g.\ that it is possible that for some $\VecState \in \Predecessor{\StateRegion}{\ControlSpace}{\StateTarget}$ $\Action{\VecState}{\StateTarget} = \emptyset$, even though this should be true theoretically (see section \in[sec:abstraction-analysis-correctness]).

    Finally, the concrete action ($\ActC$) is introduced.
    As for the precise predecessor, a state space partition is required for the operator to be meaningful:

    \startformula
        \startalign[n=2,align={right,left}]
            \NC \ConcreteAction{\StateRegion}{\IndexedSet{\State{j}}{j \in J}} := \Big\{ \VecControl \in \ControlSpace \Bigmid
            \NC \Posterior{\StateRegion}{\VecControl} \subseteq \bigcup_{j \in J} \State{j} \;\text{and}
            \NR
            \NC \empty
            \NC ~\forall j \in J : \Posterior{\StateRegion}{\VecControl} \cap \State{j} \neq \emptyset \Big\} \EndPeriod
            \NR
        \stopalign
    \stopformula

    Under the resulting control inputs, only $\StateTarget$ can be reached in one step from $\StateRegion$ (first condition) and for every part in the target region a state in $\StateRegion$ can be found so that the probability of reaching the part is non-zero.
    It should be noted that the conditions of $\PreP$ and $\ActC$ are quite different from one another:
    The precise predecessor requires some control input for every of its states for which both conditions (subset and part intersection) are fulfilled.
    The concrete action requires that for every of its control vectors the subset condition is fulfilled for every origin state while the part intersections can be fulfilled independently by different states in $\StateRegion$, as long as every part is reachable from some state.
    This implies that,

    \startformula
        \ConcreteAction{\StateRegion}{\IndexedStates{j}{J}} \,\cap\, \ConcreteAction{\StateRegion}{\IndexedStates{j}{J'}} = \emptyset
    \stopformula

    for all $J \neq J'$.
    The same does not hold for precise predecessors, which can overlap for different target sets.
    Concrete actions can therefore be computed directly from actions with

    \startformula
        \startalign[n=2,align={right,left}]
            \NC \ConcreteAction{\StateRegion}{\IndexedStates{j}{J}} =
            \NC \bigcap_{j \in J} \Action{\StateRegion}{\State{j}} \setminus \bigcup_{j \in I \setminus J} \Action{\StateRegion}{\State{j}} \EndPeriod
            \NR
        \stopalign
    \stopformula

    Note that \cite[Svorenova2017] use the notation $U_i^J$ for the set $\ConcreteAction{\State{i}}{\IndexedStates{j}{J}}$.

\stopsection

