Restriction of this work: consideration of only low-dimensional problems with simple specifications.
While it was argued in section \in[sec:TODO], that this should not restrict the applicability of results to higher dimensions and more complex objectives, it would still be interesting to see the procedure applied to more involved problems.
In particular, performance for high-dimensional state spaces and investigation into decomposition approaches for complex automata are of interest.
Problems in robotics where the state space is a physical phase space as in the pendulum case study easily have 6 dimensions (3 location, 3 velocity) or more.
Expectation is that higher dimensions quickly eat up improvements in refinement efficiency (see corridor vs double integrator game size with only one dimension difference in control space).

Limitation of full dimensional polytopes is one source of non-termination.
It would be interesting to see if it is possible to identify such behaviour analytically.
It could be possible to design rounding approaches to resolve the issue.
The problem is only an artefact of implementation/desire to have robust (non-empty) control inputs, solution can be computed with arbitrary precision by reducing size of $\epsilon$-ball.
One could just decide if the limit-region should be satisfying or not and round to force termination of the procedure overall and accelerate refinement, where the issue occurs as well.

GR(1) restriction here was mainly for convenience of solving game.
Nothing stands in the way of full LTL if a sufficiently potent game sover is introduced, as noted already by \cite[Svorenova].
Care has be be taken when applying transition refinement to objectives that contain the next operator but everything else is transferable in a straightforward manner.

Refinement that takes probabilistic aspects of system fully into account.
Optimization of refinement for different partition qualities (least states, fastest time to solution etc.).
United view of refinement for purposes of analysis and controller synthesis.
Latter was barely touched on here and shows similar problems.
Key could be transition-based decomposition an chaining of finite strategies.
Further investigation in the context of stutter-equivalent traces.
Self-loop removal showed potential as a refinement technique, but approach of \cite[TODO] is more sophisticated, does not require introduction of additional states and its addition to the procedure could be a valuable asset for reduction of computational complexity.

%TODO: Is approach compatible with quantitative analysis a la BMDPs? Too ambitious?

