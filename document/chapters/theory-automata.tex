A language is a collection of words, formed from an alphabet of symbols according to a set of rules.
Finite automata are transition systems with which such rules can be expressed.


\startsubsection[title={\omega-Automata},reference=sec:theory-automata-omega]

    A deterministic \omega-automaton is a 5-tuple 

    \startformula
        \Automaton = (Q, \Alphabet, \Transition, q_{\mathrm init}, \Condition) \EndComma
    \stopformula

    where
    $Q \neq \emptyset$ is a finite set of states,
    $\Alphabet$ is a finite alphabet of symbols,
    $\Transition: Q \times \Alphabet \rightarrow Q$ is a deterministic transition relation and
    $q_{\mathrm init} \in Q$ is the initial state of the automaton.
    If the transition relation is defined for every combination of state in $Q$ and symbol in $\Alphabet$, the automaton is called complete, otherwise it is called incomplete.

    A run is an infinite sequence of states $q_0 q_1 ... \in Q^\omega$ such that $q_0 = q_init$ and $q_{i+1} = \delta(q_i, w_i)$, $w_i \in \Alphabet$ for all $i \in \naturalnumbers_0$.
    Because the transitions of $\Automaton$ are deterministic, every run is uniquely associated with a word $w = w_0 w_1 ... \in \Alphabet^\omega$ through the transition labels from $\Sigma$ assigned by $\Transition$.
    The language $\Language(\Automaton)$ of an automaton $\Automaton$ is defined as the set of all words whose corresponding runs are accepted by the acceptance condition $\Condition$.
    Note that a \quotation{run} refers to a sequence of automaton states, a \quotation{word} to sequence of symbols from an alphabet, a \quotation{trace} to a sequence of states of an LSS and a \quotation{play} is sequence of states of a game.

\stopsubsection


\startreusableMPgraphic{theory-automaton-example}
    beginfig(0);
        with spacing((30,10)) matrix.a(3,8);
        node_dash.a[1][1](btex $q_0$ etex);
        node.a[1][4](btex $q_1$ etex);
        node_double.a[1][7](btex $q_2$ etex);
        incoming(0, "") (a[1][1]) 180;
        loop.rt(.4, btex \;$a$ etex) (a[1][1]) 90;
        arrow.top(.5, btex $b$ etex) (a[1][1],a[1][4]) a[1][1].c..a[1][4].c;
        arrow.top(.5, btex $c$ etex) (a[1][4],a[1][7]) a[1][1].c..a[1][7].c;
        loop.rt(.4, btex \;$d$ etex) (a[1][7]) 90;
    endfig;
\stopreusableMPgraphic

\startsubsection[title={Acceptance Conditions},reference=sec:theory-automata-acceptance]

    An acceptance condition $\Condition$ specifies the set of all accepted runs.
    If a run is not accepted, it is rejected.
    The set of all accepted runs is generally an infinite set.
    The acceptance condition is therefore usually expressed by other means than an enumeration of runs from the accepted set.
    A Büchi acceptance condition is defined by a set of states of which at least one has to be visited infinitely often in a run or the run is rejected.
    A Büchi-implication condition, also known as a one-pair Streett condition, is specified by a tuple of sets $\mathcal{C} = (E, F) \subseteq Q \times Q$ and accepts all runs $r$ satisfying

    \startformula
        (\InfinitelyOften{r} \cap E \ne \emptyset) \Longrightarrow (\InfinitelyOften{r} \cap F \ne \emptyset) \EndComma
    \stopformula

    where the set of states which occur infinitely often in the run $r$ is denoted by $\InfinitelyOften{r}$.
    If any state from $E$ occurs infinitely often in a run, a state from $F$ has to occur infinitely often as well otherwise the run is rejected.
    The condition is trivially fulfilled if $E$ is empty.
    If $F$ is empty, all runs in which a state from $E$ occurs infinitely often are rejected.

    \placefigure[top][fig:theory-automaton-example]{
        A deterministic, incomplete \omega-automaton with a one-pair Streett acceptance condition $\Tuple{E}{F} = \Tuple{\Set{q_0}}{\Set{q_2}}$.
        States from $E$ are highlighted with a dashed border and states from $F$ with a double border.
        The language of the automaton consists of all words of the form $a^\ast b c d^\omega$.
    }{
        % Put in wide box so that figure caption has proper width
        \framed[width=\textwidth,frame=off]{\reuseMPgraphic{theory-automaton-example}}
    }

    Figure \in[fig:theory-automaton-example] shows the automaton

    \startformula
        (\Set{q_0, q_1, q_2}, \Set{a, b, c, d}, \Transition, q_0, (\Set{q_0}, \Set{q_2}))
    \stopformula

    with a one-pair Streett acceptance condition.
    The transition relation defines $\delta(q_0, a) = q_0$, $\delta(q_0, b) = q_1$, $\delta(q_1, c) = q_2$ and $\delta(q_2, d) = q_2$.
    The automaton is therefore incomplete.
    This automaton accepts all runs that stay in $q_0$ for a finite number of steps before transitioning to $q_1$ and then immediately to $q_2$, where they stay forever.
    The corresponding language of the automaton contains all words starting with a finite number of $a$s, followed by a single $b$, a single $c$ and then $d$s forever.
    Words of this form are written as $a^\ast b c d^\omega$, where $a^\ast$ denotes any finite repetition of $a$ and $d^\omega$ the infinite repetition of $d$.

\stopsubsection

