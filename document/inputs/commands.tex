% CamelCase for custom commands

\define\DocTitleFooter{Abstraction Refinement for Continuous-Space MDPs} % TODO
\define\Author{Christopher Polster}
\define\YearOfCompletion{2019}
\define\DateOfCompletion{\YearOfCompletion-??-??} % TODO

% Vertical position of default coloncolon command is not right
\definemathcommand[coloncolon][rel]{\colon\!\colon}
% Tall version of \mid
\definemathcommand[Bigmid][rel]{\mathrel{\Big|}}

% Equation endings
\define\EndComma{\;\text{,}}
\define\EndPeriod{\;\text{.}}
\define\EndAnd{\;\text{and}}


% General math
% ------------

% 2-Norm
\define[1]\TwoNorm{\lVert{#1}\rVert}

% Vector variables with bold font
\define[1]\Vec{{\bf #1}}
\define\VecC{\Vec{c}}
\define\VecU{\Vec{u}}
\define\VecV{\Vec{v}}
\define\VecX{\Vec{x}}
\define\VecY{\Vec{y}}
\define\VecZ{\Vec{z}}

% Matrix variables with bold font
\define[1]\Mat{{\bf #1}}
\define\MatA{\Mat{A}}
\define\MatB{\Mat{B}}
\define\MatU{\Mat{U}}

% Vectors and matrices with square brackets
\definemathmatrix[sqmatrix][left={\left\lbrack\,},right={\,\right\rbrack},strut=0.8,distance=0.8em]
\define[4]\TwoByTwo{ \startsqmatrix[n=2,align={middle,middle}] \NC #1 \NC #2 \NR \NC #3 \NC #4 \NR \stopsqmatrix }
\define[2]\TwoByOne{ \startsqmatrix[n=1,align={middle}] \NC #1 \NR \NC #2 \NR \stopsqmatrix }
\define[2]\OneByTwo{ \startsqmatrix[n=2,align={middle,middle}] \NC #1 \NC #2 \NR \stopsqmatrix }
\define[1]\OneByOne{ \startsqmatrix[n=1,align={middle}] \NC #1 \NR \stopsqmatrix }

% Inline vectors and matrices with square brackets (https://www.mail-archive.com/ntg-context@ntg.nl/msg78899.html)
\definemathmatrix[ssqmatrix][left=\left\lbrack,right=\right\rbrack,style=\scriptstyle,strut=0.5,distance=0.5em]
\define[2]\TwoByOneSmall{ \startssqmatrix[n=1,align={middle}] \NC #1 \NR \NC #2 \NR \stopssqmatrix }

% Sets and Intervals
\define[1]\Set{\{ {#1} \}}
\define[1]\BigSet{\Big\{ {#1} \Big\}}
\define[2]\IndexedSet{\Set{#1}_{#2}}
\define[2]\ClosedInterval{\left[\, #1,\, #2 \,\right]}


% Polytopic Computations
% ----------------------

% Custom function names
\definemathcommand[Hull][nolop]{\mfunction{hull}}
\definemathcommand[Vertices][nolop]{\mfunction{vert}}
\definemathcommand[Post][nolop]{\mfunction{Post}}
\definemathcommand[Pre][nolop]{\mfunction{Pre}}
\definemathcommand[PreR][nolop]{\mfunction{PreR}}
\definemathcommand[PreP][nolop]{\mfunction{PreP}}
\definemathcommand[Attr][nolop]{\mfunction{Attr}}
\definemathcommand[AttrR][nolop]{\mfunction{AttrR}}

% States
\define\StateSpace{X}
\define[1]\State{X_{#1}}

% Control
\define\ControlSpace{U}
\define[2]\Action{U_{#1}^{\{#2\}}}
\define[2]\ActionPolytope{U^{#1 \rightarrow #2}}

% Random
\define\RandomSpace{W}

% Polytopic Operators
\define[2]\Posterior{\Post({#1},\, {#2})}
\define[3]\Predecessor{\Pre({#1},\, {#2},\, {#3})}
\define[3]\RobustPredecessor{\PreR({#1},\, {#2},\, {#3})}
\define[3]\PrecisePredecessor{\PreP({#1},\, {#2},\, {#3})}
\define[3]\Attractor{\Attr({#1},\, {#2},\, {#3})}
\define[3]\RobustAttractor{\AttrR({#1},\, {#2},\, {#3})}


% Logic and Automata
% ------------------

\definemathcommand[Inf][nolop]{\mfunction{Inf}}

\define\True{{\mathss true}}
\define\Next{{\mathss X}}
\define\Until{{\mathss U}}
\define\Finally{{\mathss F}}
\define\Globally{{\mathss G}}

\define[1]\InfinitelyOften{\Inf(#1)}

