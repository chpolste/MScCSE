\startreusableMPgraphic{mp:problem-approach-flowchart}
    with spacing((25,10)) matrix.a(11,15);
    with shape(none) node.a[0][0](btex \ssd \strut LSS etex);
    with shape(none) node.a[0][3](btex \ssd \strut Linear Predicates etex);
    with shape(none) node.a[0][9](btex \ssd \strut Objective etex);
    with shape(circle) with size(3) node.a[1][0]("");
    with shape(circle) with size(3) node.a[1][2]("");
    with shape(circle) with size(3) node.a[1][9]("");
    with shape(boxed) with border(none) node.a[3][12]("");
    with shape(boxed) with border(none) node.a[3][12]("");
    with shape(boxed) with border(none) node.a[5][12]("");
    with fixedboxwidth(80) with fixedboxheight(30) with shape(roundfixedbox) with filling(solid) with fillingcolor(lightgray) node.a[4][1](btex \ss Abstraction etex);
    with fixedboxwidth(80) with fixedboxheight(30) with shape(roundfixedbox) with filling(solid) with fillingcolor(lightgray) node.a[4][9](btex \ss Analysis etex);
    with fixedboxwidth(80) with fixedboxheight(30) with shape(roundfixedbox) with filling(solid) with fillingcolor(lightgray) node.a[10][5](btex \ss Refinement etex);
    arrow.rt(.5, "") (a[1][0],a[4][1]) a[1][0].c..a[4][0].c;
    arrow.rt(.5, "") (a[1][2],a[4][1]) a[1][2].c..a[4][2].c;
    arrow.rt(.5, "") (a[1][9],a[4][9]) a[1][9].c..a[4][9].c;
    arrow.top(.5, btex \ssd Game Graph etex) (a[4][1],a[4][9]) a[4][1].c..a[4][9].c;
    arrow.bot(.5, btex \ssd Undecided States etex) (a[4][9],a[10][5]) a[4][9].c---a[10][9].c---a[10][5].c;
    arrow.bot(.5, btex \ssd State Space Partition etex) (a[10][5],a[4][1]) a[10][5].c---a[10][1].c---a[4][1].c;
    arrow.rt(1, btex \ssd Part of ${\InitialStates}$ etex) (a[4][9],a[3][12]) a[3][9].c---a[3][12].c;
    arrow.rt(1, btex \ssd Part of ${\StateSpace \setminus \InitialStates}$ etex) (a[4][9],a[5][12]) a[5][9].c---a[5][12].c;
\stopreusableMPgraphic


The analysis and synthesis problems for the given problem setup have been solved by \cite[Svorenova2017].
An overview of their approach is depicted in Figure \in[fig:problem-approach-flowchart] and summarized here.
The individual steps are reviewed in detail and, in case of refinement, extended by novel additions in the following chapters.

First, the state space of the linear stochastic system (\in[eq:problem-setup-lsseq]) is partitioned into a finite set of disjunct, convex and polytopic equivalence classes.
To compensate for the lack of knowledge of precise system state due to the introduction of these equivalence classes, a second player has to be introduced to MDP of the LSS.
The result is a 2½-player game abstraction of the LSS built from the state space partitioning and polytopic operations in the state- and control space relating to the dynamics of the system.
The initial partition is induced by the set of linear predicates used in the objective specification.
This is the \quotation{Abstraction} phase.

Next, the synchronous product of the game graph and the \omega-automaton representing the objective is constructed.
The resulting game's winning condition is transferred from the automaton and, for practical purposes, transformed into a parity-3 objective.
The product game is then analysed once for an adversarial player 2 and once for a cooperative player 2.
From both solutions, parts of the state space abstraction can be identified as belonging to $\InitialStates$ or $\StateSpace \setminus \InitialStates$.
However, generally, some parts can remain undecided due to a coarse abstraction.
Therefore, refinement is applied to the undecided parts of the state space partition.
With heuristic methods based on the analysis results and system dynamics, a new state space partition is constructed and the procedure returns to the \quotation{Abstraction} phase.

The procedure is constructed such that after every iteration, a partial solution to the problem is obtained \cite[authoryears][Svorenova2017].
That means once a part of the state space has been identified as a part of $\InitialStates$ or $\StateSpace \setminus \InitialStates$, this identification is provably correct and will not change in subsequent analyses or after refinement.
If the procedure terminates, the analysis problem from \in[sec:problem-statement-analysis] is solved.
However, refinement heuristics can only provide progress guarantees in some circumstances and there are systems for which this procedure will provably never terminate (see TODO).

Finally, a controller can be synthesized based on the game graph and analysis results.
\cite[Svorenova2017] presented a non-randomized (i.e.\ pure), finite-memory controller, but \quotation{better} controllers can be constructed.
For example, memoryless controllers can be synthesized for reachability objectives through deeper analysis of the game graph \cite[authoryears][TODO].

\placefigure[top][fig:problem-approach-flowchart]{
    Flowchart representation of the solution approach described in section \in[sec:problem-approach].
    $\InitialStates$ is the set of states for which a controller can guarantee satisfaction of the objective when a trace originates from within.
    Adapted from Fig.\ 1 of \cite[Svorenova2017] with inspiration from Fig.\ 1 of \cite[Lahijanian2015].
}{
    \framed[width=\textwidth,frame=off]{\reuseMPgraphic{mp:problem-approach-flowchart}}
}

