The goal of model checking is to verify with an automatic procedure if some system meets a specification.
To achieve this, the two components \quotation{system} and \quotation{specification} have to be represented in a way that they can be brought together and then analysed.
Typically the system is abstracted to a system model, able to reproduce all behaviour required to accurately verify the properties of interest.
For the specification, a formulation in logic, usually a temporal logic, is sought.
The specifications may concern qualitative (e.g. \quotation{can something bad happen?}) or quantitative questions (e.g. \quotation{will something bad happen with more than 90\% probability in the first hour of operation?}) about the system.

Model checking is a brute-force approach in which the system model is explored exhaustively, i.e. every possible system state is considered during the verification procedure.
This approach ensures that no malicious behaviour of the system is missed and is also able to produce counterexamples when a system model fails to meet the specification.
Counterexamples guide the search for system errors making them a valuable part of the verification process.
However, the exhaustive approach means that substantial resources are required to carry out the desired analysis.
Appropriate and expressive abstractions for the system and specification lead to a large number of possible states that need to be checked, often growing exponentially with the complexity of the abstractions.
Dealing with this so-called state space explosion is a major task when designing a model checking procedure.
To combat state space explosion and ensure the feasability of finding solutions, abstractions have to be chosen with care and effective state space exploration techniques must be used.

Additionally, the advent of increasingly powerful computers has brought larger and larger systems into the reach of formal verification by model checking.
Since the beginning of model checking by TODO and TODO, procedures for a multitude of kinds of systems and specifications have been devised and are actively being reasearched. % TODO inventors of model checking
Model checking today has widespread applications from detecting deadlocks in concurrent programs to robot motion planning to hardware and security protocol verification. %TODO some e.g. references

Next, the concept of transition systems, which are the foundation of virtually every system abstraction seen in model checking, is introduced.

