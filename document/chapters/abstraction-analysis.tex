Winning condition is parity 3.
Equivalent to one-pair Streett and can be easily translated.
 and therefore assign parity 0 to ..., parity 1 to ..., parity 2 to ....


\startbuffer[buf:abstraction-analysis-solution-algorithm]
    \startalgorithmic
        \FUNCTION{SolveParity3}{$\ProductGame = (TODO)$: Game}
            \STATE{ $X, {\bar X}, Y, {\bar Y}, Z, {\bar Z}$: Set}
            \STATE{ ${\bar X} \leftarrow S$ }
            \STATE{ ${\bar Y} \leftarrow S$ }
            \STATE{ ${\bar Z} \leftarrow \emptyset$ }
            \DO
                \STATE{ $X \leftarrow {\bar X}$ }
                \DO
                    \STATE{ $Y \leftarrow {\bar Y}$ }
                    \DO
                        \STATE{ $Z \leftarrow {\bar Z}$ }
                        \STATE{ ${\bar Z} \leftarrow (F \cap \Pre_1(X)) \cup (E \cap \Pre_2(X, Y)) \cup (D \cap \Pre_3(X, Y, Z))$ }
                    \ENDDOWHILE{ $Z \ne {\bar Z}$ }
                    \STATE{ ${\bar Y} \leftarrow Z$ }
                    \STATE{ ${\bar Z} \leftarrow S$ }
                \ENDDOWHILE{ $Y \ne {\bar Y}$ }
                \STATE{ ${\bar X} \leftarrow Y$ }
                \STATE{ $Y \leftarrow \emptyset$ }
            \ENDDOWHILE{ $X \ne {\bar X}$ }
            \RETURN{ $X$ }
        \ENDFUNCTION
    \stopalgorithmic
\stopbuffer

\startsubsection[title={Product Game Solution}]

    To decide desired properties, game needs to be analysed twice:
    once with adversarial player 2 (to yield winning states where even in the "worst case", game can be won by player 1) and once with cooperative player 2 (to yield states where winning is possible, but not guaranteed).
    From membership in these 2 solution sets, it can be decided if states are satisfying, non-satisfying or if the abstraction is too coarse to decide and further refinement is necessary.

    Quadratic-complexity algorithm exists, but cubic, fixed-point algorithm much easier to implement.
    TODO Reference.

    Define successor operator for game

    \startformula
        \Successor{s}{a} = \Support{\Transition(s, a)}
    \stopformula

    Define conditions for state-action pairs

    \startformula
        \startalign[n=2,align={right,left}]
            \NC C_1(X) =
            \NC \Set{ (s, a) \mid \Successor{s}{a} \subseteq X }
            \NR
            \NC C_2(X, Y) =
            \NC \Set{ (s, a) \mid \Successor{s}{a} \subseteq X \MidAnd \Successor{s}{a} \cap Y \ne \emptyset }
            \NR
        \stopalign
    \stopformula

    Predecessor game operators for adversarial analysis

    \startformula
        \startalign[n=2,align={right,left}]
            \NC \Pre_1^{\Adversarial}(X) =
            \NC \Set{ s \in S_1 \mid \exists a \in Act : (s, a) \in C_1(X) }
            \NR
            \NC \empty
            \NC \quad \cup \Set{ s \in S_2 \mid \forall a \in \Act : (s, a) \in C_1(X) }
            \NR
            \NC \Pre_2^{\Adversarial}(X, Y) =
            \NC \Set{ s \in S_1 \mid \exists a \in Act : (s, a) \in C_2(X, Y) }
            \NR
            \NC \empty
            \NC \quad \cup \Set{ s \in S_2 \mid \forall a \in \Act : (s, a) \in C_2(X, Y) }
            \NR
            \NC \Pre_3^{\Adversarial}(X, Y, Z) =
            \NC \Set{ s \in S_1 \mid \exists a \in Act : (s, a) \in C_2(X, Y) \cup C_1(Z) }
            \NR
            \NC \empty
            \NC \quad \cup \Set{ s \in S_2 \mid \forall a \in Act : (s, a) \in C_2(X, Y) \cup C_1(Z) }
            \NR
        \stopalign
    \stopformula

    and for cooperative analysis

    \startformula
        \startalign[n=2,align={right,left}]
            \NC \Pre_1^{\Cooperative}(X) =
            \NC \Set{ s \in S_1 \cup S_2 \mid \exists a \in Act : (s, a) \in C_1(X) }
            \NR
            \NC \Pre_2^{\Cooperative}(X, Y) =
            \NC \Set{ s \in S_1 \cup S_2 \mid \exists a \in Act : (s, a) \in C_2(X, Y) }
            \NR
            \NC \Pre_3^{\Cooperative}(X, Y, Z) =
            \NC \Set{ s \in S_1 \cup S_2 \mid \exists a \in Act : (s, a) \in C_2(X, Y) \cup C_1(Z) }
            \NR
        \stopalign
    \stopformula

    \placealgorithm[top][fig:abstraction-analysis-solve]{
        Solver for parity-3 games using fixed-point iteration \cite[authoryears][Svorenova2017]. Cubic complexity.
    }{
        % https://mailman.ntg.nl/pipermail/ntg-context/2016/087390.html
        \startframedtext[width=\textwidth,frame=off]
            \getbuffer[buf:abstraction-analysis-solution-algorithm]
        \stopframedtext
    }


\stopsubsection


\startsubsection[title={Product Game Simplification}]

    Only construct reachable states.

    Reminder of how to include co-safe switch into game creation.

    Now: If partial results are already available stop construction once decided state is reached.
    Discuss problems with product game simplification and the available compromise.

\stopsubsection

