Quadratic-complexity algorithm exists, but cubic, fixed-point algorithm much easier to implement.
TODO Reference.

\placealgorithm[top][fig:abstraction-analysis-solve]{
    Solver for parity-3 games using fixed-point iteration \cite[authoryears][Svorenova2017]. Cubic complexity.
}{
    % https://mailman.ntg.nl/pipermail/ntg-context/2016/087390.html
    \startframedtext[width=\textwidth,frame=off]
        \startalgorithmic
            \FUNCTION{SolveParity3}{$\ProductGame = (TODO)$: Game}
                \STATE{ $X, {\bar X}, Y, {\bar Y}, Z, {\bar Z}$: Set}
                \STATE{ ${\bar X} \leftarrow S$ }
                \STATE{ ${\bar Y} \leftarrow S$ }
                \STATE{ ${\bar Z} \leftarrow \emptyset$ }
                \DO
                    \STATE{ $X \leftarrow {\bar X}$ }
                    \DO
                        \STATE{ $Y \leftarrow {\bar Y}$ }
                        \DO
                            \STATE{ $Z \leftarrow {\bar Z}$ }
                            \STATE{ ${\bar Z} \leftarrow (F \cap \Pre_1(X)) \cup (E \cap \Pre_2(X, Y)) \cup (D \cap \Pre_3(X, Y, Z))$ }
                        \ENDDOWHILE{ $Z \ne {\bar Z}$ }
                        \STATE{ ${\bar Y} \leftarrow Z$ }
                        \STATE{ ${\bar Z} \leftarrow S$ }
                    \ENDDOWHILE{ $Y \ne {\bar Y}$ }
                    \STATE{ ${\bar X} \leftarrow Y$ }
                    \STATE{ $Y \leftarrow \emptyset$ }
                \ENDDOWHILE{ $X \ne {\bar X}$ }
                \RETURN{ $X$ }
            \ENDFUNCTION
        \stopalgorithmic
    \stopframedtext
}

Define successor operator for game

\startformula
    ...
\stopformula

Define conditions for state-action pairs

\startformula
    \startalign[n=2,align={right,left}]
        \NC C_1(X) =
        \NC \Set{ ...  }
        \NR
        \NC C_2(X, Y) =
        \NC \Set{ ...  }
        \NR
        \NC C_3(X, Y, Z) =
        \NC \Set{ ...  }
        \NR
    \stopalign
\stopformula


Predecessor game operators for adversarial analysis

\startformula
    \startalign[n=2,align={right,left}]
        \NC \Pre_1(X) =
        \NC \Set{ ...  } \cup \Set{ ... }
        \NR
        \NC \Pre_2(X, Y) =
        \NC \Set{ ...  } \cup \Set{ ... }
        \NR
        \NC \Pre_3(X, Y, Z) =
        \NC \Set{ ...  } \cup \Set{ ... }
        \NR
    \stopalign
\stopformula

and for cooperative analysis

\startformula
    \startalign[n=2,align={right,left}]
        \NC \Pre_1^{\Coop}(X) =
        \NC \Set{ ...  } \cup \Set{ ... }
        \NR
        \NC \Pre_2^{\Coop}(X, Y) =
        \NC \Set{ ...  } \cup \Set{ ... }
        \NR
        \NC \Pre_3^{\Coop}(X, Y, Z) =
        \NC \Set{ ...  } \cup \Set{ ... }
        \NR
    \stopalign
\stopformula

To decide desired properties, game needs to be analysed twice:
once with adversarial player 2 (to yield winning states where even in the "worst case", game can be won by player 1) and once with cooperative player 2 (to yield states where winning is possible, but not guaranteed).
From membership in these 2 solution sets, it can be decided if states are satisfying, non-satisfying or if the abstraction is too coarse to decide and further refinement is necessary.

Discuss problems with product game simplification and the available compromise.
Reminder of how to include co-safe switch into game creation.

