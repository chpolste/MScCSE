For every trace starting in a state recognized as a member of $\InitialStates$, a strategy exists that leads to player 1 winning almost-surely.
A witness strategy can be constructed based on the result from the analysis of the product game in the following way \cite[authoryears][Svorenova2017]:
At every step of the trace in the LSS, a control input is chosen such that the corresponding play in the product game only visits states from the set $\AlmostAdv{\ProductGame}$.
The control inputs are obtained from the $\ActC$-regions associated with player 1 actions targeting only states in $P_\Yes$.
These are selected in a round-robin fashion every time a trace visits a player 1 state.
This is a pure strategy that requires a finite memory to keep track of the round-robin selection of actions.

In general, memoryless strategies exist for one-pair Streett objectives of 2½-player games, because they can be converted into parity-3 objectives for which such strategies are known to exist \cite[authoryears][Chatterjee2012].
Deeper analysis of the game is required to derive these strategies.
A memoryless strategy for problems with robust solutions based on a geometric distance metric induced by the LSS dynamics is briefly discussed in \in{Section}[sec:cases-corridor-synthesis].
Otherwise, the controller synthesis problem is not a focus of this work.

