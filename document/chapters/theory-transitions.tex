Transitions systems are a versatile abstraction, used ubiquitously in computer science to model soft- and hardware systems \cite[alternative=authoryears][Baier2008].
They provide a common basis for many kinds of problems and can be extended by additional features necessary to express the behaviour required for many applications. % TODO: this is very vague

A basic transition system is built from a finite or inifinite set of states and actions and a transition function that combines the two into a directed graph with states as nodes and actions as edges.
The transition relation describes how the system evolves and can be deterministic with a unique target state for every transition or have multiple target states with non-deterministic or a stochastic choice.
Labeling functions can be used to associate states and/or actions with additional properties.
An initial state or a set of initial states is usually defined to provide an entry point for the system.
The evolution of a system from state to state along the transitions induces a seqence of states, typically called a run, that is a recording of the states passed in the sequence.

Transitions systems are for example the foundation of turn-based games, with game states and actions selectable by a set of players who move in turns.
The states of the system are divided into disjunct subsets for each player and player actions alternate between these sets to enforce that the players apply their moves in turns.
Additionally, the game features some condition or reward system, that determines which player wins the game.
Automata representing a language of words are another common application of transition systems.
Labeling the transitions with symbols that make up the words of the language, each run through the automaton generates a word.
An acceptance condition decides if a generated word belongs to the language or not.

Transition systems can be combined to yield new transition systems that combine properties from the systems they are made up of. % TODO combine ... combine ... combine
An important operation is the product of two or more transition systems that must evolve in a synchronized fashion.
This kind of product results in a single transition system that enforces the synchronized evolution.
With such systems, properties of all involved original systems can be tested simultaneously and be combined to form more complex tests.
In model checking, a typical approach is to combine an abstraction model of some system of interest with an automaton specifying the behaviour that is required from the system using the synchronized product.
The resulting transition system is then inspected to determine if the abstraction model fulfills the specification.

In order to be able to decide if some specification is met, one has to determine which behaviour of a system is essential and needs to be preserved in the abstraction used in the verification procedure.
It is generally not easy to find the right set of features a transition system abstraction must exhibit to ensure decidability is not easy.
Fortunately, the technique of abstraction refinement provides a way out of this predicament.

% TODO: add something about problem of figuring out a-priori what is important
% TODO: redistribute content of this section and refinement section. Some stuff is too early and without motivation. Only talk about abstractions and then a quick intro into general transition systems?

