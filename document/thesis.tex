\usecolors[xwi]
\usemodule[algorithmic]

% Fixes for broken rendering
% SVG conversion patch: because my svg exports can contain objects outside of
% the visible area, inkscape extends the size of the page when converting to
% pdf such that the extent of these objects is included in the view (even
% though they are not drawn). With the parameter --export-area-page, this
% behaviour can be changed to match the expected output (no empty borders).
% However, the parameters given to inkscape (which context calls for the
% conversion) are hard-coded into the lua code, so the lua code has to be
% patched.
% The following is a patch for the file
% context/tex/texmf-context/tex/context/base/mkiv/grph-con.lua
% and overrides the svg converter defined there.
\startluacode
do

    local longtostring      = string.longtostring
    local expandfilename    = dir.expandname
    local figures           = figures
    local converters        = figures.converters
    local programs          = figures.programs

    local svgconverter = converters.svg
    converters.svgz    = svgconverter

    local runner = sandbox.registerrunner {
        name     = "custom svg to something",
        program  = "inkscape",
        template = longtostring [[
            "%oldname%"
            --export-area-page
            --export-dpi=%resolution%
            --export-%format%="%newname%"
        ]],
        checkers = {
            oldname    = "readable",
            newname    = "writable",
            format     = "string",
            resolution = "string",
        },
        defaults = {
            format     = "pdf",
            resolution = "600",
        }
    }

    programs.inkscape = {
        runner = runner,
    }

    function svgconverter.pdf(oldname,newname)
        runner {
            format     = "pdf",
            resolution = "600",
            newname    = expandfilename(newname),
            oldname    = expandfilename(oldname),
        }
    end

    function svgconverter.png(oldname,newname)
        runner {
            format     = "png",
            resolution = "600",
            newname    = expandfilename(newname),
            oldname    = expandfilename(oldname),
        }
    end

    svgconverter.default = svgconverter.pdf

end
\stopluacode


% Fix for mathmatrix/mathalign problem:
% https://mailman.ntg.nl/pipermail/ntg-context/2019/094460.html
% https://tex.stackexchange.com/questions/481411:
\unprotect
\newcount\c_math_eqalign_column_saved
\newcount\c_math_eqalign_first_saved

\unexpanded\def\math_matrix_start#1%
   {\begingroup
    \globalpushmacro\c_math_matrix_first
    \c_math_eqalign_column_saved\c_math_eqalign_column
    \c_math_eqalign_first_saved \c_math_eqalign_first
    \edef\currentmathmatrix{#1}%
    \dosingleempty\math_matrix_start_indeed}

\def\math_matrix_stop
   {\math_matrix_stop_processing
    \global\c_math_eqalign_column\c_math_eqalign_column_saved
    \global\c_math_eqalign_first\c_math_eqalign_first_saved
    \globalpopmacro\c_math_matrix_first
    \endgroup}
\protect


% Vertical position of default coloncolon command is not right
\definemathcommand[coloncolon][rel]{\colon\!\colon}


% Fix missing space between compressed years in authoryear cite style
% (from context/tex/texmf-context/tex/context/base/mkiv/bibl-tra.mkiv)
\unprotect

\unexpanded\def\findmatchingyear
  {\edef\wantednumber{\the\bibitemwanted}%
   \getfromcommacommand[\thebibyears][\wantednumber]%
   \ifx\commalistelement\empty
     \edef\myyear{{\myyear}}%
   \else
     \edef\myyear{{\commalistelement, \myyear}}%
   \fi
   \edef\newcommalistelement{\myyear}%
   \doglobal\replaceincommalist \thebibyears \wantednumber}

\protect


% Custom TeX-commands

% CamelCase for custom commands

\define\DocTitleFooter{Refinement for game-based abstractions of continuous-space LSSs} % TODO
\define\Author{Christopher Polster}
\define\YearOfCompletion{2019}
\define\DateOfCompletion{\YearOfCompletion-??-??} % TODO

% Tall version of \mid
\definemathcommand[Bigmid][rel]{\mathrel{\Big|}}

% Equation endings
\define\EndComma{\;\text{,}}
\define\EndPeriod{\;\text{.}}
\define\EndAnd{\;\text{and}}
% Inline connectors
\define\MidComma{\text{,}\;}
\define\MidAnd{\;\text{and}\;}


% General math
% ------------

\define[3]\Function{#1:#2\rightarrow#3}

% 2-Norm
\define[1]\TwoNorm{\lVert{#1}\rVert}

% Vector variables with bolditalic font
\define[1]\Vec{{\bi #1}}
\define\VecC{\Vec{c}}
\define\VecU{\Vec{u}}
\define\VecV{\Vec{v}}
\define\VecW{\Vec{w}}
\define\VecX{\Vec{x}}
\define\VecY{\Vec{y}}
\define\VecZ{\Vec{z}}

% Matrix variables with bolditalic font
\define[1]\Mat{{\bi{#1}}}
\define\MatA{\Mat{A}}
\define\MatB{\Mat{B}}
\define\MatU{\Mat{U}}

% Vectors and matrices with square brackets
\definemathmatrix[sqmatrix][left={\left\lbrack\,},right={\,\right\rbrack},strut=0.8,distance=0.8em]
\define[4]\TwoByTwo{ \startsqmatrix[n=2,align={middle,middle}] \NC #1 \NC #2 \NR \NC #3 \NC #4 \NR \stopsqmatrix }
\define[2]\TwoByOne{ \startsqmatrix[n=1,align={middle}] \NC #1 \NR \NC #2 \NR \stopsqmatrix }
\define[2]\OneByTwo{ \startsqmatrix[n=2,align={middle,middle}] \NC #1 \NC #2 \NR \stopsqmatrix }
\define[1]\OneByOne{ \startsqmatrix[n=1,align={middle}] \NC #1 \NR \stopsqmatrix }

% Inline vectors and matrices with square brackets (https://www.mail-archive.com/ntg-context@ntg.nl/msg78899.html)
\definemathmatrix[ssqmatrix][left=\left\lbrack,right=\right\rbrack,style=\scriptstyle,strut=0.5,distance=0.5em]
\define[2]\TwoByOneSmall{ \startssqmatrix[n=1,align={middle}] \NC #1 \NR \NC #2 \NR \stopssqmatrix }

% Sets and Intervals
\define[1]\Set{{\{ {#1} \}}}
\define[1]\BigSet{{\Big\{ {#1} \Big\}}}
\define[2]\IndexedSet{\Set{#1}_{#2}}
\define[2]\ClosedInterval{\left[\, #1,\, #2 \,\right]}

% N-Tuples
\define[2]\Tuple{({#1},\,{#2})}
\define[2]\BigTuple{\Big({#1},\,{#2}\Big)}
\define[3]\Triple{({#1},\,{#2},\,{#3})}
\define[3]\BigTriple{\Big({#1},\,{#2},\,{#3}\Big)}

% Inline multiline aligned blocks (http://dl.contextgarden.net/myway/mathalign.pdf)
\definemathmatrix[gathered][n=1,align=left,style=\displaystyle]

% Polytopic Computations
% ----------------------

% Polytope components
\definemathcommand[Hull][nolop]{\mfunction{hull}}
\definemathcommand[Vertices][nolop]{\mfunction{vert}}
% Posterior
\definemathcommand[Post][nolop]{\mfunction{Post}}
% Predecessors
\definemathcommand[Pre][nolop]{\mfunction{Pre}}
\definemathcommand[PreR][nolop]{\mfunction{PreR}}
\definemathcommand[PreP][nolop]{\mfunction{PreP}}
% Attractors
\definemathcommand[Attr][nolop]{\mfunction{Attr}}
\definemathcommand[AttrR][nolop]{\mfunction{AttrR}}
% Action Polytopes
\definemathcommand[Act][nolop]{\mfunction{Act}}
\definemathcommand[ActR][nolop]{\mfunction{ActR}}
\definemathcommand[ActC][nolop]{\mfunction{ActC}}

% States
\define\StateSpace{X}
\define\StateRegion{X'}
\define\ExtendedStateSpace{\StateSpace_{ext}}
\define\InitStateSpace{\StateSpace_{init}}
\define\VecState{\VecX}
\define[1]\State{X_{#1}}
\define[2]\IndexedStates{\IndexedSet{\State{#1}}{{#1} \in {#2}}}
\define\Predicates{\Pi}
\define[1]\PredicatesOf{\pi({#1})}

% Control
\define\ControlSpace{U}
\define\VecControl{\VecU}
\define[2]\Action{\Act({#1},\, {#2})}

% Random
\define\RandomSpace{W}
\define\VecRandom{\VecW}

% Polytopic Operators
\define[2]\Posterior{\Post({#1},\, {#2})}
\define[3]\Predecessor{\Pre({#1},\, {#2},\, {#3})}
\define[3]\RobustPredecessor{\PreR({#1},\, {#2},\, {#3})}
\define[3]\PrecisePredecessor{\PreP({#1},\, {#2},\, {#3})}
\define[3]\Attractor{\Attr({#1},\, {#2},\, {#3})}
\define[3]\RobustAttractor{\AttrR({#1},\, {#2},\, {#3})}
\define[2]\Action{\Act({#1},\, {#2})}
\define[2]\RobustAction{\ActR({#1},\, {#2})}
\define[2]\ConcreteAction{\ActC({#1},\, {#2})}


% Logic and Automata
% ------------------

\definemathcommand[Inf][nolop]{\mfunction{Inf}}
\define[1]\InfinitelyOften{\Inf(#1)}

% Words
\define\Automaton{{\mathcal A}}
\define\Language{{\mathcal L}}
\define\Condition{{\mathcal C}}
\define\Transition{\delta}
\define[1]\RepeatFinitely{{#1}^{\ast}}
\define[1]\RepeatInfinitely{{#1}^{\omega}}

\definemathcommand[True]{{\mathss true}}
\definemathcommand[Next]{{\mathss X}}
\definemathcommand[Until]{\,{\mathss U}\,}
\definemathcommand[Finally]{{\mathss F}}
\definemathcommand[Globally]{{\mathss G}}


% Games
% -----

\definemathcommand[AlmostSure][nolop]{\mfunction{Almost}}

\define\GameGraph{{\mathcal G}}
\define[2]\Strategy{S_{#2}^{#1}}
\define[1]\Almost{\AlmostSure_{#1}}
\define[1]\AlmostCoop{\AlmostSure_{#1}^{coop}}
\define[1]\ProbDist{\mathcal{D}(#1)}

\define\ProductGame{{\mathcal P}}
\define[2]\PlayerOneAction{{#1}\rightarrow{#2}}
\define[3]\PlayerTwoAction{\Tuple{#1}{#2}\rightarrow{#3}}


% Physics
% -------

\define\DDt{\frac{{\mathrm d}}{{\mathrm d}t}}

\define\Angle{\alpha}
\define\AngularMomentum{L}
\define\Torque{\tau}
\define\Force{F}
\define\MaxForce{\tilde\Force}
\define\MomentOfInertia{I}
\define\Mass{m}
\define\Length{l}
\define\Gravity{g}
\define\Friction{\mu}
\define\Thrust{u}
\define\Deltat{\Delta t}



% Language
\mainlanguage[en]
\hyphenation{}

% Page setup
\setuppapersize[A4]
\setuplayout[backspace=35mm,width=150mm,header=0mm,footer=0mm]

% Setup links and assign PDF metadata
\setupinteraction[
    state=start,
    color=black,
    contrastcolor=black,
    style=,
    focus=standard,
    title={}, % TODO
    subtitle={}, % TODO
    author={Christopher Polster},
    ]

% Title of table of content
\setupheadtext[content={Table of Contents}]
% Title of bibliography
\setupheadtext[pubs={References}]

% Fonts
% Text uses serif font
\setupbodyfont[11pt,serif]
\definebodyfontenvironment[11pt][a=12pt,b=13pt,c=14pt,d=9pt]

% Headings etc. are sans-serif
\definefontfamily[titlefont][sans][dejavusans]
\definefont[CoverTitleFont][dejavusansbold at 19pt]
\definefont[CoverSubtitleFont][dejavusansbold at 15pt]
\definefont[AbstractHeadingFont][dejavusans at 11pt]
\definefont[ChapterFont][dejavusans at 19pt]
\definefont[SectionFont][dejavusans at 13pt]
\definefont[SubsectionFont][dejavusans at 12pt]
\definefont[TOCHeadingFont][dejavusans at 11pt][2] % 3rd [] is line height
\definefont[PageNumberFont][dejavusans at 10pt]
\definefont[FooterFont][dejavusans at 8pt]
\definefont[FigureFont][dejavusans at 8.5pt][1.1]
\definefont[FigureCaptionFont][dejavusansbold at 8.5pt][1.1]

% Styles for headings etc.
\definealternativestyle[FigureStyle][\small\FigureFont] % The \small ensures formulas are resized too
\definealternativestyle[FigureCaptionStyle][\FigureCaptionFont]
\definealternativestyle[TOCStyle][\TOCHeadingFont]
\definealternativestyle[ChapterStyle][\ChapterFont]
\definealternativestyle[SectionStyle][\SectionFont]
\definealternativestyle[SubsectionStyle][\SubsectionFont]

% No chapter numbering, apply heading styles
\setuphead[chapter][number=no,style=ChapterStyle,page=yes]
\setuphead[section][style=SectionStyle,sectionsegments=section]
\setuphead[subsection][style=SubsectionStyle,sectionsegments=section:subsection]

% Don't reset section numbering in new chapter
\definestructureresetset[nosecreset][1,1,0][1] % [part, chapter, section][default]
\setuphead[sectionresetset=nosecreset]

% Table of contents
\setuplist[chapter][style=TOCStyle]
\setupcombinedlist[section,subsection][alternative=c]
\setuplist[section][width=9mm]
\setuplist[subsection][width=9mm,margin=9mm]

% Which captions to include in the table of contents?
\placebookmarks[chapter,section,subsection]

% Footnotes
\define[1]\footnotebrack{\narrownobreakspace\high{[#1]}}
\setupnotation[footnote][alternative=text]
\setupfootnotes[
    way=bytext,
    frameoffset=0mm,
    topframe=on,
    rule=off,
    toffset=1mm,
    roffset=-14cm,
    before={\blank[7mm]},
    %textcommand=\footnotebrack
    ] % fixes weird spacing when only one line of footnote

% Float captions
\setupcaptions[
    style={FigureStyle},
    suffix={:},
    headstyle={FigureCaptionStyle},
    prefixsegments=none,
    width=fit,
    way=bytext,
    spaceafter=2mm,
    ]

% Figures
\setupexternalfigures[directory=figures/]

% Algorithms
\definefloat[algorithm][algorithms]
\setupalgorithmic[
    numbering=yes,
    width=1.2em,
    numberwidth=2em,
    margin=3.2em,
    align=flushright,
    location=text,
    ]

% Formulas
\defineseparatorset[none][][]
\setupformulas[way=bytext,prefixsegments=none,numberseparatorset=none]
% Increase spacing around binary relation symbols (default is 5mu plus 5mu).
\thickmuskip=7mu plus 5mu

% Bibliography
\setupbibtex[database={inputs/references},sort=author]
\setuppublications[criterium=cite,alternative=apa,sorttype=bbl,refcommand=authoryear]
\setuppublicationlist[artauthoretallimit=40,criterium=all]
\setupcite[inbetween={\ }] % fix spacing after et al.
\setupcite[authoryears][
    pubsep={; },
    lastpubsep={; },
    inbetween={\ }, % no comma
    compress=no,
    left={(},
    right={)},
    ]

% Taken from bibl-apa.tex and added doi
\setuppublicationlayout[article]{%
   \insertartauthors{}{ }{\insertthekey{}{ }{}}%
   \insertpubyear{(}{). }{\unskip.}%
   \insertarttitle{\bgroup }{\egroup. }{}%
   \insertjournal{\bgroup \it}{\egroup}
    {\insertcrossref{In }{}{}}%
   \insertvolume
    {\bgroup \it, }
    {\egroup\insertissue{\/(}{)}{}\insertpages{, }{.}{.}}
    {\insertpages{, pp. }{.}{.}}%
   \insertdoi{ doi:\mbox\bgroup}{\egroup.}{}% no linebreaks in doi
   \insertnote{ }{.}{}%
   \insertcomment{}{.}{}%
}
% TODO also add doi to inproceedings

% MetaPost: used to draw omega-automata with package automata.mp from
% https://www.ctan.org/tex-archive/graphics/metapost/contrib/macros/automata.
\startMPinclusions
    input inputs/automata;
    % Global settings for package automata
    size := 30;
    incominglength := 35;
    loopsize := 30;
    tipangle := 30;
    tipsharpness := 25;
    tipsize := 6;
    arrowmargin := 1;
    doublesize := 2;
\stopMPinclusions


% Cover
\definesectionblock[cover][,]
\setupsectionblock[cover][page=right]
\startsectionblockenvironment[cover]
    \switchtobodyfont[titlefont,10pt]
    \setupinterlinespace[line=14pt]
    \setupwhitespace[medium]
\stopsectionblockenvironment

% Abstract
\definesectionblock[abstract][,]
\setupsectionblock[abstract][page=right]
\startsectionblockenvironment[abstract]
    \setuppagenumbering[state=stop,alternative=doublesided]
    \setupinterlinespace[line=1.5em]
    \setupwhitespace[medium]
    \setupnarrower[middle=10mm]
    \page[right]
\stopsectionblockenvironment

% Table of contents
\definesectionblock[tableofcontents][,]
\setupsectionblock[tableofcontents][page=right]
\startsectionblockenvironment[tableofcontents]
    \setupwhitespace[0.5em]
    \setuppagenumbering[state=stop,alternative=doublesided]
\stopsectionblockenvironment

% Frontmatter: acknowledgements, preface
\setupsectionblock[frontpart][page=right]
\startsectionblockenvironment[frontpart]
    \setuplayout[footerdistance=8mm,footer=6mm]
    \setcounter[userpage][1] % Reset page counter
    \setuppagenumbering[
            state=start,
            location={footer,right},
            left={},
            right={},
            alternative=doublesided,
            style=\PageNumberFont
            ]
    \setupuserpagenumber[numberconversion=romannumerals]
    \setupwhitespace[medium]
    \setupinterlinespace[line=1.5em]
\stopsectionblockenvironment

% Bodymatter: text
\setupsectionblock[bodypart][page=right]
\startsectionblockenvironment[bodypart]
    \setuplayout[footerdistance=8mm,footer=6mm]
    \setcounter[userpage][1] % Reset page counter
    \setuppagenumbering[
            state=start,
            location={footer,right},
            left={},
            right={},
            alternative=doublesided,
            style=\PageNumberFont
            ]
    \setupfootertexts[{\FooterFont {\DocTitleFooter}}]
                     [pagenumber]
                     [pagenumber]
                     [{\FooterFont Master's Thesis of {\Author} (\YearOfCompletion)}]
    \setupbackgrounds[footer][text][topframe=on]
    \setupwhitespace[medium]
    \setupinterlinespace[line=1.5em]
\stopsectionblockenvironment

% Backmatter: bibliography
\setupsectionblock[backpart][page=right]
\startsectionblockenvironment[backpart]
    \setuppagenumbering[alternative=doublesided,style=\PageNumberFont]
    \setuplayout[footerdistance=8mm,footer=6mm]
    \setupfootertexts[{\FooterFont {\DocTitleFooter}}]
                     [pagenumber]
                     [pagenumber]
                     [{\FooterFont Master's Thesis of {\Author} (\YearOfCompletion)}]
    \setupbackgrounds[footer][text][topframe=on]
    \setupwhitespace[medium]
    \setupinterlinespace[line=1.5em]
\stopsectionblockenvironment

% Declaration
\definesectionblock[declaration][,]
\setupsectionblock[declaration][page=right]
\startsectionblockenvironment[declaration]
    \setuppagenumbering[state=stop,alternative=doublesided]
    \setupinterlinespace[line=1.5em]
    \setupwhitespace[medium]
\stopsectionblockenvironment



\starttext

    \startsectionblock[cover]
        \startalignment[flushright]
    \dontleavehmode
    \externalfigure[TUM-logo.pdf][height=10mm]
\stopalignment

\blank[15mm]

{ \CoverTitleFont
    \strut Abstraction Refinement \blank[3mm]
    \strut for Continuous-Space \blank[3mm]
    \strut Markov Decision Processes
}

\blank[7.5mm]

{\CoverSubtitleFont Subtitle TBD} % TODO

\blank[100mm]

Master's Thesis in Computational Science and Engineering

Technical University of Munich

Department of Informatics

\blank[20mm]

\starttable[s0|lp(35mm)|lp(115mm)|]
    \NC \bf Supervisor \NC Univ.-Prof. Dr. Jan Křetínský \NC \AR
    \NC \NC Chair for Foundations of Software Reliability and 
    Theoretical Computer Science \NC \AR
    \SR
    \NC \bf Author \NC \Author \NC \AR
    \SR
    \NC \bf Submission \NC Munich, \DateOfCompletion \NC \AR
\stoptable


    \stopsectionblock

    \startsectionblock[abstract]
        \startnarrower

    \blank[10mm]
    {\AbstractHeadingFont Abstract (EN)}
        
    ...

    \blank[10mm]
    {\AbstractHeadingFont Abstract (DE)}
    
    ...

\stopnarrower


    \stopsectionblock

    \startsectionblock[tableofcontents]
        \completecontent
    \stopsectionblock

    \startfrontmatter

        \startchapter[title=Acknowledgements]
            ...
        \stopchapter

        \page[even,empty]

        \startchapter[title=Notation]
            ...
        \stopchapter

        \page[even,empty]

    \stopfrontmatter

    \startbodymatter

        \startchapter[title={Introduction}]

            In the early 1980's model checking was introduced by E. M. Clarke and E. A. Emerson and started to establish itself as an automated and practical alternative to the manual and proof-theoretic reasoning methods used to verify software until then \cite[authoryears][Emerson2008].
The novel combination of finite, state-based models and specifications given in a temporal logic had allowed the verification of concurrent, \quotation{reactive} programs.
Exponentially increasing computing power due to Moore's law as well as the development of abstraction techniques has helped to reduce the complications of the state space explosion problem.
Today, model checking is established as a mature procedure in the formal verification community, although it's industrial adoption is lacking behind the research activity \cite[authoryears][Bennion2014].
Due to its flexible approach, model checking has found new areas of application since its inception.
Its correct-by-design philosopy and synthesis capabilities are enjoyed by the computer science and control engineering communities for the verification of cyber-physical systems and their controllers \cite[authoryears][Ehlers2017,Balkan2018].



            \startsection[title={Model Checking of Hybrid Systems},reference=sec:introduction-topic]
                Systems that combine interacting discrete and continuous domains are called hybrid systems.
\cite[VanDerSchaft2000] remarked at the start of the 21st century that the area of hybrid systems was \quotation{still largely unexplored}, but two decades later \cite[Lin2014] introduce hybrid systems by observing that they \quotation{have recently been at the center of intense research activity}.
The reason this interest has emerged is the broad, multidisciplinary application spectrum of the hybrid system framework and its many related theoretical research questions.
Control problems of hybrid systems arise for example in robotics, where a robot has to complete discrete tasks while moving in a continuous environment, autonomous driving, where vehicle trajectories have to satisfy safety specifications, or the modelling and supervision of chemical and biological processes which exhibit threshold behaviour.

The probably most often used example of a simple hybrid system is a heater controlled by a thermostat.
The heater works in a discrete domain and its state is either on or off.
The thermostat measures temperature, which is a continuous, real-valued variable.
Both are coupled by discrete if-then-else rules: if the observed temperature falls below some threshold, the thermostat activates the heater.
Once the desired temperature is reached, the heater is turned off.

There are two essential problems in the control of hybrid systems:
The verification or analysis problem is to determine from which intital states the system evolution can satisfy a specification.
The synthesis problem involves the construction of a control strategy that achieves satisfaction.
Both problems are linked and typically solved together, i.e.\ the system anaylsis provides the information necessary to synthesize satisfying control strategies.
Model checking approaches have been established that solve these problems of some classes of hybrid systems.

The category of hybrid systems encompasses a variety of characteristics that can be combined to form a specific problem.
The system dynamics governs the evolution of trajectories, i.e.\ how the state of a specific problem instance evolves in time.
This evolution is generally modelled by differential equations in contexts with continuous time or difference equations when time is discrete.
Linear and piece-wise linear (also known as piece-wise affine) equations are commonly found due to their simplicity and invertibility but non-linear dynamics can be encountered in the general case.
Aside from deterministic evolution, the dynamics can also prescribe a probabilistic evolution in environments with uncertainty.
Control of the system is excerted by manipulating the dynamics, e.g. through a control term in the evolution equation or by switching between multiple admissible system evolutions.
In unknown or changing environments, the controller may not be able to observe the entire system state which introduces unknowns that have to be accounted for by solution approaches.

The verification and synthesis problems require a specification or an objective that define which behaviour is considered \quotation{good} or \quotation{bad}.
Such objectives can be basic properties like reachability of safety or rich specifications, usually expressed in a temporal logic.
The temporal logic is chosen according to the properties of the system under consideration and the specification.
Objectives can be qualitative or quantitative with respect to the probabilistic aspects of the system or imposed reward schemes.
It may also be desirable to account for changes in objectives with time, e.g.\ due to external requirements.
In such cases, solutions must be able to dynamically adjust the strategy while the system is evolving.

Model checking approaches have to be adapted to the given characteristics of a given problem.
In general, model checking approaches will construct a discretization of the continuous domain of the hybrid system such that the essential behaviour of the system is captured.
The concrete discrete abstraction model depends on the system characteristics and specification.
Commonly encountered are automata, Markov models and (probabilistic) games.
The system abstraction is then synchronized with the specification, which is first translated to an automaton.
The resulting finite system model is then verified and used to synthesize discrete controllers.
Post-processing of these discrete controllers can be applied to obtain continuous feedback controllers.

System abstraction and the synchronization of sytem and specifications generally cause an explosion of the size of the state space.
This makes the verification and synthesis procedures computationally demanding and is a major challenge for hybrid system analysis.
Restrictions can be imposed on the expressivity of the temporal logic to reduce the complexity of the specification automaton and verification procedure.
Iterative refinement techniques aim to generate small partitions of the continuous domain based on the system dynamics and therefore limit the extent of the state space explosion.

Verification and synthesis problems for hybrid systems have been developed and applied for example in the following works and the references therein:
\cite[Kloetzer2008] consider verification and control strategy synthesis for continuous-time linear system and $\Next$-free linear temporal logic (LTL).
\cite[Yordanov2010] and \cite[Yordanov2012] deal with discrete-time piecewise-affine systems and LTL specifications, using iterative refinement and conservative analysis of transition system abstractions, respectively.
Markov set-chain abstractions for uncontrolled discrete-time, stochastic dynamics are investigated by \cite[Abate2011].
\cite[Hahn2011] construct game-abstractions with iterative refinement for probabilistic hybrid automata with discrete probability distributions.
\cite[AydinGol2014,AydinGol2015] develop an abstraction refinement technique for the verification of deterministic, discrete-time linear systems with respect to co-safe LTL specifications and derive optimal model predictive control strategies with a cost-minimization approach.
Discrete-time, continuous-space switched systems are considered by \cite[Lahijanian2015] who solve the verification and synthesis problem for specifications given in probabilistic computation tree logic with iteratively refined interval-valued Markov chain and bounded parameter Markov decision process abstractions.


            \stopsection

            \startsection[title={Review of Related Work},reference=sec:introduction-review]
                Is it ok that literature review comes before the theoretical foundations (there is another review in the "Related Systems" section)?

Discuss the previous work by \cite[Svorenova2017].

Look at previous uses of GR(1)/LTL objectives and practical applications.
Look at similar hybrid systems covered by other authors.
Present different problem setups and common solution approaches (model checking, automata, games, learning, ...).


            \stopsection

            \startsection[title={Goals of This Work},reference=sec:introduction-goals]
                What did I set out to achive?
Formulate concrete goals:
\cite[Svorenova2017] foundation of thesis.
Application of the procedure to problems other than reachability, which was the focus of \cite[Svorenova2017].
Construct refinement procedures for general GR(1) objectives and investigate refinement techniques in general.
Personal perspective: little experience with computational geometry, LSS abstraction, model checking.

Built an educational tool, useful for interactive exploration.
Helps to build intuition for the problem which can be used for the design of refinement heuristics.

Focus on analysis and refinement, not controller synthesis which is a secondary, though related, problem.


            \stopsection

        \stopchapter

        \startchapter[title={Preliminaries}]

            This chapter introduces the fundamental concepts and notation needed to express the content of this work.
The introductions are kept relatively brief.
The interested reader is invited to consult the references in the text for a more extensive treatment of the presented topics.
In particular, the book of \cite[Baier2008] is an excellent resource for everything related to model checking and \cite[Baotic2009] provides a good introduction to polytopic geometry for control problems.



            \startsection[title={Model Checking},reference=sec:theory-checking]
                The goal of model checking is to verify with an automatic procedure if some system meets a specification.
To achieve this, the two components \quotation{system} and \quotation{specification} have to be represented in a way that they can be brought together and then analysed.
Typically the system is abstracted to a system model, able to reproduce all behaviour required to accurately verify the properties of interest.
For the specification, a formulation in logic, usually a temporal logic, is sought.
The specifications may concern qualitative (e.g. \quotation{can something bad happen?}) or quantitative questions (e.g. \quotation{will something bad happen with more than 90\% probability in the first hour of operation?}) about the system.

Model checking is a brute-force approach in which the system model is explored exhaustively, i.e. every possible system state is considered during the verification procedure.
This approach ensures that no malicious behaviour of the system is missed and is also able to produce counterexamples when a system model fails to meet the specification.
Counterexamples guide the search for system errors making them a valuable part of the verification process.
However, the exhaustive approach means that substantial resources are required to carry out the desired analysis.
Appropriate and expressive abstractions for the system and specification lead to a large number of possible states that need to be checked, often growing exponentially with the complexity of the abstractions.
Dealing with this so-called state space explosion is a major task when designing a model checking procedure.
To combat state space explosion and ensure the feasability of finding solutions, abstractions have to be chosen with care and effective state space exploration techniques must be used.

Additionally, the advent of increasingly powerful computers has brought larger and larger systems into the reach of formal verification by model checking.
Since the beginning of model checking by TODO and TODO, procedures for a multitude of kinds of systems and specifications have been devised and are actively being reasearched. % TODO inventors of model checking
Model checking today has widespread applications from detecting deadlocks in concurrent programs to robot motion planning to hardware and security protocol verification. %TODO some e.g. references

Next, the concept of transition systems, which are the foundation of virtually every system abstraction seen in model checking, is introduced.


            \stopsection

            \startsection[title={Transition Systems},reference=sec:theory-transitions]
                Transitions systems are a versatile abstraction, used ubiquitously in computer science to model soft- and hardware systems \cite[alternative=authoryears][Baier2008].

A basic transition system is built from a finite or inifinite set of states and actions and a transition function that combines the two into a directed graph with states as nodes and actions as edges.
The transition relation describes how the system evolves from its states can be deterministic with a unique target state for every transition or have multiple target states with non-determinism or a stochastic choice.
Labeling functions can be used to associate states and/or actions with additional information, typically valuations of atomic propositions.
An initial state or a set of initial states is usually defined to provide an entry point for the system.
The evolution of a system from state to state along the transitions induces a seqence of states, typically called a run, that is a recording of the states passed in a run.
For specific applications, the system may be extended with other necessary features.

Transitions systems are e.g. the foundation of turn-based games, with game states and actions triggered by a set of players who move in turns.
The states of the system are divided into disjunct subsets for each player and player actions alternate between these sets to enforce that the players apply their moves in turns.
Additionally, the system will feature some condition or reward system, that determines which player wins the game.
Automata representing a language of words are another common application of transition systems.
By labeling the transitions with symbols that make up the words of the language, each run through the automaton generates word.
An acceptance condition then decides if a generated word belongs to the language or not.

Transition systems can be combined to yield new transition systems that contain properties from the systems they are made up from.
An important operation is the product of two or more transition systems that must evolve in synchronized fashion.
The synchronized product will result in a single transition system that enforces the synchronized evolution.
This way, properties of all involved subsystems can be tested together and combined to form more complex tests.
In model checking, the typical approach is to combine an abstraction model of some system of interest with an automaton specifying the behaviour that is required from the system using the synchronized product and then inspecting the resulting transition system to determine if the abstraction model fulfills the specification.

In order to be able to decide if a specification is met, one has to determine which behaviour of a system is essential and needs to be preserved in the abstraction used in the verification procedure.
Finding the right set of features a transition system abstraction must exhibit to ensure decidability is not easy.
Fortunately, the technique of abstraction refinement provides a way out of this predicament.


            \stopsection

            \startsection[title={Abstraction Refinement},reference=sec:theory-refinement]
                The idea of an abstraction is to reduce a system to a set of equivalence classes, also called abstract states, whose possibly infinitely many member states of the original system share certain properties and/or behavior.
Capturing exactly those features from the original system which are required to carry out an analysis with respect to some specification is a challenging task.
Instead of going directly from the system to an abstraction on which the model checking problem can be decided, a hierarchy of intermediate system models can be constructed.
Starting from a very coarse abstraction of the system, models become more refined when moving down the hierarchy towards the original system with all its complexity.
If some abstraction in the hierarchy has been verified to meet the specification, all its refinements, including the original system, fulfil the specification as well.
This is because an abstraction can do everything its refinements can do and generally more.
In other words, for every behaviour in the refinement an equivalent exists in the abstraction, so if all behaviour of the abstraction meets the specification, all behaviour of its refinements will too.
It is important to note that this is an asymmetric, one-way relation: if a refined model meets a specification, its higher level abstrations generally do not fulfill the specification too.

In iterative abstraction refinement, the abstraction hierarchy is constructed top-down until a model is found for which the model checking problem can be decided.
One starts by applying the verification procedure on an initial, coarse abstraction of the system of interest.
Because the abstraction is coarse, it is likely that it does not meet the specification and a counterexample is found.
This counterexample has to be recreated in the original system in order to distinguish actual counterexamples, which can be realized in the original system, from spurious ones, which emerge from behaviour that is exclusive to the abstraction.
If an actual counterexample is encountered, the original system does not meet the specification and the verification procedure terminates.
If a spurious counterexample is encountered, the abstraction is refined such that the behaviour from the counterexample is removed during refinement.
The procedure then starts a new verification-refinement-cycle with the newly obtained abstraction and iterates until the abstration is sufficently close to the original system and the verification problem can be decided.

Because the counterexamples encountered in higher levels of the abstraction hierarchy inform the refinement that produces lower-level abstractions, this procedure is called counterexample-guided abstraction refinement (CEGAR).
CEGAR is an effective way to generate system models that adapt to the requirements of a specific problem.
Ideally, its abstractions expose only the minimal complexity and behaviour required to carry out the desired analysis.
In practice, the refinement steps are not always obvious and driven by imperfect heuristics, which have to be carefully tuned for the problems at hand.
Nevertheless, the flexibility of CEGAR makes it an important tool, applied frequently in model checking procedures.


            \stopsection

            \startsection[title={Hybrid Systems},reference=sec:theory-hybrids]
                Hybrid systems present a framework in which discrete-event and continuous-variable dynamics are coupled to produce heterogeneous behaviour emerging from the interaction of both domains.
While \cite[VanDerSchaft2000] remarked at the start of the 21st century that the area of hybrid systems was \quotation{still largely unexplored}, two decades later \cite[Lin2014] introduce hybrid systems observing that they \quotation{have recently been at the center of intense research activity}.

The reason this interest has emerged is the broad, multidisciplinary application spectrum of the hybrid system framework and its many related theoretical research questions.
It has found use in computer-aided verification of programs interacting with continuous environments (embedded and cyber-physical systems) and is applied to control theory as well as robotics and artificial intelligence problems. %TODO: references!
Many examples of hybrid systems are found in the introductions by \cite[VanDerSchaft2000,Lin2014] and the references therein.
Probably the most often invoked example of a simple hybrid system is a heater controlled by a thermostat.
The heater works in a discrete domain, its state is either on or off.
The thermostat measures temperature, which is a continuous, real valued variable.
Both are coupled with discrete if-then-else rules: if the observed temperature falls below some threshold, the thermostat activates the heater.
Once the desired temperature is reached, the heater is turned off.

It is essential to every hybrid system that the discrete and continuous domains not just coexist but interact with each other.
The discrete domain is often governed by a set of events resulting in threshold- or switching behaviour of the system.
The continuous variables are usually associated with physical processes such as mechanical phase space, temperature or electrical current.
Time can be both discrete or continuous, leading to a description of the dynamics with difference or differential equations, respectively.
Because hybrid systems can be assembled from such a variety of components, it is difficult to find a general definition.
The solution approaches to problems involving hybrid systems however generally revolve around the synchronous evolution of the discrete and continuous domain, formalized by so-called hybrid automata.

For the purposes of this work, a special class of hybrid system must be introduced, the linear stochastic system.


\startsubsection[title={Linear Stochastic Systems},reference=sec:theory-hybrids-lss]

    A linear stochastic system (LSS) $\LSS$ is a discrete-time continuous-space stochastic process evolving in traces governed by the evolution equation

    \placeformula[formula:lssevo]
    \startformula
        \VecX_{t+1} = \MatA \VecX_t + \MatB \VecU_t + \VecW_t \EndComma
    \stopformula

    where $\VecX_t \in X \subset \reals^n$ is the state of the trace $\VecX = \VecX_0 \VecX_1 ...$ at time $t \in \naturalnumbers_0$,
    $\VecU_t \in U \subset \reals^m$ is a control input and
    $\VecW_t \in W \subset \reals^n$ is a random perturbation, also at time $t$.
    The state space $\StateSpace$ and control space $\ControlSpace$ are embedded in Eucledian, real-valued vector spaces of dimensions $n$ and $m$, respectively.
    $X, U, W$ are bounded sets.
    The probability distribution from which the random vector is sampled is assumed to have non-zero density everywhere in $W$ for all times $t$.

    A trace evolves under the influence of matrix $\MatA \in \reals^{n \times n}$ transforming the current state $\VecState_t$, a stochastic perturbation $\VecRandom_t$ and external control, which is exerted by a control input $\VecControl_t$ and projected into the state space by matrix $\MatB \in \reals^{n \times m}$.
    The control inputs are chosen every time step by a strategy $\Function{S_\mathcal{T}}{X^+}{U}$ which takes the evolution of the trace up to its current state into account. % TODO: ^+ notation introduced later
    To reason about properties of a trace, it is augmented with discrete events, e.g. entering or exiting some region of the state space.

\stopsubsection


\startsubsection[title={Related Systems}]

    The non-stochastic variant of an LSS has the dynamics

    \placeformula[formula:linearsystem]
    \startformula
        \VecX_{t+1} = \MatA \VecX_t + \MatB \VecU_t + \VecC \EndComma
    \stopformula

    where $\VecC \in \reals^n$ can be seen as a degenerate case of an LSS with $W = \Set{\VecC}$.
    Such linear systems are a special case of piecewise affine systems, also known as piecewise linear systems, whose state space is partitioned into multiple disjunct regions, called modes, each with their own evolution equation

    \startformula
        \VecX_{t+1} = \MatA_l \VecX_t + \MatB_l \VecU_t + \VecW_{l,t} \EndComma
    \stopformula

    where $l$ is an index enumerating the modes.
    A trace evolves according the evolution equation associated with the mode its current state is a member of. % TODO phrasing
    A non-stochastic variant ($W_l = \Set{\VecC_l}$ for all modes $l$) of piecewise affine systems was used e.g. by \cite[Yordanov2009]. % TODO: more references. Also: what did they use the system for?

    More generally, any kind of (non-linear) dynamics can be considered with evolution equations of the form

    \startformula
        \VecX_{t+1} = F(\VecX_t, \VecU_t, \VecW_t) \EndPeriod
    \stopformula

    In (stochastic) switched systems control is not excerted by choosing the value of a control vector at each timestep, but choosing the entire form of the dynamics

    \startformula
        \VecX_{t+1} = U_t(\VecX_t, \VecW_t) \EndPeriod
    \stopformula
    
    The choice of $\Function{U_{t}}{X \times W}{X}$ is usually restricted to a finite set of available dynamics between which the controller can switch at each timestep.
    This kind of system was used for example by \cite[Lahijanian2015]. % TODO: more references

    All presented systems so far have used discrete time and therefore difference equations to describe the evolution of traces.
    But the dynamics can also be entirely continuous, with traces evolving in time according to differential equations.
    Continuous-time, countinuous-space systems can still have hybrid characteristics when equipped with discrete events or if controlled in a discrete manner (analogous to switched systems).
    However, continuous time is not of interest in this work and will not be discussed futher. % TODO: provide reference for continuous time systems

\stopsubsection


            \stopsection

            \startsection[title={Convex Geometry},reference=sec:theory-geometry]
                In order to describe events in and generate finite abstractions of a continuous space, a discrete representation of that space is necessary.
Descriptions in the framework of convex polytopic geometry are popular for such discretizations as they have many advantageous properties.
Convex geometry can be applied to problems of any dimension, has well understood computational properties due to its roots in linear optimization and mature libraries such as MPT \cite[authoryears][Herceg2013], TuLiP \cite[authoryears][TODO], CDD \cite[authoryears][TODO] or QHull \cite[authoryears][TODO] are freely available.
Particularly for control problems involving (piecewise) linear dynamics, convex geometry is established as a commonly used tool \cite[authoryears][Baotic2009].
It has been used e.g. by \cite[TODO], \cite[TODO], \cite[TODO] and \cite[TODO] and is likewise the foundation on which this work is build.
The word \quotation{polytope} will always refer to a convex polytope here and all polytopes mentioned are assumed to be convex.
Furthermore, only full-dimensional polytopes are considered, i.e.\ polytopes into which a non-empty ball of the dimension that the polytope is embedded in can fit.
All lower-dimensional polytopes (e.g. a line in $\reals^2$) are treated as empty.


\startsubsection[title={Polytope Representations},reference=sec:theory-geometry-representations]

    A (closed) halfspace $H \subset \reals^n$ is the set of points

    \startformula
        H = \Set{ \Vec{x} \in \reals^n \mid \VecU \cdot \VecX \leq c }
    \stopformula

    that fulfill a linear inequality governed by the normal vector $\VecU \in \reals^n$, $\VecU \neq \Vec{0}$ and an offset $c \in \reals$.
    The normal vector $\VecU$ is pointing away from the halfspace and for convenience and without restriction of generality it will always be assumed that its length is normalized, such that $\TwoNorm{\VecU} = 1$.
    Due to limitations of floating-point numbers, no distinction between closed and open halfspaces is made in this work and the closed form is generally used in the text.

    A bounded intersection of halfspaces $\IndexedSet{H_j}{j \in J}$ is a convex polytope $P$ and can be written as

    \startformula
        P = \bigcup_{j \in J} H_j = \Set{\VecX \in \reals^n \mid \MatU \VecX \leq \VecC } \EndComma % TODO find different letter than U for halfspaces everywhere (conflict with control space)
    \stopformula

    where $\Mat{U}$ is the stack of transposed normal vectors $\VecU_j$ of the halfspaces, $\Mat{C}$ the corresponding stack of offset values $c_j$ and the inequation holds component-wise.
    If the set of halfspaces is minimal, i.e.\ no halfspace can be removed without changing the bounded region, the representation is called the the H-representation of a convex polytope.
    Removing redundant halfspaces can be done by solving a series of linear programs \cite[authoryears][Baotic2009].

    A convex polytope $P$ can alternatively be defined as the convex hull

    \startformula
        P = \Hull(X) = \BigSet{ \sum_{i \in I} \lambda_i x_i \Bigmid \forall i : \lambda_i \in \ClosedInterval{0}{1}, \sum_{i \in I} \lambda_i = 1 }
    \stopformula

    of a set of points $X = \IndexedSet{x_i}{i \in I} \subset \reals^n$.
    The vertices of $P$ are the minimal set of points $\Vertices(P) \subset \reals^n$ such that $P = \Hull({\Vertices(P)})$ and uniquely define the so-called V-representation of a convex polytope.

\stopsubsection


\startsubsection[title={Operations on Convex Polytopes},reference=sec:theory-geometry-operations]

    Convex polytopes have advantageous properties making them a popular choice for practical problems in computational geometry.
    Two important advantages are that operations on convex polytopes often result in more convex polytopes and are usually very easy to express if the right representation is chosen.
    E.g., intersection of two convex polytopes is always a convex polytope and can easily be computed from the H-representations by merging the sets of bounding halfspaces and reducing to minimal form.
    The transformations between the representations are called the vertex enumeration problem (H- to V-representation) and facet enumeration problem (V- to H-representation).
    They can be handled by software packages such as CDD \cite[authoryears][TODO].

    For the purposes of this work, the following linear transformations of a convex polytope $X \subset \reals^n$ are defined:
    application of a matrix $\MatA \in \reals^{m \times n}$ from the left and translation by a vector $\VecV \in \reals^m$

    \startformula
        \MatA X + \VecV \colonequals \Set{ \VecY \in \reals^m \mid \exists \VecX \in X : \VecY = \MatA\VecX + \VecV }
    \stopformula

    and application of a matrix $B \in \reals^{n \times n}$ from the right

    \startformula
        X \MatB \colonequals \Set{ \VecY \in \reals^m \mid \MatU \MatB \VecY \leq \VecC } \EndPeriod
    \stopformula

    %The identity $X\MatB = \MatB^{-1}X$ holds for all invertible matrices $\MatB$. % TODO proof (at least a sketch)? relevant at all?
    The shorthand $-X$ is used to express the inversion operation $(-{\mathbb 1})X$, where ${\mathbb 1}$ is an identity matrix of appropropriate size.
    As evident from the definitions, matrix application from the left is easily computable with the V-representation, while the H-representation is better suited to compute the result of a matrix application from the right.
    Note that both operations may change the dimension of the polytope.

    Two binary operations are defined for convex polytopes $X, Y \subset \reals^n$. The Minkowski sum

    \startformula
        X \oplus Y \colonequals \Set{ \VecZ \in \reals^n \mid \exists \VecX \in X, \exists \VecY \in Y : \VecZ = \VecX + \VecY}
    \stopformula

    can be computed by translating every vertex of $X$ with every vertex of $Y$ and then taking the convex hull of the resulting set of points.
    The Pontryagin difference

    \startformula
        X \ominus Y \colonequals \Set{ \VecZ \in \reals^n \mid \forall \VecY \in Y : \VecZ + \VecY \in X }
    \stopformula

    can be computed by translating every halfspace of X by every vertex of Y and then taking the intersection of these halfspaces.
    While the Minkowski sum is commutative, the Pontryagin difference is not.
    Minkowski sum and Pontryagin difference are not inverse operations.
    In general it only holds that

    \startformula
        (X \ominus Y) \oplus Y \subseteq X
    \stopformula

    \cite[authoryears][Baotic2009].
    This is illustrated in Figure \in[fig:theory-geometry-operations], where a concrete counterexample is provided showing that Minkowski sum cannot generally invert a Pontryagin difference operation.

    \placefigure[top][fig:theory-geometry-operations]{
        Illustration of the Pontryagin difference (center) and Minkowski sum (right).
        Note that $(X \ominus Y) \oplus Y = Z \oplus Y \neq X$.
        Adapted from Figures 7 and 8 of \cite[Baotic2009].
    }{
        \startcombination[3*1]
            {\externalfigure[theory-geometry-operand][width=0.32\textwidth]}{}
            {\externalfigure[theory-geometry-pontryagin][width=0.32\textwidth]}{}
            {\externalfigure[theory-geometry-minkowski][width=0.32\textwidth]}{}
        \stopcombination
    }

\stopsubsection


\startsubsection[title={Non-convex Polytopic Regions},reference=sec:theory-geometry-nonconvex]

    In practice not every problem encountered is conveniently convex, but every non-convex polytopic region can be decomposed into a set of convex polytopes, e.g. by triangulation of the surface and subsequent decomposition into simplices.
    Convex polytopic geometry is therefore applicable to non-convex polytopic geometry after decomposition.
    While many operations on convex polytopes can be extended to general unions of convex polytopes in a straightforward manner, some require special care.

    Intersection can be distributed to the individual convex polytopes by taking the intersection of every polytope of one region with every polytope of the other region and keeping the non-empty intersections as the resulting union.
    The linear operations introduced above can be applied polytope-wise.
    The Minkowski sum can be distributed to the polytopes, but the result will generally not be a disjunct set of convex polytopes and may require postprocessing if this is not acceptable.
    The Pontryagin difference cannot be distributed to the convex polytopes but it is possible to express it using the Minkowski sum and set difference operations in the following way \cite[authoryears][TODO]:

    \startformula
        \startalign[n=2,align={right,left}]
            \NC X \ominus Y =
            \NC \Set{ \VecZ \in \reals^n \mid \forall \VecY \in Y : \VecZ + \VecY \in X }
            \NR
            \NC =
            \NC \Set{ \VecZ \in \reals^n \mid \forall \VecY \in Y : \VecZ + \VecY \notin (\reals^n \setminus X ) }
            \NR
            \NC =
            \NC \reals^n \setminus \Set{ \VecZ \in \reals^n \mid \neg ( \forall \VecY \in Y : \VecZ + \VecY \notin (\reals^n \setminus X ) ) }
            \NR
            \NC =
            \NC \reals^n \setminus \Set{ \VecZ \in \reals^n \mid \exists \VecY \in Y : \VecZ + \VecY \in (\reals^n \setminus X ) }
            \NR
            \NC =
            \NC \reals^n \setminus \Set{ \VecZ \in \reals^n \mid \exists \VecY \in Y \MidComma \exists \VecX \in (\reals^n \setminus X) : \VecZ + \VecY = \VecX }
            \NR
            \NC =
            \NC \reals^n \setminus \Set{ \VecZ \in \reals^n \mid \exists \VecY \in -Y \MidComma \exists \VecX \in (\reals^n \setminus X) : \VecZ = \VecX + \VecY }
            \NR
            \NC =
            \NC \reals^n \setminus ((\reals^n \setminus X) \oplus (-Y)) \EndPeriod
            \NR
        \stopalign
    \stopformula
    
    The set difference of two convex polytopes or two polytopic regions is a non-convex region in general.
    It can be computed using the regiondiff algorithm of \cite[Baotic2009], which returns the difference as set of disjunct convex polytopes.

\stopsubsection


            \stopsection

            \startsection[title={Markov Models and Probabilistic Games},reference=sec:theory-games]
                If a system of interest exhibits stochastic behaviour, there is a need for probabilistic abstractions that can reflect this stochasticity appropriately.
Markov models and the closely related probabilistic games are such models for systems with discrete time evolution.
A few of these models are introduced in this section.


\startsubsection[title={Markov Models}]

    Markov models are transition systems enriched with probabilistic behaviour and have found numerous applications, including in probabilistic model checking \cite[alternative=authoryears,left={(e.g. }][Baier2008,Svorenova2013,Chatterjee2014,Lahijanian2015].

    The simplest Markov model is the Markov chain (MC).
    A Markov chain is a tuple $\Tuple{G}{\Transition}$ where $G$ is a set of states and $\Function{\Transition}{G}{\ProbDist{G}}$ a transition relation with $\ProbDist{G}$ denoting the set of all probability distributions over $G$.
    Weith every step of a trace through the MC, a successor is chosen by sampling the probability distribution of successors of the current state defined by $\Transition$.

    Markov decision processes (MDPs) extend Markov chains by introducing actions.
    An MDP is a 3-tuple $\Triple{G}{Act}{\Transition}$, where $G$ is a set of states, $Act$ is a set of actions and $\Function{\Transition}{G \times Act}{\ProbDist{G}}$ is a probabilistic transition relation.
    With every step of a trace through the MDP, an action is selected by some decision-making process and the successor state determined by sampling the probability distribution associated with the current state and chosen action through $\Transition$.
    Not all actions are enabled in every state, but at least one action from $Act$ must be.
    A Markov chain is the degenerate case of a MDP with exactly one action enabled in every state.

    The introduction of actions in the MDP opens up a game-theoretic perspective on Markov models.
    From the perspective of games, the transition system of an MDP is a game graph on which a turn-based game is played between a player and a probabilistic environment.
    The player chooses their move each turn by picking an enabled action of the current state and the successor state is sampled by the environment from the corresponding probability distribution.
    MDPs are therefore also called 1-player probabilistic games or 1½-player games, with one \quotation{proper} player and the environment in a player-like role.
    This game-based view of Markov models extends naturally to more complicated behaviour through the introduction of additional players.

\stopsubsection


\startsubsection[title={2-Player Probabilistic Games},reference=sec:theory-games-games]

    A two-player probabilistic game, or 2½-player game, is a turn-based probabilistic game played on a game graph

    \startformula
        \GameGraph = (G_1, G_2, Act, \Transition) \EndComma
    \stopformula

    where $G_1$, $G_2$ are disjoint sets of player 1 and 2 states, respectively.
    $Act$ and $\Transition$ are defined as they were for the MDP with $G = G_1 \cup G_2$.
    A play is a sequence of states $g = g_0 g_1 ... \in G^\omega$ such that $g_i \in G_1$ for all even and $g_i \in G_2$ for all odd indices $i$, i.e. player's turns alternate beginning with player 1.
    $G^\omega$ denotes the set of all infinite sequences of members of $G$.
    When it is their turn, players choose an action that is enabled for the current game states from the set of actions $Act$.
    The next state of the play is then chosen stochastically based on the probability distribution defined by the probabilistic transition function $\Transition: G \times Act \rightarrow \mathcal{D}(G)$.
    Therefore, for every $g_i$ there must be an action $a \in Act$ such that $\Transition(g_i,\, a)(g_{i+1}) \gt 0$.

    A player $k$ strategy is a function $\Strategy{k}{\GameGraph}: G^+ \rightarrow Act$ that determines the action taken after a finite prefix of a play ending in a state of player $k$.
    Analogous to strategies for linear stochastic systems, a strategy that requires a fixed-size prefix to determine an action is called finite-memory while a strategy using only the current game state for the action selection is called memoryless.

\stopsubsection


\startsubsection[title={Winning and Solving 2-Player Probabilistic Games}]

    A notion of winning is introduced by extending the game graph $\mathcal{G}$ with an acceptance condition $\mathcal{C}$ (also called winning condition) to form the game

    \startformula
        \GameGraph' = (G_1, G_2, Act, \Transition, \Condition) \EndPeriod
    \stopformula

    The acceptance condition separates the set of all possible plays into those which are won by player 1 and those plays are won by player 2.
    Every play has exacly one winning player, no plays are won by neither or both players.
    Acceptance conditions of games are analogous to those of \omega-automata and are discussed further in section \in[sec:theory-automata-acceptance].

    The solution of a game are the sets of initial states for which player 1 (2) has a winning strategy, which is a strategy that player 1 (2) can use to ensure winning under some circumstances.
    The circumstances depend on the type of game analysis that is carried out.
    In this work, (qualititive) almost-sure analysis for an adversarial and cooperative player 2 is the main concern, i.e.\ game states are sought for which a player 1 strategy exists that leads to player 1 winning the game with probability 1.
    In the adversarial setting, the strategy must lead to a player 1 win for every possible strategy of the opponent, while in the cooperative setting an almost-sure winning strategy only has to exist for some strategy of the other player, meaning that player 1 can win if the opponent is \quotation{nice} and cooperates.
    The set of initial states for which player 1 has an almost-sure winning strategy are denoted in the adversarial setting by $\Almost{\GameGraph'}$ and in the cooperative setting as $\AlmostCoop{\GameGraph'}$.

\stopsubsection


            \stopsection

            \startsection[title={Automata and Languages},reference=sec:theory-automata]
                Automata provide a way to represent a language.
A Language contains words made up from symbols originating from an alphabet.
An automaton is a transition system constructed such that any given word induces a run, which is either accepted or rejected.
All words inducing accepted runs are part of the language of the automaton, while rejected words are not.
Many different types of automata exist for many different types of languages.
In this work, deterministic finite-state automata (DFAs) and languages containing infinite words are of interest, for which \omega-automata are appropriate.

Before defining \omega-automata, some notation for working with languages is introduced:
The language of an automaton ${\mathcal A}$ is denoted as $\Language({\mathcal A})$.
A word is written as a sequence of symbols, e.g. $w_0 w_1 w_2$.
The set of all finite words over an alphabet $\Sigma$ is written as $\RepeatFinitely{\Sigma}$, while the set of all infinite words is written as $\RepeatInfinitely{\Sigma}$.
This notation is more generally used for sets of sequences and for (\omega-)regular expressions over an alphabet, where it represents finite ($\RepeatFinitely{\,}$) or infinite ($\RepeatInfinitely{\,}$) recurrence of a symbol (consult chapter 4 in the book of \cite[Baier2008] for a more thorough introduction to regular expressions).


\startsubsection[title={\omega-Automata}]

    A deterministic \omega-automaton is type of DFA, specified by a tuple 

    \startformula
        \mathcal{A} = (Q, \Sigma, \delta, q_0, \mathcal{C}) \EndComma
    \stopformula

    where
    $Q \neq \emptyset$ is a finite set of states,
    $\Sigma$ is a finite alphabet of symbols,
    $\delta: Q \times \Sigma \rightarrow Q$ is a deterministic transition function,
    $q_0 \in Q$ is the initial state of the automaton.
    If the transition function is defined for every combination of state from $Q$ and symbol from $\Sigma$, the automaton is called complete otherwise is is called incomplete.

    A run is an infinite sequence of states $q_0 q_1 ... \in \RepeatInfinitely{Q}$ such that $q_{i+1} = \delta(q_i, w_i)$, $w_i \in \Sigma$ for all $i \in \naturalnumbers_0$.
    Every run is associated with a word $w = w_0 w_1 ... \in \RepeatInfinitely{\Sigma}$, induced by the symbols that label the transitions.
    The purpose of the acceptance condition $\mathcal{C}$ is to specify which words are accepted by the automaton and therefore part of $\Language({\mathcal A})$ and which which words are rejected.
    It can be expressed in different ways.

\stopsubsection


\startreusableMPgraphic{theory-automaton-example}
    beginfig(0);
        with spacing((30,10)) matrix.a(3,8);
        node_dash.a[1][1](btex $q_0$ etex);
        node.a[1][4](btex $q_1$ etex);
        node_double.a[1][7](btex $q_2$ etex);
        incoming(0, "") (a[1][1]) 180;
        loop.rt(.4, btex \;$a$ etex) (a[1][1]) 90;
        arrow.top(.5, btex $b$ etex) (a[1][1],a[1][4]) a[1][1].c..a[1][4].c;
        arrow.top(.5, btex $c$ etex) (a[1][4],a[1][7]) a[1][1].c..a[1][7].c;
        loop.rt(.4, btex \;$d$ etex) (a[1][7]) 90;
    endfig;
\stopreusableMPgraphic

\startsubsection[title={Acceptance Conditions}]

    For a run to be accepted by the automaton it has to start in the initial state $q_0$ and all transitions required to realize the run have to be defined by the transition function.
    In case of an incomplete automaton, all words that require a transition not defined by $\delta$ in their corresponding run are rejected.
    In addition to these basic requirements of accepted runs, the acceptance condition specifies further restrictions.

    Formally, the acceptance condition ${\mathcal C}$ divides the set of all runs into a set containing all accepted runs and a set containing all not accepted runs.
    A word is called accepted if the corresponding run is accepted.
    The set of all accepted words is the language of the automaton.
    Since there are usually infinitely many accepted words, the acceptance condition is expressed by other means than a set of words.

    A Büchi condition is expressed by a set of states and requires that at least one state from this set is visited infinitely often in every accepted run.
    A generalization of the Büchi acceptance condition is the Büchi-implication condition, also known as the one-pair Streett condition.
    It is specified by a tuple of sets $\mathcal{C} = (E, F) \subseteq Q \times Q$ and accepts all runs $r$ where

    \startformula
        (\InfinitelyOften{r} \cap E \ne \emptyset) \Longrightarrow (\InfinitelyOften{r} \cap F \ne \emptyset) \EndComma
    \stopformula

    where the set of states which occur infinitely often in the run $r$ is denoted by $\InfinitelyOften{r}$.
    I.e., in every accepted run an infinite occurrence of at least one state from $E$ implies that at least one state from $F$ is also visited infinitely often.

    \placefigure[top][fig:theory-automaton-example]{
        An automaton with the one-pair Streett acceptance condition $(\Set{q_0}, \Set{q_2})$.
    }{
        % Put in wide box so that figure caption has proper width
        \framed[width=\textwidth,frame=off]{\reuseMPgraphic{theory-automaton-example}}
    }

    For example, consider the automaton with one-pair Streett acceptance condition

    \startformula
        (\Set{q_0, q_1, q_2}, \Set{a, b, c, d}, \delta, q_0, (\Set{q_0}, \Set{q_2})) \EndComma
    \stopformula

    where the transition function defines $\delta(q_0, a) = q_0$, $\delta(q_0, b) = q_1$, $\delta(q_1, c) = q_2$ and $\delta(q_2, d) = q_2$.
    This automaton accepts all words starting with a finite number of $a$s, followed by a single $b$, a single $c$ and then $d$s forever, i.e. all words of the form $\RepeatFinitely{a} b c \RepeatInfinitely{d}$.
    It is shown in Figure \in[fig:theory-automaton-example], which demonstrate show states from one-pair Streett acceptance sets $E$ (dashed border) and $F$ (double border) are highlighted in automaton depictions here.

\stopsubsection


            \stopsection

            \startsection[title={Temporal Logic},referenc=sec:theory-logic]
                A temporal logic is an extension of propositional logic by a set of temporal connectives, enabling the expression of statements about not just the current but also the future state of a system.
While formulae of propositional logic are evaluated with a single valuation, temporal logic formulae are evaluated with a sequence or a tree of valuations, describing the evolution of the system state in time.

A comprehensive and rigorous introduction to temporal logics is given e.g. by \cite[Baier2008]. % TODO another reference
Here, a summarized look at the basics of linear time logic is presented and supplemented by brief survey of related temporal logics.


\startsubsection[title={Linear Temporal Logic}]

    Linear temporal logic (LTL) is an extension of propositional logic applied to discrete-time infinite sequences of valuations of propositional atoms.
    LTL formulae are able to express the temporal order of events occurring in such a sequence.
    As the name suggests, LTL is concerned with a linear, path-based understanding of time, considering one specific path through time when evaluated, with exactly one successor for each state in the sequence.

    The syntax of LTL formulae over a set of atomic propositions $AP$ follows the grammar
    
    \startformula
        \varphi \coloncolonequals \True \mid a \mid \varphi_1 \wedge \varphi_2 \mid \neg \varphi \mid \Next \varphi \mid \varphi_1 \Until \varphi_2 \EndComma
    \stopformula

    where $a \in AP$.
    $\wedge$ and $\neg$ are the known propositional operators \quotation{and} and \quotation{not}.
    $\Next$ and $\Until$ are temporal operators named \quotation{next} and \quotation{until}, respectively.
    $\True$ is the tautology symbol.
    Operator precedence is adopted from propositional logic where possible, $\Next$ binds as strongly as $\neg$ and $\Until$ takes precedence over all propositional binary operators.

    LTL formulae are interpreted over paths $w = w_0 w_1 ... \in (2^{AP})^\omega$ which are sequences of valuations and the words in a formula's language over the alphabet $AP$.
    The semantics of LTL, expressed by the satisfaction relation $\vDash$, is 

    \startformula
        \startalign[n=3,align={right,left,left}]
            \NC w \vDash \NC \True \NC ~~\text{unconditionally (tautology),} \NR
            \NC w \vDash \NC a \NC \iff a \in w_0 \EndComma \NR
            \NC w \vDash \NC \neg \varphi \NC \iff w \nvDash \varphi \EndComma \NR
            \NC w \vDash \NC \varphi_1 \wedge \varphi_2 \NC \iff w \vDash \varphi_1 \;\text{and}\; w \vDash \varphi_2 \EndComma \NR
            \NC w \vDash \NC \Next \varphi \NC \iff w_1 w_2 ... \vDash \varphi \EndComma \NR
            \NC w \vDash \NC \varphi_1 \Until \varphi_2 \NC \iff \exists j \ge 0 : w_j w_{j+1} ... \vDash \varphi_2 \;\text{and}\; \forall 0 \le i \lt j : w_i w_{i+1} ... \vDash \varphi_1 \EndPeriod \NR
        \stopalign
    \stopformula

    Other connectives are derived, such as

    \startformula
        \startalign[n=3,align={right,left,middle}]
            \NC \varphi_1 \vee \varphi_2 \NC \colonequals \neg(\neg \varphi_1 \wedge \neg \varphi_2) ~~~\NC \text{\quotation{or},} \NR
            \NC \Finally \varphi \NC \colonequals \True \Until \varphi \NC \text{\quotation{finally},} \NR
            \NC \Globally \varphi \NC \colonequals \neg \Finally \neg \varphi \NC \text{\quotation{globally}.} \NR
        \stopalign
    \stopformula

    Paths are usually derived from a transition system which might have more than one possible successor to a state.
    Therefore multiple futures, i.e. multiple paths, are realizable when starting in such states.
    The sematics of LTL is extended to a state-based notion in the following way:

    \startformula
        s \vDash \phi \iff \forall w ~\text{starting in}~ s: w \vDash \phi \EndPeriod
    \stopformula

    If and only if all possible paths starting in some state $s$ satisfy an LTL formula $\varphi$, $s$ is said to satisfy the the formula.

\stopsubsection


\startsubsection[title={Other Temporal Logics}]

    The path based, linear time semantics of LTL was extended to states by looking at all possible paths starting from a common initial state.
    In a branching time framework, the future is not a collection of paths starting in the present but a tree of possible futures, branching further outward whenever there are multiple successors of a moment (i.e. state) in time.
    Computation Tree Logic (CTL) is a temporal logic with this understanding of time and allows reasoning over the branching future with quantifiers $\exists$ (there is some future) and $\forall$ (for all possible futures).
    The expressiveness of CTL only partially overlaps with LTL, i.e. some properties can be expressed by both LTL and CTL while some properties can only be expressed with LTL and some properties can only be expressed by CTL. % TODO reference
    Despite the overlapping expressiveness, model checking approaches for linear and branching time properties are quite different. % TODO is this interesting information? Yes, if reference(s) for examples are added
    Both LTL and CTL are unified in the temporal logic CTL*, that additionally extends the expressiveness of both.

    LTL and CTL are only concerned with the ordering of events in time but one might want to formulate properties that include the distance between events in time. % TODO phrasing
    Temporal logic can be equipped with the concept of a clock that measures time in discrete units and operators that depend on the amount of time as measured by this clock.
    Therefore, properties such as \quotation{in x units of time} or \quotation{for x units of time} can be expressed.
    Timed CTL \cite[authoryears][Baier2008] or Metric Time Logic \cite[authoryears][TODO] are examples of temporal logics that include this concept of time intervals.

    Not only the view of time (linear, branching) but also the properties of the transition system under consideration influence what is expressible in a temporal logic. % TODO phrasing, this is really about the "construction" of a temporal logic
    For example, if the transition system exhibits probabilistic behaviour and/or associates rewards to paths, one should be able to express qualitative and/or quantitative probabilistic and/or reward properties, too.
    Therefore many variants of temporal logics for such purposes have been constructed, e.g. Probabilistic CTL or Probabilistic Reward CTL \cite[authoryears][Baier2008].

\stopsubsection


\startreusableMPgraphic{automaton-reachability}
    with spacing((30,10)) matrix.a(3,6);
    node_dash.a[1][1](btex $q_0$ etex);
    node_double.a[1][4](btex $q_1$ etex);
    incoming(0, "") (a[1][1]) 180;
    loop.rt(.4, btex \small ${\neg \varphi}$ etex) (a[1][1]) 90;
    arrow.top(.5, btex \small ${\varphi}$ etex) (a[1][1],a[1][4]) a[1][1].c..a[1][4].c;
    loop.rt(.4, btex \small ${\True}$ etex) (a[1][4]) 90;
\stopreusableMPgraphic

\startreusableMPgraphic{automaton-avoidance}
    with spacing((30,10)) matrix.a(3,6);
    node_dash.a[1][1](btex $q_0$ etex);
    node_double.a[1][4](btex $q_1$ etex);
    incoming(0, "") (a[1][1]) 180;
    loop.rt(.4, btex \small ${\neg \theta}$ etex) (a[1][1]) 90;
    arrow.top(.5, btex \small ${\varphi}$ etex) (a[1][1],a[1][4]) a[1][1].c..a[1][4].c;
    loop.rt(.4, btex \small ${\True}$ etex) (a[1][4]) 90;
\stopreusableMPgraphic

\startreusableMPgraphic{automaton-recurrence}
    with spacing((30,10)) matrix.a(3,6);
    node_dash.a[1][1](btex $q_0$ etex);
    node_double.a[1][4](btex $q_1$ etex);
    incoming(0, "") (a[1][1]) 180;
    loop.rt(.4, btex \small ${\neg \varphi}$ etex) (a[1][1]) 90;
    arrow.top(.5, btex \small ${\varphi}$ etex) (a[1][1],a[1][4]) a[1][1].c..a[0][2].c..a[0][3].c..a[1][4].c;
    loop.rt(.4, btex \small ${\varphi}$ etex) (a[1][4]) 90;
    arrow.bot(.5, btex \small ${\neg \varphi}$ etex) (a[1][4],a[1][1]) a[1][4].c..a[2][3].c..a[2][2].c..a[1][1].c;
\stopreusableMPgraphic

\startreusableMPgraphic{automaton-safety}
    with spacing((30,10)) matrix.a(3,3);
    node.a[1][1](btex $q_0$ etex);
    incoming(0, "") (a[1][1]) 180;
    loop.rt(.4, btex \small ${\neg \theta}$ etex) (a[1][1]) 90;
\stopreusableMPgraphic

\startreusableMPgraphic{automaton-eventualsafety}
    with spacing((30,10)) matrix.a(3,6);
    node_dash.a[1][1](btex $q_0$ etex);
    node.a[1][4](btex $q_1$ etex);
    incoming(0, "") (a[1][1]) 180;
    loop.rt(.4, btex \small ${\,\theta}$ etex) (a[1][1]) 90;
    arrow.top(.5, btex \small ${\neg \theta}$ etex) (a[1][1],a[1][4]) a[1][1].c..a[0][2].c..a[0][3].c..a[1][4].c;
    loop.rt(.4, btex \small ${\neg \theta}$ etex) (a[1][4]) 90;
    arrow.bot(.5, btex \small ${\theta}$ etex) (a[1][4],a[1][1]) a[1][4].c..a[2][3].c..a[2][2].c..a[1][1].c;
\stopreusableMPgraphic

\startsubsection[title={Automata for LTL Objectives}]

    Every LTL formula can be represented by an \omega-automaton constructed such that the formula and automaton accept the same language of words over the alphabet $2^{AP}$.
    Tools for the automatic translation of temporal logic formulae to various types of \omega-automata have beed developed, e.g. by \cite[Kretinsky2018], \cite[Duret2016] and \cite[Gastin2001].

    Table \in[tab:theory-logic-objectives] presents five basic LTL formulas and corresponding \omega-automata with one-pair Streett acceptance conditions.
    The automata for objectives reachability, recurrence and eventual safety are complete, whereas reachability/avoidance and safety are incomplete.
    Noteworthy is the Streett pair of the eventual safety automaton., which expresses that state $q_0$ must not be visited infinitely often.
    The empty set in the pair cannot be visited infinitely often, therefore the condition can only be fulfilled by visiting $q_0$ only a finitely often.

    \placetable[top][tab:theory-logic-objectives]{
        Description of four basic linear time objectives and their corresponding \omega-automata with acceptance condition.
    }{
        \setupTABLE[frame=off,option=stretch]
        \setupTABLE[r][each][bottomframe=on]
        \setupTABLE[r][last][bottomframe=off]
        \setupTABLE[c][1][align={justified,lohi}]
        \setupTABLE[c][2][toffset=2mm,align={middle,lohi}]
        \bTABLE
            \bTR
                \bTD \underbar{Reachability}: $\Finally \varphi$. \par Eventually reach a state where $\varphi$ is satisfied. \par One-pair Streett condition: $(\Set{q_0}, \Set{q_1})$ \eTD
                \bTD {\leavevmode\reuseMPgraphic{automaton-reachability}} \eTD
            \eTR
            \bTR
                \bTD \underbar{Reachability/Avoidance}: $\neg \theta \Until \varphi$. \par Avoid satisfying $\theta$ until a state satisfying $\varphi$ is reached. \par One-pair Streett condition: $(\Set{q_0}, \Set{q_1})$ \eTD
                \bTD {\leavevmode\reuseMPgraphic{automaton-avoidance}} \eTD
            \eTR
            \bTR
                \bTD \underbar{Recurrence}: $\Globally \Finally \varphi$. \par Satisfy $\varphi$ again and again. \par One-pair Streett condition: $(\Set{q_0}, \Set{q_1})$ \eTD
                \bTD {\leavevmode\reuseMPgraphic{automaton-recurrence}} \eTD
            \eTR
            \bTR
                \bTD \underbar{Safety}: $\Globally \neg \theta$. \par Forever avoid states where $\theta$ is satisfied. \par Acceptance expressed through incompletness of $\delta$. \eTD
                \bTD {\leavevmode\reuseMPgraphic{automaton-safety}} \eTD
            \eTR
            \bTR
                \bTD \underbar{Eventual Safety}: $\Finally \Globally \neg \theta$. \par Eventually, $\theta$ can never be satisfied. \par One-pair Streett condition: $(\Set{q_0}, \emptyset)$ \eTD
                \bTD {\leavevmode\reuseMPgraphic{automaton-eventualsafety}} \eTD
            \eTR
        \eTABLE
    }

\stopsubsection


\startsubsection[title={Fragments of LTL}]

    Model checking with general LTL specifications is computationally demanding and in fact PSPACE-hard \cite[authoryears][Baier2008].
    Fragments of LTL exist that offer more favorable computational complexity and simple automaton constructions at the cost of restrictions in expressivity.
    However, if designed carefully, such restricted subsets of LTL can still include many properties that are relevant in practical applications.

    One such fragment is the subset of Generalized Ractivity(1) formulas of form

    \placeformula[fml:theory-logic-gr]
    \startformula
        \bigwedge_{i = 1 ... n} \Globally \Finally \mu_i \rightarrow \bigwedge_{i = 1 ... m} \Globally \Finally \pi_i
    \stopformula

    where $\mu_i$ and $\pi_i$ are propositional formulas \cite[authoryears][TODO].

    It was first introduced by \cite[TODO] and is used frequently in robotics applications \cite[authoryears][TODO].
    The restriction to formulas of form (\in[fml:theory-logic-gr]) allows for the use of model checking algorithms with polynomial complexity.
    An extended variant of GR(1) contains all formulas with form

    \placeformula[fml:theory-logic-grextended]
    \startformula
        \Big( \displaystyle\bigwedge_{i = 1 ... n} \mu_i \Big) \Longrightarrow \Big( \displaystyle\bigwedge_{i = 1 ... m} \pi_i \Big) \EndComma
    \stopformula

    where every \math{\mu_i} and \math{\pi_i} is representable by a deterministic \omega-automaton with Büchi acceptance condition.
    For the entire GR(1) formula, an \omega-automaton with one-pair Streett acceptance condition can then be constructed based on the individual Büchi automata for the terms $\mu_i$ and $\pi_i$. % TODO: reference, Svorenova?
    Therefore, all objectives in Table \in[tab:theory-logic-objectives] are part of the (extended) GR(1) fragment.
    In this work GR(1) always refers to the extended variant.

    Linear temporal logic as defined above is interpreted over infinite sequences.
    However, for a subset of formulas, satisfiability can be decided based on a finite prefix of the sequence.
    These formulas from the class of co-safe LTL \cite[authoryears][TODO].
    % TODO see [5] of Lacerda 2014, syntactic restrictions
    Co-safe objectives occur in many real-world scenarios, e.g.\ in robotics where typical objectives of a robot are usually not infinite procedures or allowed to take infinite time, but are isolated, finite tasks that can be chained to form more complex behaviour. % TODO: references that use co-safe objectives
    In Table \in[fig:tab-theory-logic-objectives]), both reachability and reachability/avoidance are co-safe objectives.

    %Co-safe objectives are also useful when dealing with problems where some accumulated cost is supposed to be minimized.
    %In an infinite setting costs might accumulate forever and not yield a finite value that can be optimized. % TODO: reference (Lacerda 2014?)

\stopsubsection


            \stopsection

        \stopchapter

        \startchapter[title={Problem Formulation}]

            As this is a continuation of work done by \cite[Svorenova2017], the same problem is posed:
Analysis and controller synthesis in the framework of linear stochastic systems and GR(1) objective specifications.



            \startsection[title={System Setup},reference=sec:problem-setup]
                LSS from section \in[sec:theory-hybrids-lss]:

\placeformula[fml:problem-setup-lsseq]
\startformula
    \VecX_{t+1} = \MatA \VecX_t + \MatB \VecU_t + \VecW_t \EndComma
\stopformula

corresponds to discrete-time, continuous-space MDP.

State space, control space and random space are convex polytopes (simplification, non-convex shapes can always be generated through the introduction of safety/avoidance objectives).
Traces are defined in the state space and fulfil the evolution equation at every step, control strategies as TODO.

Objective from GR(1) LTL or generally every objective expressible by a one-pair Streett automaton, linked to LSS by a set of linear predicates.


            \stopsection

            \startsection[title={Problem Statement},reference=sec:problem-statement]
                Two related questions are posed about the problem setup.
The first asks for the existence of controllers that ensure some specified behaviour of traces in the system.
The second requires the construction of such controllers.
These problems were asked before by \cite[Svorenova2017] and are repeated here.


\startsubsection[title={Satisfiability Analysis}]

    The first problem considered is an analysis of the system \in[fml:problem-setup-lsseq] with a temporal logic objective $\varphi$.
    The aim of the analysis is to identify the set of initial states $\InitialStates$ for which a control strategy exists such that the objective $\varphi$ can be fulfilled with probability $1$ when starting a trace from within $\InitialStates$.
    This is called almost-sure analysis.
    Some authors \cite[alternative=authoryears,left={(e.g.\ }][Lahijanian2015] use the term \quotation{verification} instead of \quotation{analysis}.

    The specification $\varphi$ is formulated using the linear predicates in $\Predicates$ as atomic propositions.
    Here, objectives are limited to formulas from the extended GR(1) fragment of LTL, i.e. all objectives that can be expressed with an \omega-automaton with one-pair Streett acceptance condition (section \in[sec:theory-logic-fragments]).
    If a formula allows a co-safe interpretation in addition to an infinite interpretation, both are considered.
    In the infinite interpretation the trace is never allowed to leave the state space, whereas in the co-safe interpretation the trace can go anywhere once the objective has been satisfied.

    A notable special case is reachability analysis using the co-safe objective $\varphi = \Finally \phi$, where $\phi$ is a proposition over $\Predicates$.
    The specification is fulfilled when a trace reaches the region specified by $\phi$.
    Reachability analysis was studied in-depth by \cite[Svorenova2017] and is of great importance here as well.
    As shown in section \in[sec:refinement-reachability], reachability is the central building block of solutions for more complex problems.

\stopsubsection


\startsubsection[title={Controller Synthesis}]

    The second problem considered is the immediate follow-up problem to an almost-sure analysis: controller synthesis.
    In the analysis, initial states are identified from which traces can be controlled such that the objective is fulfilled with probability 1.
    The aim of controller synthesis is to constructively find such a controller.

\stopsubsection


            \stopsection

            \startsection[title={Solution Approach},reference=sec:problem-approach]
                \startreusableMPgraphic{flowchart-approach}
    with spacing((25,10)) matrix.a(11,15);
    with shape(none) node.a[0][0](btex \ssd \strut LSS etex);
    with shape(none) node.a[0][3](btex \ssd \strut linear predicates etex);
    with shape(none) node.a[0][9](btex \ssd \strut objective etex);
    with shape(circle) with size(3) node.a[1][0]("");
    with shape(circle) with size(3) node.a[1][2]("");
    with shape(circle) with size(3) node.a[1][9]("");
    with shape(boxed) with border(none) node.a[3][12]("");
    with shape(boxed) with border(none) node.a[3][12]("");
    with shape(boxed) with border(none) node.a[5][12]("");
    with fixedboxwidth(80) with fixedboxheight(30) with shape(roundfixedbox) with filling(solid) with fillingcolor(lightgray) node.a[4][1](btex \ss Abstraction etex);
    with fixedboxwidth(80) with fixedboxheight(30) with shape(roundfixedbox) with filling(solid) with fillingcolor(lightgray) node.a[4][9](btex \ss Analysis etex);
    with fixedboxwidth(80) with fixedboxheight(30) with shape(roundfixedbox) with filling(solid) with fillingcolor(lightgray) node.a[10][5](btex \ss Refinement etex);
    arrow.rt(.5, "") (a[1][0],a[4][1]) a[1][0].c..a[4][0].c;
    arrow.rt(.5, "") (a[1][2],a[4][1]) a[1][2].c..a[4][2].c;
    arrow.rt(.5, "") (a[1][9],a[4][9]) a[1][9].c..a[4][9].c;
    arrow.top(.5, btex \ssd game graph etex) (a[4][1],a[4][9]) a[4][1].c..a[4][9].c;
    arrow.bot(.5, btex \ssd undecided states etex) (a[4][9],a[10][5]) a[4][9].c---a[10][9].c---a[10][5].c;
    arrow.bot(.5, btex \ssd state space partition etex) (a[10][5],a[4][1]) a[10][5].c---a[10][1].c---a[4][1].c;
    arrow.rt(1, btex \ssd part of ${\InitialStates}$ etex) (a[4][9],a[3][12]) a[3][9].c---a[3][12].c;
    arrow.rt(1, btex \ssd part of ${\StateSpace \setminus \InitialStates}$ etex) (a[4][9],a[5][12]) a[5][9].c---a[5][12].c;
\stopreusableMPgraphic


\placefigure[top][fig:review-flowchart-approach]{
    TODO
}{
    \framed[width=\textwidth,frame=off]{\reuseMPgraphic{flowchart-approach}}
}

\cite[Svorenova2017] presented a game-based solution to the (co-safe) almost-sure analysis problem which is repeated in the following chapter.
The solution approach to the analysis problem is depicted in Figure \in[fig:review-flowchart-approach].

Convex discretization, initially with linear predicates.
LSS is continuous-space MDP, suitable abstraction is 2-player probabilistic game, second player represents non-determinism introduced by abstracted states (one doesn't know where exactly in the state one is).
Almost-sure analysis: product of automaton and game, convert to parity objective, adversarial and cooperative solution.
Iterative refinement with heuristics informed by dynamics and analysis results.
Strategy synthesis makes use of player 1 actions of the abstraction and analysis results to select control inputs.

\cite[Svorenova2017] showed that a non-randomized (i.e.\ pure), finite-memory controller ensuring the specification for all traces starting in $\InitialStates$ can be synthesized from the analysis results.
However, this controller is not \quotation{optimal} in practice and can be improved upon.
The issue of controller \quotation{optimality} will be discussed briefly in section \in[sec:abstraction-synthesis].
A demonstration of a slightly improved improved controller for reachability problems is given in the corridor case study (section \in[sec:cases-corridor]).


            \stopsection

        \stopchapter

        \startchapter[title={Abstraction and Analysis}]

            The abstraction and analysis phases of the solution procedure outlined in \in{Section}[sec:problem-approach] are described in this chapter, followed by a brief discussion of the controller synthesis problem.
The solution was first proposed and implemented by \cite[Svorenova2017].
It is reviewed here.

First, a set of polytopic operators capturing important aspects of the dynamics is introduced in \in{Section}[sec:abstraction-operators].
A 2½-player game graph is then constructed from the linear stochastic system in \in{Section}[sec:abstraction-graph].
This game graph is synchronized with a deterministic \omega-automaton obtained from the GR(1) specification in \in{Section}[sec:abstraction-product].
\in{Section}[sec:abstraction-analysis] demonstrates how to obtain a solution to the analysis problem from the synchronized game.
Based on this solution, an almost-sure winning control strategy for player 1 is synthesized in \in{Section}[sec:abstraction-synthesis].

Throughout this chapter the simple, 1-dimensional LSS 

\placeformula[fml:abstraction-example]
\startformula
    \VecState_{t+1} = \VecState_t + \VecControl_t + \VecRandom_t \EndComma
\stopformula

where $\VecState_t \in \StateSpace = \ClosedInterval{0}{4}$, $\VecRandom_t \in \RandomSpace = \ClosedInterval{-0.1}{0.1}$ and $\VecControl_t \in \ControlSpace = \ClosedInterval{-1}{1}$ for all $t$, is used to illustrate the construction of the solution.



            \startsection[title={Dynamics Operators},reference=sec:abstraction-operators]
                A meaningful game-based abstraction of a linear stochastic system must reflect its dynamics.
Here, operators on the state- and control space are defined.
Each operator expresses an aspect of characteristic one-step behaviour of the system dynamics.
All operator introductions are complemented by descriptions of their computation based on the polytopic operations from \in{Section}[sec:theory-geometry].

Because all operators are based on the dynamical properties of the LSS, they require knowledge of the LSS parameters, i.e.\ $\MatA$, $\MatB$, $\StateSpace$, $\RandomSpace$, $\ControlSpace$.
The association between an operator and its LSS is not captured explicitly by the presented notation and should always be clear from context that the operator is used in.


\startsubsection[title={Posterior}]

    The posterior ($\Post$) is a forward-looking operator.
    Given an origin region $\StateRegion \subseteq \StateSpace$ and control input region $\ControlSpace' \subseteq \ControlSpace$, it computes the set of states which are reachable from the origin region under the control inputs with non-zero probability.
    The returned region is therefore the one-step reachable set

    \startformula
        \Posterior{\StateRegion}{\ControlRegion} := \Set{ \VecState \in \reals^n \mid \exists \VecState' \in \StateRegion\MidComma \exists \VecControl' \in \ControlRegion\MidComma \exists \VecRandom \in \RandomSpace : \VecState = \MatA \VecState' + \MatB \VecControl' + \VecRandom } \EndPeriod
    \stopformula

    The posterior allows the computation of an extended state space $\ExtendedStateSpace = \StateSpace \,\cup\, \Posterior{\StateSpace}{\ControlSpace}$, which is the union of the original state space and its one-step reachable set.

    $\Post$, as defined above, takes set-valued arguments.
    The notation of this operator and all following operators is additionally overloaded for vector-valued state- and control-space arguments as well as sets of elements of a state space decomposition in the following way:
    Let $\VecState' \in \StateSpace$, $\VecControl' \in \ControlSpace$ and $\State{j} \subseteq \StateSpace$ for all $j \in J$.
    Then

    \startformula
        \startalign[n=2,align={right,left}]
            \NC \Posterior{\VecState'}{\ControlRegion} :=
            \NC \Posterior{\Set{\VecState'}}{\ControlRegion} \EndComma
            \NR
            \NC \Posterior{\StateRegion}{\VecControl'} :=
            \NC \Posterior{\StateRegion}{\Set{\VecControl'}} \EndAnd
            \NR
            \NC \Posterior{\IndexedStates{j}{J}}{\ControlRegion} :=
            \NC \BigPosterior{\bigcup_{j \in J}\State{j}}{\ControlRegion} \EndPeriod
            \NR
        \stopalign
    \stopformula

    For polytopic inputs, $\Post$ can be computed with the Minkowski sum as

    \startformula
        \Posterior{\StateRegion}{\ControlSpace'} = \MatA \StateRegion \oplus \MatB \ControlRegion \oplus \RandomSpace \EndPeriod
    \stopformula

\stopsection


\startsubsection[title={Predecessors}]

    As the name suggests, predecessors are backward-looking operators, computing origin regions with specific properties for a given state space target and control input.
    The predecessor ($\Pre$) and robust predecessor ($\PreR$) are defined as

    \startformula
        \startalign[n=2,align={right,left}]
            \NC \Predecessor{\StateRegion}{\ControlSpace'}{\StateTarget} :=
            \NC \Set{ \VecState' \in \StateRegion \mid \exists \VecControl' \in \ControlSpace' : \Posterior{\VecState'}{\VecControl'} \cap \StateTarget \neq \emptyset } \EndAnd
            \NR
            \NC \RobustPredecessor{\StateRegion}{\ControlSpace'}{\StateTarget} :=
            \NC \Set{ \VecState' \in \StateRegion \mid \exists \VecControl' \in \ControlSpace' : \Posterior{\VecState'}{\VecControl'} \subseteq \StateTarget } \EndComma
            \NR
        \stopalign
    \stopformula

    where the second argument $\ControlRegion \subseteq \ControlSpace$ is a control-space region, the third argument $\StateTarget \subseteq \ExtendedStateSpace$ is a target region in the extended state space and the first parameter $\StateRegion \subseteq \StateSpace$ is a region of the state space to which the returned predecessor is restricted and exists mainly for convenience.

    From any state in a predecessor set, the specified target region is reachable with non-zero probability in one step of the system evolution for some control input in $\ControlRegion$.
    The control inputs that enable these transitions can be different for every state in the predecessor origin region.
    The probability of reaching the target region in case of the robust predecessor is 1.
    The robust predecessor is therefore robust in the sense that the target will be reached exclusively from the computed origin region under the given control inputs, irrespective of the stochastic dynamics.
    Trivially, it holds that

    \startformula
        \RobustPredecessor{\StateRegion}{\ControlSpace'}{\StateTarget} \subseteq \Predecessor{\StateRegion}{\ControlSpace'}{\StateTarget} \EndPeriod
    \stopformula

    For target regions where $\StateTarget \ominus \RandomSpace = \emptyset$, the $\PreR$ is always empty.
    This can easily be seen from the operator's computations based on polytopic operators:

    \startformula
        \startalign[n=2,align={right,left}]
            \NC \Predecessor{\StateRegion}{\ControlSpace'}{\StateTarget} =
            \NC \StateRegion \cap \Big( \StateTarget \oplus -(\MatB \ControlRegion \oplus \RandomSpace) \Big) \MatA \EndAnd
            \NR
            \NC \RobustPredecessor{\StateRegion}{\ControlSpace'}{\StateTarget} =
            \NC \StateRegion \cap \Big( (\StateTarget \ominus \RandomSpace) \oplus -\MatB \ControlRegion \Big) \MatA \EndPeriod
            \NR
        \stopalign
    \stopformula

    Finally, the precise predecessor is defined.
    This operator only makes sense when used when applied to elements of a state space partition:

    \startformula
        \startalign[n=2,align={right,left}]
            \NC \PrecisePredecessor{\StateRegion}{\ControlSpace'}{\IndexedSet{\State{j}}{j \in J}} := \Big\{ \VecState \in \StateRegion \Bigmid \exists \VecControl' \in \ControlSpace' :
            \NC \Posterior{\VecState}{\VecControl'} \subseteq \bigcup_{j \in J} \State{j} \;\text{and}
            \NR
            \NC \empty
            \NC ~\forall j \in J : \Posterior{\VecState}{\VecControl'} \cap \bigcup_{j \in J} \State{j} \neq \emptyset \Big\} \EndPeriod
            \NR
        \stopalign
    \stopformula

    Every state in the resulting origin set fulfills the robust predecessor property with respect to the entire target region $\bigcup_{j \in J} \State{j}$ for some $\VecControl' \in \ControlRegion$, while simultaneously fulfilling the predecessor property with respect to every individual part of the target region for the same $\VecControl'$.
    In other words, for any state in a $\PreP$ set, a control input in $\ControlRegion$ exists such that both a state in $\bigcup_{j \in J} \State{j}$ is reached almost surely after one step and the probability of ending up in any one of the target region parts $\IndexedStates{j}{J}$ is non-zero.
    The precise predecessor can be computed with

    \startformula
        \PrecisePredecessor{\StateRegion}{\ControlSpace'}{\IndexedSet{\State{j}}{j \in J}} =
        \StateRegion \cap \Big( \Big( \bigcap_{j \in J} (\State{j} \oplus - \RandomSpace) \setminus \bigcup_{j \in I \setminus J} ( \State{j} \oplus - \RandomSpace ) \Big) \oplus -\MatB \ControlSpace' \Big) \MatA \EndComma
    \stopformula

    where $I$ is the index set enumerating the partition of the extended state space.

\stopsection


\startsubsection[title={Attractors}]

    Analogous to the predecessor and robust predecessor operators, the attractor ($\Attr$) and robust attractor $\AttrR$ are defined as

    \startformula
        \startalign[n=2,align={right,left}]
            \NC \Attractor{\StateRegion}{\ControlSpace'}{\StateTarget} :=
            \NC \Set{ \VecState \in \StateRegion \mid \forall \VecControl' \in \ControlSpace' : \Posterior{\VecState}{\VecControl'} \cap \StateTarget \neq \emptyset } \EndAnd
            \NR
            \NC \RobustAttractor{\StateRegion}{\ControlSpace'}{\StateTarget} :=
            \NC \Set{ \VecState \in \StateRegion \mid \forall \VecControl' \in \ControlSpace' : \Posterior{\VecState}{\VecControl'} \subseteq \StateTarget } \EndComma
            \NR
        \stopalign
    \stopformula

    i.e. predecessor properties hold for all control inputs in $\ControlRegion$ for the attractors, not just some.
    Therefore,

    \startformula
        \startalign[n=2,align={right,left}]
            \NC \Attractor{\StateRegion}{\ControlSpace'}{\StateTarget} \subseteq
            \NC \Predecessor{\StateRegion}{\ControlSpace'}{\StateTarget} \EndComma
            \NR
            \NC \RobustAttractor{\StateRegion}{\ControlSpace'}{\StateTarget} \subseteq
            \NC \RobustPredecessor{\StateRegion}{\ControlSpace'}{\StateTarget} \EndAnd
            \NR
            \NC \RobustAttractor{\StateRegion}{\ControlSpace'}{\StateTarget} \subseteq
            \NC \Attractor{\StateRegion}{\ControlSpace'}{\StateTarget} \EndPeriod
            \NR
        \stopalign
    \stopformula

    The attractor operators can be computed from the predecessor operators as

    \startformula
        \startalign[n=2,align={right,left}]
            \NC \Attractor{\StateRegion}{\ControlSpace'}{\StateTarget} =
            \NC \StateRegion \setminus \RobustPredecessor{\StateRegion}{\ControlSpace'}{\ExtendedStateSpace \setminus \StateTarget} \EndComma
            \NR
            \NC \RobustAttractor{\StateRegion}{\ControlSpace'}{\StateTarget} =
            \NC \StateRegion \setminus \Predecessor{\StateRegion}{\ControlSpace'}{\ExtendedStateSpace \setminus \StateTarget} \EndPeriod
            \NR
        \stopalign
    \stopformula

    Although these operators are not used in the construction of the game graph, they play an important role in the refinement procedures presented later.

\stopsection


\startsubsection[title={Actions},reference=sec:abstraction-operators-actions]

    The last operators defined here are control space operators, i.e.\ the output of these operators is a subset of $\ControlSpace$, not $\StateSpace$.
    The action ($\Act$) and robust action ($\ActR$), given by

    \startformula
        \startalign[n=2,align={right,left}]
            \NC \Action{\StateRegion}{\StateTarget} =
            \NC \Set{ \VecControl \in \ControlSpace \mid \Posterior{\StateRegion}{\VecControl} \cap \StateTarget \neq \emptyset } \EndAnd
            \NR
            \NC \RobustAction{\StateRegion}{\StateTarget} =
            \NC \Set{ \VecControl \in \ControlSpace \mid \Posterior{\StateRegion}{\VecControl} \subseteq \StateTarget } \EndComma
            \NR
        \stopalign
    \stopformula

    return the sets of control inputs with which the probability of transitioning to the target region $\StateTarget$ from origin region $\StateRegion$ in one step is non-zero and 1, respectively.
    These are probabilistic variants of the operators $U^{X \rightarrow Y}$ (corresponding to $\Action{X}{Y}$) and $U^{X \Rightarrow Y}$ (corresponding to $\RobustAction{X}{Y}$) defined by \cite[Yordanov2009] for a deterministic setting.

    Both operators can be computed with the Minkowski sum and Pontryagin difference as

    \startformula
        \startalign[n=2,align={right,left}]
            \NC \Action{\State{i}}{\State{j}} =
            \NC \ControlSpace \cap ( \State{j} \oplus - (\MatA \State{i} \oplus \RandomSpace) ) \MatB \EndAnd
            \NR
            \NC \RobustAction{\State{i}}{\State{j}} =
            \NC \ControlSpace \cap ( \State{j} \ominus (\MatA \State{i} \oplus \RandomSpace) ) \MatB \EndPeriod
            \NR
        \stopalign
    \stopformula

    Because only full-dimensional polytopes are considered here (see \in{Section}[sec:theory-geometry]), control-space regions of lower dimension than $\ControlSpace$ are not computable by the action operators.
    With this constraint, it is possible that $\RobustAction{\VecState}{\StateTarget} = \emptyset$ for some $\VecState \in \RobustPredecessor{\StateRegion}{\ControlSpace}{\StateTarget}$, even though this should not be possible if lower-dimensional regions were allowed.
    This artifact of the computational constraints plays an important role in \in{Section}[sec:abstraction-analysis-correctness].

    Completing the set of action operators is the concrete action ($\ActC$).
    As for the precise predecessor, a state space partition is required for this operator to be meaningful:

    \startformula
        \startalign[n=2,align={right,left}]
            \NC \ConcreteAction{\StateRegion}{\IndexedSet{\State{j}}{j \in J}} := \Big\{ \VecControl \in \ControlSpace \Bigmid
            \NC \Posterior{\StateRegion}{\VecControl} \subseteq \bigcup_{j \in J} \State{j} \;\text{and}
            \NR
            \NC \empty
            \NC ~\forall j \in J : \Posterior{\StateRegion}{\VecControl} \cap \State{j} \neq \emptyset \Big\} \EndPeriod
            \NR
        \stopalign
    \stopformula

    Under the returned control inputs only $\StateTarget$ can be reached in one step from states in $\StateRegion$ (first condition) and for every element of the partitioned target region a state in $\StateRegion$ exists such that the probability of reaching this element is non-zero.
    It must be noted that the conditions of $\PreP$ and $\ActC$ are quite different from one another:
    The precise predecessor requires some control input for every of its states for which both conditions (subset and part intersection) are fulfilled.
    The concrete action requires that for every of its control vectors the subset condition is fulfilled for every origin state while the part intersections can be fulfilled independently by different states in $\StateRegion$ as long as every part is reachable from some state.
    Hence,

    \startformula
        \ConcreteAction{\StateRegion}{\IndexedStates{j}{J}} \,\cap\, \ConcreteAction{\StateRegion}{\IndexedStates{j}{J'}} = \emptyset
    \stopformula

    for all $J \neq J'$.
    The same does not hold for precise predecessors, which can overlap for different target sets.
    Concrete actions are computable directly from $\Act$ with

    \startformula
        \startalign[n=2,align={right,left}]
            \NC \ConcreteAction{\StateRegion}{\IndexedStates{j}{J}} =
            \NC \bigcap_{j \in J} \Action{\StateRegion}{\State{j}} \setminus \bigcup_{j \in I \setminus J} \Action{\StateRegion}{\State{j}} \EndComma
            \NR
        \stopalign
    \stopformula

    where $I$ is again the index set enumerating the elements of the partition of $\ExtendedStateSpace$.
    Note that \cite[Svorenova2017] use $U_i^J$ to denote the set $\ConcreteAction{\State{i}}{\IndexedStates{j}{J}}$, an abbreviation that is adopted here as well when space constraints demand it.

\stopsection


            \stopsection

            \startsection[title={Game Graph Construction},reference=sec:abstraction-graph]
                To analyse the continuous-state, continuous-action LSS, it must be discretized by replacing the infinite-member state and control spaces with finite counterparts.
In the discrete abstraction, every trace realizable in the LSS must also be realizable.
But additionally, the abstraction may allow additional behaviour.
If the temporal logic specification is satisfied by the abstraction, it will also be satisfied by the original system whose behaviour is a subset of that of the abstraction.

The abstraction model chosen by \cite[Svorenova2017] is a probabilistic game graph $\GameGraph$ with 2-players, governed by the dynamics operators from \in[sec:abstraction-operators].
Its construction is described in the following text.
In contrast to \cite[Svorenova2017], the game is build directly here, without introducing a non-deterministic transition system first and then extending to a 2½-player game, thereby reintroducing the stochasticity of the LSS.


\startsubsection[title={Player 1},reference=sec:abstraction-graph-playerone]

    The state space discretization requires a grouping of all state space vectors into a finite set of disjunct regions.
    As seen in section \in[sec:theory-geometry-properties], convex geometry has many advantageous computational properties, therefore a convex, polytopic partition of the extended state space $\ExtendedStateSpace$ is chosen.
    The inclusion of $\ExtendedStateSpace \setminus \StateSpace$ in addition to the state space will simplify the handling of transition out of the state space during construction.
    While iterating the abstraction-analysis-refinement cycle from Figure \in[fig:problem-approach-flowchart], this partition is subject to change.
    A sensible partition of $\StateSpace$ to initiate the procedure is based on the equivalence relation

    \placeformula[fml:abstraction-graph-decomposition]
    \startformula
        \VecState \sim_{\Predicates} \VecState' \;\Longleftrightarrow\; \PredicatesOf{\VecState} = \PredicatesOf{\VecState'} \EndComma
    \stopformula

    induced by the linear predicates $\Predicates$.
    It is polytopic, convex and allows a straightforward connection to the temporal logic specification later (see \in[sec:abstraction-product]).
    The region $\ExtendedStateSpace \setminus \StateSpace$ can be partitioned arbitrarily into convex polytopes.
    It only exists as a convenient transition target and does not require refinement in any subsequent iteration.

    The player 1 states

    \startformula
        G_1 = \IndexedStates{i}{I}
    \stopformula

    of the game abstraction $\GameGraph$ are based on this decomposition of $\ExtendedStateSpace$.
    The states are directly identified with the corresponding polytopes from the partition, which are enumerated with the index set $I$.
    In a geometric context $\State{i}$ will refer to the polytope, while in a game-theoretic context, $\State{i}$ will refer to the associated player 1 state.

    % TODO: example, introduce linear predicate and decomposition, show figure with extended state space and partition

    For any given state vector $\VecState$ of the state space, the partition induces an equivalence relation $\sim_\VecState$ over the control space

    \startformula
        \VecControl \sim_\VecState \VecControl' \;\Longleftrightarrow\;
        \forall j \in \StateIndices: \Big(
            ( \Posterior{\VecState}{\VecControl} \cap \State{j} = \emptyset ) \,\leftrightarrow\,
            ( \Posterior{\VecState}{\VecControl'} \cap \State{j} = \emptyset )
        \Big) \EndComma
    \stopformula

    i.e.\ two control vectors are equivalent if and only if the same set of state space partition elements is reachable after one step of system evolution when starting in $\VecState$.
    In order to obtain actions for a state $\State{i}$ based on this relation it must be extended to the entire polytope.
    However, not all states in $\State{i}$ necessarily produce the same relation, so a common $\sim_{\State{i}}$ does not generally exist.
    Instead, it is defined as

    \startformula
        \VecControl \sim_{\State{i}} \VecControl' \;\Longleftrightarrow\;
        \forall j \in \StateIndices: \Big(
            ( \Posterior{\State{i}}{\VecControl} \cap \State{j} = \emptyset ) \,\leftrightarrow\,
            ( \Posterior{\State{i}}{\VecControl'} \cap \State{j} = \emptyset )
        \Big)
    \stopformula

    and used to generate the player 1 actions

    \startformula
        Act_1 = \BigSet{ \PlayerOneAction{i}{J} \Bigmid i \in I \MidComma J \subseteq I } \EndComma
    \stopformula

    named after the index of the origin state $\State{i}$ and the indices of reachable target states $\IndexedStates{j}{J}$, governed by $\sim_{\State{i}}$.
    No actions are defined for the outer states 

    \startformula
        \Set{ \State{i} \mid i \in I \MidComma \State{i} \subseteq \ExtendedStateSpace \setminus \StateSpace } \EndPeriod
    \stopformula

    Note that every set of reachable target states has a unique associated game action and control space region.
    The region is given by the concrete action dynamics operator $\ActC$ which exactly reflects the action-generating relation $\sim_{\State {i}}$.

    % TODO: example: calculate actions for a state
    Non-empty concrete actions for $\State{4} = \ClosedInterval{0}{2}$:

    \startformula
        \startalign[n=3,align={left,left,left}]
            \NC \ConcreteAction{\State{4}}{\Set{\State{3},\State{4}}}
            \NC = \ClosedInterval{0.1}{1}
            \NC = U_4^{\Set{3, 4}} \EndComma
            \NR
            \NC \ConcreteAction{\State{4}}{\Set{\State{1},\State{3},\State{4}}}
            \NC = \ClosedInterval{-0.1}{0.1}
            \NC = U_4^{\Set{1, 3, 4}} \EndComma
            \NR
            \NC \ConcreteAction{\State{4}}{\Set{\State{1},\State{4}}}
            \NC = \ClosedInterval{-1}{-0.1}
            \NC = U_4^{\Set{1, 4}} \EndPeriod
            \NR
        \stopalign
    \stopformula

    Therefore, 3 player 1 actions are obtained for this state: $\PlayerOneAction{4}{\Set{1, 4}}$, $\PlayerOneAction{4}{\Set{1, 3, 4}}$ and $\PlayerOneAction{4}{\Set{3, 4}}$.

\stopsubsection


\startsubsection[title={Player 2},reference=sec:abstraction-graph-playertwo]

    In the original LSS, after a control input has been selected, the next state is determined stochastically according the the evolution equation and the probability distribution over $\RandomSpace$.
    In the abstraction, a probability distribution that works for all $\VecState \in \State{i}$ does not exist generally, so the exact probabilities of reaching another target state when player 1 has selected an action are unknown until the trace and control input are exactly localized.
    In particular, because of a possible mismatch between $\sim_\VecState$ and $\sim_{\State{i}}$, not every state in the action's target set may be reachable once a trace has been localized in a state $\VecState \in \State{i}$.
    A simple probabilistic transition after a player 1 action to the next player 1 state is therefore not possible.
    The trace has to be localized in $\State{i}$ first and the exact control input selected so that the transition probability distribution over the set of target states can be determined.
    Hence, a second player with the power to localize the trace and select a specific control vector from the region associated with the player 1 action is required in the abstraction.
    The game can then transition to any of the reachable states by sampling the resulting probability distribution.

    Since every target set of state space partition elements has exactly one associated action, player 2 states

    \startformula
        G_2 = \BigSet{\Tuple{\State{i}}{J} \Bigmid i \in I \MidAnd J \subseteq I} \EndComma
    \stopformula

    are simply tuples of player 1 states and actions.
    The transition relation

    \startformula
        \Transition_\GameGraph
            \Big( \State{i}, \PlayerOneAction{i'}{J'} \Big)
            \Big( \Tuple{\State{i}}{J} \Big)
        = \startmathcases
            \NC 1
            \MC \startgathered
                    \NC \StartIf i = i' \MidAnd J = J'
                    \NR
                    \NC \quad \MidAnd \ConcreteAction{\State{i}}{\IndexedStates{j}{J}} \neq \emptyset
                    \NR
                \stopgathered
            \NR
            \NC 0
            \NC otherwise
            \NR
        \stopmathcases
    \stopformula

    matches a player 1 state and action with the corresponding player 2 state deterministically.
    As noted before, the appropriate dynamics operator to express the relation $\sim_{\State {i}}$, which generates the player 1 actions, is $\ActC$.

    Player 2's actions are supposed to determine the probability distribution used for the actual system transition.
    A finite set of actions is desired but there are potentially infinitely many because the probability distributions will be different for every $\VecState \in \State{i}$ and $\VecControl \in \ConcreteAction{\State{i}}{\IndexedStates{j}{J}}$ that player 2 can choose from.
    Fortunately, in the context of almost-sure analysis, all non-zero probabilities are equivalent. % TODO add a reference and intuitive explanation
    This reduces the choice of the probability distribution to a choice of the support set of the probability distribution.
    So, (finitely many) player 2 actions

    \startformula
        Act_2 = \BigSet{ \PlayerTwoAction{i}{J}{K} \Bigmid i \in I \MidComma K \subseteq J \subseteq I } \EndComma
    \stopformula

    where $K$ is the set of target state indices from the support of the probability distribution, can be defined without losing relevant behaviour of the system for the purposes of almost-sure analysis.
    The equivalency of all probability distributions with the same support sets also means that uniform distributions can be conveniently chosen everywhere for the player 2 transition relation

    \startformula
        \Transition_\GameGraph
            \Big( \Tuple{\State{i}}{J}, \PlayerTwoAction{i'}{J'}{K} \Big)
            \Big( \State{k} \Big)
        = \startmathcases
            \NC \displaystyle\frac{1}{|K|}
            \MC \startgathered
                    \NC \StartIf i = i' \MidAnd J = J' \MidAnd k \in K
                    \NR
                    \NC \quad \MidAnd \PrecisePredecessor{\State{i}}{U_i^J}{\IndexedStates{k}{K}} \neq \emptyset
                    \NR
                \stopgathered
            \NR
            \NC 0
            \NC otherwise \EndComma
            \NR
        \stopmathcases
    \stopformula

    where $U_i^J = \ConcreteAction{\State{i}}{\IndexedStates{j}{J}}$ is the control input associated with the player 1 action $\PlayerOneAction{i}{J}$ that lead to the player 2 state from player 1 state $\State{i}$.
    The dynamics operator used is the precise predecessor, which corresponds to the origin regions in $\State{i}$ for which probability distribution support sets $\IndexedStates{k}{K}$ are identical for some $\VecControl \in U_i^J$.

    % TODO example
    So e.g. action $\PlayerOneAction{4}{\Set{1, 3, 4}}$ leads (determinisitically) to state $\Tuple{\State{4}}{\Set{1, 3, 4}}$, where one finds these non-empty precise predecessors:

    \startformula
        \startalign[n=2,align={left,left}]
            \NC \PrecisePredecessor{\State{4}}{U_4^{\Set{1, 3, 4}}}{\Set{\State{1},\State{4}}}
            \NC = \ClosedInterval{0}{0.2}
            \NR
            \NC \PrecisePredecessor{\State{4}}{U_4^{\Set{1, 3, 4}}}{\Set{\State{3},\State{4}}}
            \NC = \ClosedInterval{1.8}{2}
            \NR
            \NC \PrecisePredecessor{\State{4}}{U_4^{\Set{1, 3, 4}}}{\Set{\State{4}}}
            \NC = \ClosedInterval{0}{2}
            \NR
        \stopalign
    \stopformula

    Therefore, player 2 has actions $\PlayerTwoAction{4}{\Set{1, 3, 4}}{\Set{1, 4}}$, $\PlayerTwoAction{4}{\Set{1, 3, 4}}{\Set{3, 4}}$ and $\PlayerTwoAction{4}{\Set{1, 3, 4}}{\Set{4}}$ available.

\stopsubsection


\startreusableMPgraphic{example-play-paths}
    with spacing((45,60)) matrix.a(6,9);
    % Player 1 states (origin)
    with fixedboxwidth(40) with fixedboxheight(30) with shape(fixedbox) node.a[0][4](btex $\State{4}$ etex);
    % Player 2 states
    with fixedboxwidth(80) with fixedboxheight(30) with shape(fixedbox) with filling(solid) with fillingcolor(lightgray) node.a[2][1](btex $\Tuple{\State{4}}{\Set{1, 4}}$ etex);
    with fixedboxwidth(80) with fixedboxheight(30) with shape(fixedbox) with filling(solid) with fillingcolor(lightgray) node.a[2][4](btex $\Tuple{\State{4}}{\Set{1, 3, 4}}$ etex);
    with fixedboxwidth(80) with fixedboxheight(30) with shape(fixedbox) with filling(solid) with fillingcolor(lightgray) node.a[2][7](btex $\Tuple{\State{4}}{\Set{3, 4}}$ etex);
    % Player 1 states (target)
    with fixedboxwidth(40) with fixedboxheight(30) with shape(fixedbox) node.a[5][1](btex $\State{1}$ etex);
    with fixedboxwidth(40) with fixedboxheight(30) with shape(fixedbox) node.a[5][4](btex $\State{4}$ etex);
    with fixedboxwidth(40) with fixedboxheight(30) with shape(fixedbox) node.a[5][7](btex $\State{3}$ etex);
    % Player 1 actions
    with shape(circle) with size(35) node.a[1][1](btex \ssd $\Set{1, 4}$ etex);
    with shape(circle) with size(35) node.a[1][4](btex \ssd $\Set{1, 3, 4}$ etex);
    with shape(circle) with size(35) node.a[1][7](btex \ssd $\Set{3, 4}$ etex);
    % Player 2 actions
    with shape(circle) with filling(solid) with fillingcolor(lightgray) with size(35) node.a[3][0](btex \ssd $\Set{1}$ etex);
    with shape(circle) with filling(solid) with fillingcolor(lightgray) with size(35) node.a[3][1](btex \ssd $\Set{1, 4}$ etex);
    with shape(circle) with filling(solid) with fillingcolor(lightgray) with size(35) node.a[3][2](btex \ssd $\Set{4}$ etex);
    with shape(circle) with filling(solid) with fillingcolor(lightgray) with size(35) node.a[3][3](btex \ssd $\Set{1, 4}$ etex);
    with shape(circle) with filling(solid) with fillingcolor(lightgray) with size(35) node.a[3][4](btex \ssd $\Set{4}$ etex);
    with shape(circle) with filling(solid) with fillingcolor(lightgray) with size(35) node.a[3][5](btex \ssd $\Set{3, 4}$ etex);
    with shape(circle) with filling(solid) with fillingcolor(lightgray) with size(35) node.a[3][6](btex \ssd $\Set{4}$ etex);
    with shape(circle) with filling(solid) with fillingcolor(lightgray) with size(35) node.a[3][7](btex \ssd $\Set{3, 4}$ etex);
    with shape(circle) with filling(solid) with fillingcolor(lightgray) with size(35) node.a[3][8](btex \ssd $\Set{3}$ etex);
    % Arrows (player 1)
    with tipsize(0) arrow.top(.9, "") (a[0][4],a[1][1]) a[0][4].c..a[1][1].c;
    with tipsize(0) arrow.rt(.9, "") (a[0][4],a[1][4]) a[0][4].c..a[1][4].c;
    with tipsize(0) arrow.top(.9, "") (a[0][4],a[1][7]) a[0][4].c..a[1][7].c;
    arrow.rt(.5, "1") (a[1][1],a[2][1]) a[1][1].c..a[2][1].c;
    arrow.rt(.5, "1") (a[1][4],a[2][4]) a[1][4].c..a[2][4].c;
    arrow.rt(.5, "1") (a[1][7],a[2][7]) a[1][7].c..a[2][7].c;
    % Arrows (player 2)
    with tipsize(0) arrow.top(.9, "") (a[2][1],a[3][0]) a[2][1].c..a[3][0].c;
    with tipsize(0) arrow.top(.9, "") (a[2][1],a[3][1]) a[2][1].c..a[3][1].c;
    with tipsize(0) arrow.top(.9, "") (a[2][1],a[3][2]) a[2][1].c..a[3][2].c;
    with tipsize(0) arrow.top(.9, "") (a[2][4],a[3][3]) a[2][4].c..a[3][3].c;
    with tipsize(0) arrow.top(.9, "") (a[2][4],a[3][4]) a[2][4].c..a[3][4].c;
    with tipsize(0) arrow.top(.9, "") (a[2][4],a[3][5]) a[2][4].c..a[3][5].c;
    with tipsize(0) arrow.top(.9, "") (a[2][7],a[3][6]) a[2][7].c..a[3][6].c;
    with tipsize(0) arrow.top(.9, "") (a[2][7],a[3][7]) a[2][7].c..a[3][7].c;
    with tipsize(0) arrow.top(.9, "") (a[2][7],a[3][8]) a[2][7].c..a[3][8].c;
    arrow.rt(.6, btex $1$ etex) (a[3][0],a[5][1]) a[3][0].c..a[5][1].c;
    arrow.rt(.6, btex $\frac{1}{2}$ etex) (a[3][1],a[5][1]) a[3][1].c..a[5][1].c;
    arrow.rt(.6, btex $\frac{1}{2}$ etex) (a[3][1],a[5][4]) a[3][1].c..a[5][4].c;
    arrow.rt(.6, btex $1$ etex) (a[3][2],a[5][4]) a[3][2].c..a[5][4].c;
    arrow.rt(.6, btex $\frac{1}{2}$ etex) (a[3][3],a[5][1]) a[3][3].c..a[5][1].c;
    arrow.rt(.6, btex $\frac{1}{2}$ etex) (a[3][3],a[5][4]) a[3][3].c..a[5][4].c;
    arrow.rt(.6, btex $1$ etex) (a[3][4],a[5][4]) a[3][4].c..a[5][4].c;
    arrow.rt(.6, btex $\frac{1}{2}$ etex) (a[3][5],a[5][7]) a[3][5].c..a[5][7].c;
    arrow.rt(.6, btex $\frac{1}{2}$ etex) (a[3][5],a[5][4]) a[3][5].c..a[5][4].c;
    arrow.rt(.6, btex $1$ etex) (a[3][6],a[5][4]) a[3][6].c..a[5][4].c;
    arrow.rt(.6, btex $\frac{1}{2}$ etex) (a[3][7],a[5][7]) a[3][7].c..a[5][7].c;
    arrow.rt(.6, btex $\frac{1}{2}$ etex) (a[3][7],a[5][4]) a[3][7].c..a[5][4].c;
    arrow.rt(.6, btex $1$ etex) (a[3][8],a[5][7]) a[3][8].c..a[5][7].c;
\stopreusableMPgraphic

\startsubsection[title={Synopsis},reference=sec:abstraction-graph-synopsis]

    In total, the constructed 2½-player game graph is

    \startformula
        \GameGraph = (G_1, G_2, Act, \Transition_\GameGraph) \EndComma
    \stopformula

    where $Act = Act_1 \cup Act_2$.

    To summarize the process of a play starting in a player 1 state:
    First, player 1 selects a set of elements of the state space partition reachable from the current state.
    One of these elements should be randomly selected as a successor but because the real state of the trace that the play corresponds to is unknown in the discretized state space, the distribution from which the successor is sampled is not unique and must be chosen by player 2.
    Conveniently, only the supports of the probability distributions matter for almost-sure analysis, so player 2 can choose from a (finite) set of uniform distributions with supports from the power set of the reachable states selected by player 1.

    Translated to the corresponding trace in the LSS, this means player 1 chooses a region of the control space given by one of the non-empty concrete actions of the state space partition element the trace is currently located in.
    Player 2 then chooses a control vector from this region and reveals the exact location of the trace in the state space.
    The trace steps forward in time according to the evolution equation and it is player 1's turn again.

    % TODO: example game graph construction
    Complete graph of actions and states for $\State{4}$.

    \placefigure[top][fig:abstraction-graph-xfour]{
        Possible paths of a trace through game graph abstraction of the example system (\in[fml:abstraction-example]) from initial state $\State{4}$.
        Rectangles: states.
        Circles: actions.
        White: player 1.
        Grey: player 2.
        Transition probabilities labeled.
        The two depicted states $\State{4}$ are the same, shown separately for improved readability.
    }{
        \framed[width=\textwidth,frame=off]{\reuseMPgraphic{example-play-paths}}
    }

\stopsubsection


            \stopsection

            \startsection[title={Product Game Construction},reference=sec:abstraction-product]
                The game graph constructed in previous section has no winning condition, but in order to analyse the game with respect to a temporal logic specification one is required.
Translating the specifiction into an \omega-automaton yields an acceptance condition for the automaton which can be transfered to the game through construction of their synchronous product.
The resulting product game enforces the synchronized evaluation of the objective automaton during plays on the game graph and can be given to a game solver for analysis purposes.


\startsubsection[title={Objective Automaton}]

    The GR(1) objective formula $\varphi$ from section \in[sec:problem-statement] is formulated over a set of linear predicates $\Predicates$.
    The initial decomposition of the state space (\in[fml:abstraction-graph-decomposition]) uses an equivalence relation based on these linear predicates, so every state of a given $\State{i}$ fulfills and rejects exactly the same set of linear predicates, $\PredicatesOf{\State{i}} \in 2^\Predicates$.
    A play of $\GameGraph$ therefore induces a word over the alphabet $2^\Predicates$.
    Using this alphabet, an \omega-automaton

    \startformula
        \Automaton = (Q,\, 2^\Predicates,\, \Transition_\Automaton,\, q_0,\, \Condition_\Automaton)
    \stopformula

    corresponding to $\varphi$ can be constructed.
    The restriction to GR(1) formulas allows expression of the acceptance condition with a Streett pair $\Transition_\Automaton = \Tuple{E_\Automaton}{F_\Automaton}$ (see section \in[sec:theory-automata]).

\stopsubsection


\startsubsection[title={Synchronized Product}]

    To enforce the evaluation of the automaton synchronized to plays from the game, their synchronous product

    \startformula
        \ProductGame = ( P_1, P_2, Act, \Transition, \Condition )
    \stopformula

    is constructed.
    $\ProductGame$ is a 2½-player game with a one-pair Streett winning condition $\Condition$ modeled after $\Condition_\Automaton$.
    Player states are $P_1 = G_1 \ftimes Q$ and $P_2 = G_2 \ftimes Q$, where $\ftimes$ is the normal cartesian product but with flattened result to reduce visual clutter.
    The set of action is taken directly from the game graph $\GameGraph$.
    The synchronous evolution is upheld through the transition relation

    \startformula
        \Transition
            \Big( \Tuple{\State{i}}{q}, \PlayerOneAction{i}{J} \Big)
            \Big( \Triple{\State{i}}{J}{q'} \Big)
        = \startmathcases
            \NC \Transition_\GameGraph
            \Big( \State{i}, \PlayerOneAction{i}{J} \Big)
            \Big( \Tuple{\State{i}}{J} \Big)
            \MC \StartIf \Transition_\Automaton(q, \PredicatesOf{\State{i}}) = q'
            \NR
            \NC 0
            \NC otherwise
            \NR
        \stopmathcases
    \stopformula

    for player 1 and

    \startformula
        \Transition
            \Big( \Triple{\State{i}}{J}{q}, \PlayerTwoAction{i}{J}{K} \Big)
            \Big( \Tuple{\State{k}}{q'} \Big)
        = \startmathcases
            \NC \Transition_\GameGraph
            \Big( \Tuple{\State{i}}{J}, \PlayerTwoAction{i}{J}{K} \Big)
            \Big( \State{k} \Big)
            \MC \StartIf q = q'
            \NR
            \NC 0
            \NC otherwise
            \NR
        \stopmathcases
    \stopformula

    for player 2.
    The Streett pair is given by $\Condition = \Tuple{E_\Automaton}{F_\Automaton}$ with $E_\Automaton = (G_1 \cup G_2) \ftimes E_\Automaton$ and $F_\Automaton = (G_1 \cup G_2) \ftimes F_\Automaton$.
    Note that automaton transitions occur only after player 1 actions.

\stopsubsection


\startreusableMPgraphic{mp:abstraction-product-endloop}
    with spacing((30,30)) matrix.a(9,10);
    movepos.a[2][0](-25,0);
    % States
    with fixedboxwidth(50) with fixedboxheight(30) with shape(fixedbox) node.a[2][0](btex $\Tuple{\State{i}}{q}$ etex);
    with fixedboxwidth(50) with fixedboxheight(30) with shape(fixedbox) node.a[2][9](btex $\State{\DeadEnd1}$ etex);
    with fixedboxwidth(50) with fixedboxheight(30) with shape(fixedbox) with filling(solid) with fillingcolor(lightgray) node.a[2][5](btex $\State{\DeadEnd2}$ etex);
    % Actions
    with shape(circle) with size(35) node.a[4][7](btex \ssd $\DeadEnd$ etex);
    with shape(circle) with size(35) node.a[2][2](btex \ssd $\DeadEnd$ etex);
    with shape(circle) with filling(solid) with fillingcolor(lightgray) with size(35) node.a[0][7](btex \ssd $\DeadEnd$ etex);
    % Arrows
    with tipsize(0) arrow.rt(.9, "") (a[2][9],a[4][7]) a[2][9].c..a[3][9].c..a[4][8].c..a[4][7].c;
    arrow.llft(.5, "1") (a[4][7],a[2][5]) a[4][7].c..a[4][6].c..a[3][5].c..a[2][5].c;
    with tipsize(0) arrow.rt(.9, "") (a[2][5],a[0][7]) a[2][5].c..a[1][5].c..a[0][6].c..a[0][7].c;
    arrow.urt(.5, "1") (a[0][7],a[2][9]) a[0][7].c..a[0][8].c..a[1][9].c..a[2][9].c;
    with tipsize(0) arrow.rt(.9, "") (a[2][0],a[2][2]) a[2][0].c..a[2][2].c;
    arrow.top(.5, "1") (a[2][2],a[2][5]) a[2][2].c..a[2][5].c;
\stopreusableMPgraphic

\startsubsection[title={Dead-end States}]

    \placefigure[top][fig:abstraction-product-endloop]{
        A player 1 state $\Tuple{\State{i}}{q}$ of the product game $\ProductGame$, connected to the dead-end loop.
    }{
        \framed[width=\textwidth,frame=off]{\reuseMPgraphic{mp:abstraction-product-endloop}}
    }

    If $\Automaton$ is an incomplete \omega-automaton, $\ProductGame$ will contain dead-end states (i.e.\ states without any outgoing actions) in $P_1$.
    Because a word is rejected immediately by an incomplete automaton when a non-existing transition is found, these dead-end states can easily be removed from the product game:
    A player 1 state $\State{\DeadEnd1}$, a player 2 state $\State{\DeadEnd2}$ and an action $\DeadEnd$ are added to $P_1$, $P_2$ and $Act$, respectively.
    States $\State{\DeadEnd1}, \State{\DeadEnd2}$ form a deterministic loop that traps any entering plays:

    \startformula
        \startalign[n=2,align={right,left}]
            \NC \Transition(\State{\DeadEnd1}, \DeadEnd)(s) =
            \NC \startmathcases
                    \NC 1
                    \MC \StartIf s = \State{\DeadEnd2}
                    \NR
                    \NC 0
                    \NC otherwise
                    \NR
                \stopmathcases
            \NR
            \NC \Transition(\State{\DeadEnd2}, \DeadEnd)(s) =
            \NC \startmathcases
                    \NC 1
                    \MC \StartIf s = \State{\DeadEnd1}
                    \NR
                    \NC 0
                    \NC otherwise
                    \NR
                \stopmathcases
            \NR
        \stopalign
    \stopformula

    Any dead-end state $\State{i}$ is then deterministically connected to this loop using the action $\DeadEnd$ as depicted in Figure \in[fig:abstraction-product-endloop].
    To ensure that a play stuck in the loop will never lead to player 1 winning the game, states $\State{\DeadEnd1}$ and $\State{\DeadEnd2}$ are added to the set $E_\Automaton$ of the one-pair Streett acceptance condition $\Condition$.

    Dead-end states will also occur due to the outer states from $\ExtendedStateSpace \setminus \StateSpace$ which were explicitly constructed without actions in the game graph $\GameGraph$.
    These are connected to the same dead-end loop used for handling automaton dead ends, as traces that leave the state space immediately violate the specification.

\stopsubsection


\startsubsection[title={Co-safe Interpretation}]

    There is an exception to the last statement of the previous section:
    In the co-safe interpretation, traces that have fulfilled their objective are free to go anywhere, even outside of the state space.
    It is possible to accomodate this interpretation in the product game construction with a postprocessing step analogous to the one that resolves dead-end states.
    A new player 1 state $\State{\SatEnd1}$ and player 2 state $\State{\SatEnd2}$ is introduced together with an action $\SatEnd$.
    Final states, i.e.\ states triggering satisfaction of the co-safe objective, are stripped of their actions and then deterministically redirected into a loop constructed exactly as in Figure \in[fig:abstraction-product-endloop] but using $\State{\SatEnd1}$ instead of $\State{\DeadEnd1}$, $\State{\SatEnd2}$ instead of $\State{\DeadEnd2}$ and $\SatEnd$ instead of $\DeadEnd$.
    The only states occuring infinitely often in a trace trapped in this loop are $\State{\SatEnd1}$ and $\State{\SatEnd2}$ which are not included in either of the acceptance sets of the one-pair Streett condition.
    Trapped traces will therefore guarantee that player 1 wins the game no matter what happens after the objective has been satisfied.

    The following requirements are expected for one-pair Streett acceptance conditions $\Tuple{E}{F}$ with co-safe interpretation:
    $F$ contains all final states of the automaton or game and must not be empty.
    All other states are members of $E$.
    Using this convention, automata and games with the these properties can be interpreted in both the infinite framework, where traces must never leave the state space, and the co-safe framework, where this requirement is lifted once a final state has been reached.

\stopsubsection


            \stopsection

            \startsection[title={Example},reference=sec:abstraction-example]
                Example: 1D variant of illustrative example and recurrence

\startformula
    x_{t+1} = x + u + w
\stopformula

Game graph:
Non-empty concrete actions for $\State{4} = \ClosedInterval{0}{2}$:

\startformula
    \startalign[n=3,align={left,left,left}]
        \NC \ConcreteAction{\State{4}}{\Set{\State{3},\State{4}}}
        \NC = \ClosedInterval{0.1}{1}
        \NC = U_4^\Set{3, 4} \EndComma
        \NR
        \NC \ConcreteAction{\State{4}}{\Set{\State{1},\State{3},\State{4}}}
        \NC = \ClosedInterval{-0.1}{0.1}
        \NC = U_4^\Set{1, 3, 4} \EndComma
        \NR
        \NC \ConcreteAction{\State{4}}{\Set{\State{1},\State{4}}}
        \NC = \ClosedInterval{-1}{-0.1}
        \NC = U_4^\Set{1, 4} \EndPeriod
        \NR
    \stopalign
\stopformula

Therefore, we obtain 3 player 1 actions for this state: $\PlayerOneAction{4}{\Set{1, 4}}$, $\PlayerOneAction{4}{\Set{1, 3, 4}}$ and $\PlayerOneAction{4}{\Set{3, 4}}$.
So e.g. action $\PlayerOneAction{4}{\Set{1, 3, 4}}$ leads (determinisitically) to state $\Tuple{\State{4}}{\Set{1, 3, 4}}$, where one finds these non-empty precise predecessors:

\startformula
    \startalign[n=2,align={left,left}]
        \NC \PrecisePredecessor{\State{4}}{U_4^\Set{1, 3, 4}}{\Set{\State{1},\State{4}}}
        \NC = \ClosedInterval{0}{0.2}
        \NR
        \NC \PrecisePredecessor{\State{4}}{U_4^\Set{1, 3, 4}}{\Set{\State{3},\State{4}}}
        \NC = \ClosedInterval{1.8}{2}
        \NR
        \NC \PrecisePredecessor{\State{4}}{U_4^\Set{1, 3, 4}}{\Set{\State{4}}}
        \NC = \ClosedInterval{0}{2}
        \NR
    \stopalign
\stopformula

From that one finds that player 2 has actions $\PlayerTwoAction{4}{\Set{1, 3, 4}}{\Set{1, 4}}$, $\PlayerTwoAction{4}{\Set{1, 3, 4}}{\Set{3, 4}}$ and $\PlayerTwoAction{4}{\Set{1, 3, 4}}{\Set{4}}$ available.

Product game:
Assume in state $\Tuple{\State{4}}{q}$, automaton transition to $q'$.
Then a play of one player 1 turn and one player 2 turn can take these paths:

\startreusableMPgraphic{example-play-paths}
    with spacing((45,60)) matrix.a(6,9);
    % Player 1 states (origin)
    with fixedboxwidth(50) with fixedboxheight(30) with shape(fixedbox) node.a[0][4](btex $\Tuple{\State{4}}{q}$ etex);
    % Player 2 states
    with fixedboxwidth(90) with fixedboxheight(30) with shape(fixedbox) with filling(solid) with fillingcolor(lightgray) node.a[2][1](btex $\Triple{\State{4}}{\Set{1, 4}}{q'}$ etex);
    with fixedboxwidth(90) with fixedboxheight(30) with shape(fixedbox) with filling(solid) with fillingcolor(lightgray) node.a[2][4](btex $\Triple{\State{4}}{\Set{1, 3, 4}}{q'}$ etex);
    with fixedboxwidth(90) with fixedboxheight(30) with shape(fixedbox) with filling(solid) with fillingcolor(lightgray) node.a[2][7](btex $\Triple{\State{4}}{\Set{3, 4}}{q'}$ etex);
    % Player 1 states (target)
    with fixedboxwidth(50) with fixedboxheight(30) with shape(fixedbox) node.a[5][1](btex $\Tuple{\State{1}}{q'}$ etex);
    with fixedboxwidth(50) with fixedboxheight(30) with shape(fixedbox) node.a[5][4](btex $\Tuple{\State{4}}{q'}$ etex);
    with fixedboxwidth(50) with fixedboxheight(30) with shape(fixedbox) node.a[5][7](btex $\Tuple{\State{3}}{q'}$ etex);
    % Player 1 actions
    with shape(circle) with size(35) node.a[1][1](btex \ssd $\Set{1, 4}$ etex);
    with shape(circle) with size(35) node.a[1][4](btex \ssd $\Set{1, 3, 4}$ etex);
    with shape(circle) with size(35) node.a[1][7](btex \ssd $\Set{3, 4}$ etex);
    % Player 2 actions
    with shape(circle) with filling(solid) with fillingcolor(lightgray) with size(35) node.a[3][0](btex \ssd $\Set{1}$ etex);
    with shape(circle) with filling(solid) with fillingcolor(lightgray) with size(35) node.a[3][1](btex \ssd $\Set{1, 4}$ etex);
    with shape(circle) with filling(solid) with fillingcolor(lightgray) with size(35) node.a[3][2](btex \ssd $\Set{4}$ etex);
    with shape(circle) with filling(solid) with fillingcolor(lightgray) with size(35) node.a[3][3](btex \ssd $\Set{1, 4}$ etex);
    with shape(circle) with filling(solid) with fillingcolor(lightgray) with size(35) node.a[3][4](btex \ssd $\Set{4}$ etex);
    with shape(circle) with filling(solid) with fillingcolor(lightgray) with size(35) node.a[3][5](btex \ssd $\Set{3, 4}$ etex);
    with shape(circle) with filling(solid) with fillingcolor(lightgray) with size(35) node.a[3][6](btex \ssd $\Set{4}$ etex);
    with shape(circle) with filling(solid) with fillingcolor(lightgray) with size(35) node.a[3][7](btex \ssd $\Set{3, 4}$ etex);
    with shape(circle) with filling(solid) with fillingcolor(lightgray) with size(35) node.a[3][8](btex \ssd $\Set{3}$ etex);
    % Arrows (player 1)
    with tipsize(0) arrow.top(.9, "") (a[0][4],a[1][1]) a[0][4].c..a[1][1].c;
    with tipsize(0) arrow.rt(.9, "") (a[0][4],a[1][4]) a[0][4].c..a[1][4].c;
    with tipsize(0) arrow.top(.9, "") (a[0][4],a[1][7]) a[0][4].c..a[1][7].c;
    arrow.rt(.5, "1") (a[1][1],a[2][1]) a[1][1].c..a[2][1].c;
    arrow.rt(.5, "1") (a[1][4],a[2][4]) a[1][4].c..a[2][4].c;
    arrow.rt(.5, "1") (a[1][7],a[2][7]) a[1][7].c..a[2][7].c;
    % Arrows (player 2)
    with tipsize(0) arrow.top(.9, "") (a[2][1],a[3][0]) a[2][1].c..a[3][0].c;
    with tipsize(0) arrow.top(.9, "") (a[2][1],a[3][1]) a[2][1].c..a[3][1].c;
    with tipsize(0) arrow.top(.9, "") (a[2][1],a[3][2]) a[2][1].c..a[3][2].c;
    with tipsize(0) arrow.top(.9, "") (a[2][4],a[3][3]) a[2][4].c..a[3][3].c;
    with tipsize(0) arrow.top(.9, "") (a[2][4],a[3][4]) a[2][4].c..a[3][4].c;
    with tipsize(0) arrow.top(.9, "") (a[2][4],a[3][5]) a[2][4].c..a[3][5].c;
    with tipsize(0) arrow.top(.9, "") (a[2][7],a[3][6]) a[2][7].c..a[3][6].c;
    with tipsize(0) arrow.top(.9, "") (a[2][7],a[3][7]) a[2][7].c..a[3][7].c;
    with tipsize(0) arrow.top(.9, "") (a[2][7],a[3][8]) a[2][7].c..a[3][8].c;
    arrow.rt(.6, btex $1$ etex) (a[3][0],a[5][1]) a[3][0].c..a[5][1].c;
    arrow.rt(.6, btex $\frac{1}{2}$ etex) (a[3][1],a[5][1]) a[3][1].c..a[5][1].c;
    arrow.rt(.6, btex $\frac{1}{2}$ etex) (a[3][1],a[5][4]) a[3][1].c..a[5][4].c;
    arrow.rt(.6, btex $1$ etex) (a[3][2],a[5][4]) a[3][2].c..a[5][4].c;
    arrow.rt(.6, btex $\frac{1}{2}$ etex) (a[3][3],a[5][1]) a[3][3].c..a[5][1].c;
    arrow.rt(.6, btex $\frac{1}{2}$ etex) (a[3][3],a[5][4]) a[3][3].c..a[5][4].c;
    arrow.rt(.6, btex $1$ etex) (a[3][4],a[5][4]) a[3][4].c..a[5][4].c;
    arrow.rt(.6, btex $\frac{1}{2}$ etex) (a[3][5],a[5][7]) a[3][5].c..a[5][7].c;
    arrow.rt(.6, btex $\frac{1}{2}$ etex) (a[3][5],a[5][4]) a[3][5].c..a[5][4].c;
    arrow.rt(.6, btex $1$ etex) (a[3][6],a[5][4]) a[3][6].c..a[5][4].c;
    arrow.rt(.6, btex $\frac{1}{2}$ etex) (a[3][7],a[5][7]) a[3][7].c..a[5][7].c;
    arrow.rt(.6, btex $\frac{1}{2}$ etex) (a[3][7],a[5][4]) a[3][7].c..a[5][4].c;
    arrow.rt(.6, btex $1$ etex) (a[3][8],a[5][7]) a[3][8].c..a[5][7].c;
\stopreusableMPgraphic

\placefigure[top][fig:example-play-paths]{
    TODO
}{
    \framed[width=\textwidth,frame=off]{\reuseMPgraphic{example-play-paths}}
}


            \stopsection

            \startsection[title={Analysis},reference=sec:abstraction-analysis]
                To decide desired properties, game needs to be analysed twice:
once with adversarial player 2 (to yield winning states where even in the "worst case", game can be won by player 1) and once with cooperative player 2 (to yield states where winning is possible, but not guaranteed).
From membership in these 2 solution sets, it can be decided if states are satisfying, non-satisfying or if the abstraction is too coarse to decide and further refinement is necessary.


\startbuffer[buf:abstraction-analysis-solution-algorithm]
    \startalgorithmic[numbering=no,margin=0em]
        \INPUT{$\Pre^{\mathrm set}_1$, $\Pre^{\mathrm set}_2$, $\Pre^{\mathrm set}_3$ of game $\ProductGame = ( P_1, P_2, Act, \Transition, \Tuple{E}{F} )$}
        \OUTPUT{$\AlmostSure^{\mathrm set}(\ProductGame)$} % TODO coop or adv
    \stopalgorithmic
    \startalgorithmic
        \STATE{ $P \leftarrow P_1 \cup P_2$ }
        \STATE{ $D \leftarrow P \setminus (E \cup F)$ }
        \STATE{ $X, {\bar X}, Y, {\bar Y}, Z, {\bar Z}$: Set}
        \STATE{ ${\bar X} \leftarrow P$ }
        \STATE{ ${\bar Y} \leftarrow P$ }
        \STATE{ ${\bar Z} \leftarrow \emptyset$ }
        \DO
            \STATE{ $X \leftarrow {\bar X}$ }
            \DO
                \STATE{ $Y \leftarrow {\bar Y}$ }
                \DO
                    \STATE{ $Z \leftarrow {\bar Z}$ }
                    \STATE{ ${\bar Z} \leftarrow (F \cap \Pre_1(X)) \cup (E \cap \Pre_2(X, Y)) \cup (D \cap \Pre_3(X, Y, Z))$ }
                \ENDDOWHILE{ $Z \ne {\bar Z}$ }
                \STATE{ ${\bar Y} \leftarrow Z$ }
                \STATE{ ${\bar Z} \leftarrow P$ }
            \ENDDOWHILE{ $Y \ne {\bar Y}$ }
            \STATE{ ${\bar X} \leftarrow Y$ }
            \STATE{ $Y \leftarrow \emptyset$ }
        \ENDDOWHILE{ $X \ne {\bar X}$ }
        \RETURN{$X$}
    \stopalgorithmic
\stopbuffer

\startsubsection[title={Product Game Solution}]

    Analysis in both adversarial and cooperative settings achievable with same algorithm.
    One-pair Streett acceptance can be transformed into parity-3, for a quadratic-complexity algorithm exists, but a cubic, fixed-point algorithm is much easier to implement and therefore used here.
    The correctness of the algorithm follows from \cite[TODO].

    Based on the successor operator

    \startformula
        \Successor{s}{a} = \Support{\Transition(s, a)} \EndComma
    \stopformula

    two conditions

    \startformula
        \startalign[n=2,align={right,left}]
            \NC C_1(X) =
            \NC \Set{ (s, a) \mid \Successor{s}{a} \subseteq X } \EndAnd
            \NR
            \NC C_2(X, Y) =
            \NC \Set{ (s, a) \mid \Successor{s}{a} \subseteq X \MidAnd \Successor{s}{a} \cap Y \ne \emptyset }
            \NR
        \stopalign
    \stopformula

    for state-action pairs of the product game graph are defined.  
    Using these conditions, two sets of predecessor operators are defined, one for the adversarial setting

    \startformula
        \startalign[n=2,align={right,left}]
            \NC \Pre_1^{\Adversarial}(X) =
            \NC \Set{ s \in P_1 \mid \exists a \in Act : (s, a) \in C_1(X) }
            \NR
            \NC \empty
            \NC \quad \cup \Set{ s \in P_2 \mid \forall a \in \Act : (s, a) \in C_1(X) } \EndComma
            \NR
            \NC \Pre_2^{\Adversarial}(X, Y) =
            \NC \Set{ s \in P_1 \mid \exists a \in Act : (s, a) \in C_2(X, Y) }
            \NR
            \NC \empty
            \NC \quad \cup \Set{ s \in P_2 \mid \forall a \in \Act : (s, a) \in C_2(X, Y) } \EndComma
            \NR
            \NC \Pre_3^{\Adversarial}(X, Y, Z) =
            \NC \Set{ s \in P_1 \mid \exists a \in Act : (s, a) \in C_2(X, Y) \cup C_1(Z) }
            \NR
            \NC \empty
            \NC \quad \cup \Set{ s \in P_2 \mid \forall a \in Act : (s, a) \in C_2(X, Y) \cup C_1(Z) }
            \NR
        \stopalign
    \stopformula

    and for the cooperative setting

    \startformula
        \startalign[n=2,align={right,left}]
            \NC \Pre_1^{\Cooperative}(X) =
            \NC \Set{ s \in P_1 \cup P_2 \mid \exists a \in Act : (s, a) \in C_1(X) } \EndComma
            \NR
            \NC \Pre_2^{\Cooperative}(X, Y) =
            \NC \Set{ s \in P_1 \cup P_2 \mid \exists a \in Act : (s, a) \in C_2(X, Y) } \EndComma
            \NR
            \NC \Pre_3^{\Cooperative}(X, Y, Z) =
            \NC \Set{ s \in P_1 \cup P_2 \mid \exists a \in Act : (s, a) \in C_2(X, Y) \cup C_1(Z) } \EndPeriod
            \NR
        \stopalign
    \stopformula

    Substituting either set of these operators into Algorithm \in[alg:abstraction-analysis-solver] for $\Pre_1$, $\Pre_2$ and $\Pre_3$ results in a procedure that ... % TODO

    \placealgorithm[top][alg:abstraction-analysis-solver]{
        Solver for parity-3 games using fixed-point iteration \cite[authoryears][Svorenova2017].
        Adversarial player 2 for ${\mathrm set} = \Adversarial$ or cooperative player 2 for ${\mathrm set} = \Cooperative$.
    }{
        % https://mailman.ntg.nl/pipermail/ntg-context/2016/087390.html
        \startframedtext[width=\textwidth,frame=off]
            \getbuffer[buf:abstraction-analysis-solution-algorithm]
        \stopframedtext
    }

    Analogous to the notation of player 1 states, the sets $\YesStates{q}$, $\NoStates{q}$ and $\MaybeStates{q}$ are overloaded for all $q \in Q$ so they refer to the region covered by the polytopes of the member player 1 states if used in a geometric context.

\stopsubsection


\startsubsection[title={Correctness and Termination}]

    Results due to \cite[Svorenova2017].
    Partial solution in every iteration: sequence of yes and no sets fulfils subset relation.
    Soundness: $\YesStates{q_0}$ is always subset of $\InitialStates$ and $\NoStates{q_0}$ is always subset of $\StateSpace \setminus \InitialStates$.
    Partial correctness: If the algorithm terminates (set of maybe-states is empty), then solution is found by last iteration.

    Two examples for non-termination given.
    First, consider LSS

    \startformula
        \VecState_{t+1} = \VecState_t + \VecControl_t + \VecRandom_t
    \stopformula

    where $\VecState_t \in \ClosedInterval{0}{3}$, $\VecControl_t \in \ClosedInterval{-1.5}{1.5}$ and $\VecRandom_t \in \ClosedInterval{-0.5}{0.5}$ for all times $t$.
    A reachability objective with target region $\ClosedInterval{1}{2}$ can be fulfilled by every $\VecState \in \ClosedInterval{0}{3}$ with control input $\VecControl = 1.5 - \VecState$.
    However, this control input is the only control input that leads to satisfaction almost-surely and it is different for every state of the state space.
    Therefore, a solution cannot be found as a finite partition of the state space.

    But even if finite state space partition does exist, the algorithm may still not terminate in practice.
    To see this, consider the LSS

    \startformula
        \VecState_{t+1} = 1.5 \VecState_t + \VecControl_t + \VecRandom_t
    \stopformula

    where $\VecState_t \in \ClosedInterval{-2}{2}$, $\VecControl_t \in \ClosedInterval{-2}{2}$ and $\VecRandom_t \in \ClosedInterval{-1}{1}$ for all times $t$.
    Objective: general safety, don't leave the state space at any time.
    Propose a partitioning into polytopes $\State{1} = \ClosedInterval{-2}{-1}$, $\State{2} = \ClosedInterval{-1}{0}$, $\State{3} = \ClosedInterval{0}{1}$ and $\State{4} = \ClosedInterval{1}{2}$.
    Interval arithmetic shows that choosing control inputs $2$ for $\State{1}$, $1$ for $\State{2}$, $-1$ for $\State{3}$ and $-2$ for $\State{4}$ keeps traces starting from any state in the state space safe:

    \startformula
        \startalign[n=4,align={right,right,left,left}]
            \NC \State{1}:
            \NC 1.5 \cdot \ClosedInterval{-2}{1} + 2 + \ClosedInterval{-1}{1}
            \NC = \ClosedInterval{-2}{1.5}
            \NC \subseteq \StateSpace
            \NR
            \NC \State{2}:
            \NC 1.5 \cdot \ClosedInterval{-1}{0} + 1 + \ClosedInterval{-1}{1}
            \NC = \ClosedInterval{-1.5}{2}
            \NC \subseteq \StateSpace
            \NR
            \NC \State{3}:
            \NC 1.5 \cdot \ClosedInterval{0}{1} - 1 + \ClosedInterval{-1}{1}
            \NC = \ClosedInterval{-2}{1.5}
            \NC \subseteq \StateSpace
            \NR
            \NC \State{4}:
            \NC 1.5 \cdot \ClosedInterval{1}{2} - 2 + \ClosedInterval{-1}{1}
            \NC = \ClosedInterval{-1.5}{2}
            \NC \subseteq \StateSpace \EndPeriod
            \NR
        \stopalign
    \stopformula

    However, it was assumed in section \in[sec:TODO], that any non-empty polytope must be full-dimensional.
    The concrete action operator from which the player 1 actions in the abstraction are derived returns a polytopic regions which means it cannot work with single-member sets.
    For a small $\delta \ge 0$, consider a trace in state $\VecState_t = -2 + \frac{2}{3} \delta$.
    Setting up the condition for safe continuation of the trace, one obtains

    \startformula
        \startalign[n=3,align={middle,right,left}]
            \NC \empty
            \NC 1.5 \Big({-2} + \frac{2}{3} \delta\Big) + \VecControl_t + \ClosedInterval{-1}{1}
            \NC \subseteq \ClosedInterval{-2}{2}
            \NR
            \NC \Leftrightarrow \quad
            \NC \ClosedInterval{-4 + \delta + \VecControl_t}{-2 + \delta + \VecControl_t}
            \NC \subseteq \ClosedInterval{-2}{2}
            \NR
            \NC \Rightarrow \quad
            \NC \VecControl_t
            \NC \ge 2 - \delta
            \NR
        \stopalign
    \stopformula

    from the lower interval bound.
    Because any control region selectable by player 1 in the abstraction must contain an interval of diameter $\epsilon > 0$ or else it is considered to be empty, the system can never be fully decided in the abstraction.
    Any $\VecState = -2 + \frac{2}{3} \delta$ requires a control input region with a diameter of $\epsilon > \delta$, meaning all any state $\VecState \lt{} -2 + \frac{2}{3} \epsilon$ cannot be decided in the restrictions of the abstraction.

\stopsubsection


\startsubsection[title={Product Game Simplification}]

    Reminder of how to include co-safe switch into game creation.

    Now: If partial results are already available stop construction once decided state is reached.
    Possible because of guarantee that any state that can be decided by the analysis is correctly analysed (partial solution) and will never change, even if other parts of the state space partition are refined.
    Because game construction and analysis is expensive, use of partial solutions to simplify the game can reduce run time of the procedure substantially.

    Discuss problems with product game simplification and the available compromise.
    Because already decided states are immediately redirected into the $\SatEnd$ loop, some of their successors are constructed in the product game.
    While results from decided states can be transfered to such states from the previous analysis, for previously undecided states that are not constructed anymore it cannot generally be decided if they should remain undecided or are actually unreachable now.
    A subset of the full product game is omitted when the simplification is applied, which does not affect the validity of the analysis especially because the initial states, which are of most interest, are always constructed.
    However, as seen in the next section this might restrict the controller synthesis process, which relies on analysis results of all states, not just the initial ones.

\stopsubsection


            \stopsection

            \startsection[title={Controller Synthesis},reference=sec:abstraction-synthesis]
                For every trace starting in a state recognized as a member of $\InitialStates$, a strategy exists that leads to player 1 winning almost-surely.
A witness strategy can be constructed based on the result from the analysis of the product game in the following way \cite[authoryears][Svorenova2017]:
At every step of the trace in the LSS, a control input is chosen such that the corresponding play in the product game only visits states from the set $\AlmostAdv{\ProductGame}$.
The control inputs are obtained from the $\ActC$-regions associated with player 1 actions targeting only states in $P_\Yes$.
These are selected in a round-robin fashion every time a trace visits a player 1 state.
This is a pure strategy that requires a finite memory to keep track of the round-robin selection of actions.

In general, memoryless strategies exist for one-pair Streett objectives of 2½-player games, because they can be converted into parity-3 objectives for which such strategies are known to exist \cite[authoryears][Chatterjee2012].
Deeper analysis of the game is required to derive these strategies.
A memoryless strategy for problems with robust solutions based on a geometric distance metric induced by the LSS dynamics is briefly discussed in \in{Section}[sec:cases-corridor-synthesis].
Otherwise, the controller synthesis problem is not a focus of this work.


            \stopsection

            \startsection[title={Properties},reference=sec:abstraction-properties]
                Soundness, Convergence (progress guarantee), Complexity.
Only reference \cite[Svorenova2017] or repeat?


            \stopsection

        \stopchapter

        \startchapter[title={Abstraction Refinement}]

            The game-based abstraction requires a state space partition from which to derive player 1 states.
As set up in section \in[sec:abstraction-graph-playerone], the initial partition is governed by the set of linear predicates.
This partition makes sense because it allows a unique association of states to sets of (satisfied) linear predicates, but in general the partition will be too coarse to answer a given analysis question, leaving most states in the set $P_\Maybe$.

In this chapter, procedures are presented that refine the partition so that a solution to the analysis problem can be found.
Ideally, the procedures should work without manual intervection, so that the abstraction-analysis-refinement cycle from Figure \in[fig:problem-approach-flowchart] is able to solve a given analysis problem autonomously.



            \startsection[title={Reachability Decomposition},reference=sec:refinement-reachability]
                Refinement necessary component due to introduction of abstraction.
Combines analysis results and knowledge of dynamics into useful heuristics.
Presented here are a few heuristics.

Why reachability is the only problem that needs to be solved: in product automaton, for each copy of the LSS abstraction states a reachability problem can be solved with the target of reaching a region that enables an automaton transition.
By targeting individual transitions, the refinement decomposes into a series of reachability problems.
Therefore, presentation of positive refinement here for reachability only.

Problem of selection of transitions for refinement is complex, automaton is hard, pops up again in controller synthesis.
Refine every transition, but some may be unused in a winning strategy.
Generally, an existing yes region is not available.
Exception is co-safe objectives: work backwards from final states, which are immediately satisfying.
For infinite objectives, which many transitions optimal path through the automaton is not obvious.


            \stopsection

            \startsection[title={Negative Refinement},reference=sec:refinement-negative]
                \cite[Svorenova2017] used the notion of negative refinement, looking at player 1 states where player two can ensure to win the game with non-zero proability.
They refine a state $\Tuple{\State{i}}{q}$ if

\placeformula[fml:refinement-negative-unsafecondition]
\startformula
    \forall (\PlayerOneAction{i}{J}) \in Act \quad \exists (\PlayerTwoAction{i}{J}{K}) \in Act \quad \exists k \in K : \State{k} \in \NoStates{q'} \EndPeriod
\stopformula

Repeat $\Attr$-based refinement and extend it with a second procedure that covers additional cases detected by this condition.


\startbuffer[buf:refinement-negative-attractor-algorithm]
    \startalgorithmic[numbering=no,margin=0em]
        \INPUT{Player 1 state $\Tuple{\State{i}}{q}$}
        \OUTPUT{Partition $Y = \IndexedSet{Y_n}{n \in N}$: $\displaystyle \bigcup_{n \in N} Y_n = \State{i}$}
    \stopalgorithmic
    \startalgorithmic
        \STATE{$Y \leftarrow \Set{\State{i}}$}
        \FORALL{$q \in Q$}
            \IF{$\State{i} \in \MaybeStates{q}$}
                \STATE{$q' \leftarrow \QNext{i}{q}$}
                \STATE{$Y' \leftarrow \emptyset$}
                \FORALL{$Y_n \in Y$}
                    \STATE{$A \leftarrow \Attractor{Y_n}{\ControlSpace}{\NoStates{q'}}$}
                    \STATE{$Y' \leftarrow Y' \cup \Convexify(A) \cup \Convexify(Y_n \setminus A)$}
                \ENDFOR
                \STATE{$Y \leftarrow Y'$}
            \ENDIF
        \ENDFOR
        \RETURN{Y}
    \stopalgorithmic
\stopbuffer

\placealgorithm[top][alg:refinement-negative-attractor]{
    Negative Attractor refinement of a player 1 state $\Tuple{\State{i}}{q}$.
}{
    \startframedtext[width=\textwidth,frame=off]
        \getbuffer[buf:refinement-negative-attractor-algorithm]
    \stopframedtext
}

Refining with $\Attractor{\State{i}}{\ControlSpace}{\NoStates{q'}}$ yields region of guaranteed bad-states.
No-stuff is inevitable, every control input has non-zero probability of leading to bad state for every state in polytope.

TODO algorithm

Impossible to do better and still have a guarantee for reachability problem.
Good idea to always do this first as it has a guarantee.

This is much stricter than condition (\in[fml:refinement-negative-unsafecondition]), where bad stuff can be avoided if player 2 plays cooperatively.
For the Attractor states removed here, game cannot be won even if player 2 plays cooperatively.


            \stopsection

            \startsection[title={Positive Refinement},reference=sec:refinement-positive]
                Instead of only avoiding bad things, actively enable good things.
Remember, ensuring good things is non-trivial, because any bad event with non-zero probability breaks everything.


\startsubsection[title={Positive Robust Attractor}]

    AttrR+, by \cite[Svorenova2017]
    Single-step lookahead.
    Progress guarantees, allows cheap analysis for reachability problems.
    AttrR+ treats system like it is deterministic.
    Waste of probabilistic properties.
    Non-convex target regions can lead to explosion of complexity.
    No progress possible if deterministic transition cannot be found, approximations do not deliver same guarantees.

\stopsubsection


\startsubsection[title={Control Input Selection}]

    AttrR+ requires control input to refine wrt.
    \cite[Svorenova2017] selected this by taking a random action, paritioning it arbitrarily and using the parts.
    From experience: this produces small states, progress is slower than it should be.

    Improvement: rate action by how much its posterior intersects target/no-region/etc, then pick "best" one.
    Do the same with parts.
    However, this still leaves room for improvement.

    Alternative: Monte Carlo-based approach
    Sample a few points from within the polytope.
    Use $ActR$ to find control inputs for those points that exclusively lead to target (if there are none, try another sample point).
    Then cluster these control inputs and select one that most points "agree" with.
    Expensive but exhaustive: compute all overlaps, similar to action computation and pick one with most contributung points.
    In practice, adding vertices to sample point pool improved performance, as these points are often associated with most "extreme" action requirements.

\stopsubsection


\startsubsection[title={Target Preprocessing}]

    Small targets that cannot be transitioned to exclusively are a problem for deterministic refinement methods such as AttrR+.
    Even regions that can be targeted individually but that are very small are a problem, as a sufficiently big target area has to be built up first, before well-performing refinement is possible.
    Due to probabilistic nature this is not always optimal.
    Target may be enlarged in these cases, carefully, so that probability of going to "real" target is non-zero everywhere in extended region and extended region cannot be targeted individually either.
    Extended region may require refinement itself, e.g. saftey refinement wrt entire extended region.

    Linear dynamics means that convex targets usually have convex reachability solutions (due to $\Pre$ being convex if state, action and random space are convex).
    Non-convex targets often introduce more non-convexity which leads to state explosion.
    Target approximation can work by hoping for probabilisitic "magic" but hard to predict when this works and when it doesn't.
    Smoothing of target region, up to taking convex hull.

\stopsection


\startsubsection[title={Layered Refinement}]

    Problem with pure AttrR+ refinement is that progress guarantee only works for AttrR-generated states.
    For some, reachability can be decided with ActR but purely deterministic view means over-refinement or conservative analysis, slowing progress.
    Analysing system from time to time required, but relatively expensive due to need for action computation.

    Layer idea outlined by \cite[Svorenova2017].
    Refinement decouples into separate subproblems.
    Problem of \cite[Svorenova2017]: Pre is not the ideal operator for procedure.

    Innovation: PreR, shrinking of layer-generating control space (show graphic/example that illustrates "convergence" behaviour at edges).
    Outline algorithm (layer generation, removal of known no-states, control selection and AttrR in inner iterations, small-state supression because states smaller W will always transition away, loops impossible), discuss tuning parameters.
    Discuss probabilistic aspect of approach (steps are deterministic, but no overall guarantee since solution for outer layers depends on solution of inner layers) but mention that at core it is deterministic.
    Hybrid approach between multi-step and single-step: general idea of moving from layer to layer is multi-step, but transitions inside layers are single step.
    Non-optimality of layers in terms of numbers of states generated (PreR is deterministic transition).
    Problem again: PreR does not exists if X - W = 0, extension by minkowski sum with origin-centered W and additional extension when target at the edge of the control space.

\stopsubsection


            \stopsection

            \startsection[title={A Reduction Approach},reference=sec:refinement-negative]
                Holistic refinement procedures seek patterns in the entire product game graph.
The procedures presented in \in{Section}[sec:refinement-holistic] only consider a single step of the system dynamics due to the high computational cost of obtaining deeper insights into the game graph.
But without the consideration of multi-step behaviour, refinement progress is limited to a local scale.
Due to the high computational cost of construction and analysis of the product game, patterns suitable for non-local refinement are expensive to identify.
Computational demands can be reduced through the extraction of meaningful subsets of the game graph.
This is particularly interesting in the context of positive refinement.
Every satisfying player 1 strategy based on a subset of the product game can also be realized in the full product game as long as the power of player 2 has not been reduced by the subset selection.

One possible subset extraction method is to remove all but one action from every player 1 state of the game graph.
This subset selection only limits the abilities of player 1, so almost-sure satisfying strategies obtained from it can be transferred to the full game.
The reduction allows to remove player 1 from the game entirely so that it turns into an MDP.
The lower computational complexity of MDP analysis and uncontrolled dynamics improves the feasibility of multi-step analysis.
A refinement procedure based on this reduction can be constructed in the following way:
First a control region is assigned to every element of the state space partition.
This turns the controlled dynamcis of the LSS into piece-wise dynamics without control.
An MDP abstraction is constructed on top of the piece-wise system.
The state space is refined based on insights from the MDP.
Finally, the state space partition of the MDP system is transferred back to the original system.

Experience gathered with a prototype implementation of such a refinement scheme revealed drawbacks.
The choice of a control region for every state space partition element is non-trivial.
A certain amount of foresight is required in order to obtain an effective refinement procedure since badly chosen control regions significantly limit its potential.
The non-continuous piece-wise dynamics causes the refinement to produce jagged regions, which result in a very fine partitioning of the state space (see \in{Figure}[fig:refinement-transition-jagged]a where this issue is illustrated for positive robust refinement).
Such jaggedness quickly snowballs into an explosion of the number of (small) partition elements, which can offset the complexity reduction gained from the MDP abstraction.
Due to these issues, the approach was abandoned in favour of another subset extraction technique.

Instead of reducing the number of actions per state, one can directly reduce the number of states.
This makes sense when the remaining subset of states and actions forms a meaningful sub-problem of the entire product game.
It is not uncommon to encounter objectives made up of multiple separable sub-objectives, e.g.\ a safety property in conjunction with a recurrence objective, which is itself a sequence of reachability problems.
If these sub-objectives can be separated in the product game and solved individually, one can expect that a solution for the composite objective emerges.
For refinement, this is advantageous because the sub-problems will often provide more immediate analysis feedback after refinement.
The analysis of simple sub-problems might be possible without the need to construct a game abstraction altogether.
E.g., saftey properties can be verified with only the $\ActR$-operator.
Parallel refinement of the sub-problems is generally possible but restricted by the fact that state space partition is shared between the individual sub-problems in the product game.


            \stopsection

        \stopchapter

        \startchapter[title={An Interactive Implementation}]

            Part of this work has been the creation of a web-based application that allows interactive exploration of the presented LSS abstraction-analysis-refinement procedure.
Its purpose is both educational and experimental.
The application supports customizable problem setups in 1 and 2 dimensions and gives the user control over system refinement and analysis as well as controller synthesis.
This chapter motivates its development and presents the main features.



            \startsection[title={The Case for Interactive Exploration}]
                ?
            \stopsection

            \startsection[title={Functionality}]
                ?
            \stopsection

        \stopchapter

        \startchapter[title={Case Studies}]

            3 case studies to illustrate challenges, demo features and show practical applicability of procedure.
2D problems for representation purposes.
Stress again that 2D is not a restriction, procedures are general and important challenges all appear already in low dimensions.

Before diving into the example, demonstration of a major development for thesis: interactive web application for system exploration and procedure inspection.



            \startsection[title={Double Integrator},reference=sec:cases-integrator]
                Consider the double integrator dynamics

\startformula
    \VecX_{t+1} = \TwoByTwo{1}{1}{0}{1} \VecState_{t} + \TwoByOne{0.5}{1} \VecControl_{t} + \VecRandom_{t} \EndComma
\stopformula

where $\VecState_{t} \in \StateSpace = \ClosedInterval{-5}{5} \times \ClosedInterval{-3}{3}$, $\VecControl_{t} \in \ControlSpace = \ClosedInterval{-1}{1}$ and $\VecRandom_{t} \in \RandomSpace = \ClosedInterval{-0.1}{0.1}^2$.
The objective is to reach $\ClosedInterval{-1}{1}^2$, therefore the linear predicates $\Predicate_{1}$, $\Predicate_{2}$, $\Predicate_{3}$ and $\Predicate_{4}$ are introduced, corresponding to halfspaces governed by the inequalities $x \leq 1$, $-x \leq 1$, $y \leq 1$ and $-y \leq 1$.
The LTL formula expressing this reachability objective is then

\startformula
    \Finally ( p_1 \wedge p_2 \wedge p_3 \wedge p_4 )
\stopformula

with a co-safe interpretation.
\cite[Svorenova2017] used this exact system for their case studies and it is used here again as a baseline for comparison.

\placefigure[top][fig:cases-integrator-initial]{
    The double integrator test system and its initial partition.
    Polytopes $\State{1}$ to $\State{4}$ (grey) are outer states from the decomposition of $\ExtendedStateSpace \setminus \StateSpace$.
    The reachability target is $\State{12}$ (green).
}{
    \externalfigure[cases-integrator-initial][width=\textwidth]
}


\startsubsection[title={Negative Refinement},reference=sec:cases-integrator-negative]

    \placetable[top][tab:cases-integrator-negative]{
        TODO \par TODO
    }{
        \RefinementTable{
            \RefinementTableRow[iteration=1,polys=9,onestates=28,oneactions=44,twostates=26,twoactions=216,
                                total={0:00},refinement={-},gamegraph={0:00},analysis={0:00},
                                yes=6.7,no=0.0,maybe=93.3,figure=cases-integrator-iteration1]
            \RefinementTableRow[iteration=2,polys=13,onestates=36,oneactions=60,twostates=38,twoactions=362,
                                total={0:00},refinement={0:00},gamegraph={0:00},analysis={0:00},
                                yes=6.7,no=11.3,maybe=82.1,figure=cases-integrator-iteration2]
            \RefinementTableRow[iteration=3,polys=17,onestates=44,oneactions=80,twostates=50,twoactions=592,
                                total={0:01},refinement={0:00},gamegraph={0:00},analysis={0:00},
                                yes=6.7,no=16.1,maybe=77.3,figure=cases-integrator-iteration3]
            \RefinementTableRow[iteration=4,polys=21,onestates=52,oneactions=104,twostates=66,twoactions=794,
                                total={0:01},refinement={0:00},gamegraph={0:00},analysis={0:00},
                                yes=6.7,no=17.1,maybe=76.2,figure=cases-integrator-iteration4]
        }
    }

    Negative attractor refinement.
    Iterate until convergence.
    Introduction into refinement progress reporting.

    Note that due to progress guarantee analysis is technically not necessary and procedure could be speed up.
    Analysis status of other states could be affected by refinement which can only be determined by an analysis.

    Show that simplification with multiple iterations of negative attractor does not generally work because of self-loops.

\stopsubsection



\startsubsection[title={Positive Robust Refinement},reference=sec:cases-integrator-positive]

    \placetable[top][tab:cases-integrator-positive-single]{
        TODO \par TODO
    }{
        \RefinementTable{
            \RefinementTableRow[iteration=5,polys=48,onestates=106,oneactions=476,twostates=407,twoactions=6188,
                                total={0:04},refinement={0:00},gamegraph={0:03},analysis={0:00},
                                yes=13.9,no=17.1,maybe=68.9,figure=cases-integrator-iteration5-prs]
            \RefinementTableRow[iteration=6,polys=166,onestates=342,oneactions=3723,twostates=3519,twoactions=79332,
                                total={2:07},refinement={0:02},gamegraph={1:59},analysis={0:03},
                                yes=36.5,no=17.1,maybe=46.4,figure=cases-integrator-iteration6-prs]
            \RefinementTableRow[iteration=7,polys=301,onestates=612,oneactions=5902,twostates=5459,twoactions=111705,
                                total={5:59},refinement={0:09},gamegraph={3:40},analysis={0:03},
                                yes=54.9,no=17.1,maybe=28.0,figure=cases-integrator-iteration7-prs]
            \RefinementTableRow[iteration=8,polys=362,onestates=734,oneactions=3096,twostates=2458,twoactions=54578,
                                total={9:41},refinement={0:07},gamegraph={3:33},analysis={0:02},
                                yes=67.0,no=17.1,maybe=15.9,figure=cases-integrator-iteration8-prs]}
    }

    \placetable[top][tab:cases-integrator-positive-double]{
        TODO \par TODO
    }{
        \RefinementTable{
            \RefinementTableRow[iteration=5,polys=68,onestates=146,oneactions=885,twostates=796,twoactions=13251,
                                total={0:11},refinement={0:00},gamegraph={0:10},analysis={0:00},
                                yes=18.4,no=17.1,maybe=64.5,figure=cases-integrator-iteration5-prm]
            \RefinementTableRow[iteration=6,polys=236,onestates=482,oneactions=6293,twostates=6003,twoactions=141782,
                                total={4:32},refinement={0:06},gamegraph={4:08},analysis={0:06},
                                yes=53.9,no=17.1,maybe=29.0,figure=cases-integrator-iteration6-prm]
            \RefinementTableRow[iteration=7,polys=333,onestates=676,oneactions=4114,twostates=3567,twoactions=53441,
                                total={6:09},refinement={0:06},gamegraph={1:29},analysis={0:02},
                                yes=77.0,no=17.1,maybe=5.9,figure=cases-integrator-iteration7-prm]
            \RefinementTableRow[iteration=8,polys=361,onestates=732,oneactions=1367,twostates=682,twoactions=7436,
                                total={6:18},refinement={0:01},gamegraph={0:08},analysis={0:00},
                                yes=82.5,no=17.1,maybe=0.3,figure=cases-integrator-iteration8-prm]
        }
    }

    \placetable[top][tab:cases-integrator-positive-suppressed]{
        TODO \par TODO
    }{
        \RefinementTable{
            \RefinementTableRow[iteration=5,polys=51,onestates=112,oneactions=507,twostates=435,twoactions=6147,
                                total={0:05},refinement={0:00},gamegraph={0:04},analysis={0:00},
                                yes=18.5,no=17.1,maybe=64.3,figure=cases-integrator-iteration5-prms]
            \RefinementTableRow[iteration=6,polys=99,onestates=208,oneactions=1360,twostates=1221,twoactions=21859,
                                total={0:26},refinement={0:02},gamegraph={0:18},analysis={0:01},
                                yes=42.4,no=17.1,maybe=40.4,figure=cases-integrator-iteration6-prms]
            \RefinementTableRow[iteration=7,polys=134,onestates=278,oneactions=1291,twostates=1069,twoactions=16174,
                                total={0:41},refinement={0:02},gamegraph={0:13},analysis={0:00},
                                yes=65.6,no=17.1,maybe=17.3,figure=cases-integrator-iteration7-prms]
            \RefinementTableRow[iteration=8,polys=161,onestates=332,oneactions=962,twostates=671,twoactions=8157,
                                total={0:48},refinement={0:01},gamegraph={0:05},analysis={0:00},
                                yes=80.1,no=17.1,maybe=2.7,figure=cases-integrator-iteration8-prms]
        }
    }

    Recreation of Svorenova positive refinement procedure, who refined wrt 3 random actions.
    Control selection for $\RefinePos$ is generally better than action-based, therefore, two single-step applications here.
    Also use PreR?

    Problem is already reachability and co-safe, therefore extraction of robust reachability problem trivial (transition $q_0$ to $q_1$).
    Positive refinement, two iterations per step, with target expansion.
    More progress per iteration.
    Should be a bit faster than 2x single-step, but similar state counts.
    Progress guarantee could be used to speed up, but this has not been implemented.
    It would only work if the target region is recognized as part of the yes states too, not in general.

    Positive refinement, two iterations per step, small state suppression.
    Additional iteration required but faster progress, much fewer states.
    Show comparison (super-small states).
    Similar result if 8 iterations are carried out directly.
    Again, for this reachability problem analysis time could be reduced by using progress guarantee, which small state-suppression delivers.
    Compare to convex hull overapproximation which gets stuck.

\stopsubsection


\startsubsection[title={Positive Robust Refinement with Layer Decomposition},reference=sec:cases-integrator-layered]

    \placetable[top][tab:cases-integrator-layered-original]{
        TODO \par TODO
    }{
        \RefinementTable{
            \RefinementTableRow[iteration=5,polys=166,onestates=342,oneactions=4364,twostates=4177,twoactions=79056,
                                total={1:24},refinement={0:03},gamegraph={1:16},analysis={0:05},
                                yes=76.3,no=17.1,maybe=6.6,figure=cases-integrator-iteration5-lprms]
        }
    }

    \placetable[top][tab:cases-integrator-layered-shrunk]{
        TODO \par TODO
    }{
        \RefinementTable{
            \RefinementTableRow[iteration=5,polys=129,onestates=268,oneactions=2565,twostates=2415,twoactions=36246,
                                total={0:29},refinement={0:01},gamegraph={0:25},analysis={0:02},
                                yes=81.8,no=17.1,maybe=1.0,figure=cases-integrator-iteration5-slprms]
        }
    }

    Positive robust refinement with layer decomposition, target expansion, small state suppression and PreR, 4 iterations.
    Show how limit behaviour causes small states to appear despite suppression.
    Instead, show with 95\% shrunk U ($\ClosedInterval{-0.95}{0.95}$).
    Limit behaviour gone, performance comparable to 9 iterations direct, but less costly refinement due to simpler robust sub-problems.
    Structure of layered refinement prescribes minimum number of states.

    Show unintuitive behaviour: only 2 layers at a time, faster time to solution despite more states.
    Reason: game simplification.
    Conclusion: progress guarantee should be used if it exists.

\stopsubsection


\startbuffer[buf:cases-integrator-results-statistics]
    \setupTABLE[frame=off,rightframe=on]
    \setupTABLE[c][first][rightframe=off]
    \setupTABLE[r][last][bottomframe=on]
    \bTABLE[align={middle,lohi}]
        \bTR[topframe=off]
            \bTH[width=0.06\textheight] \eTH
            \bTH[width=0.09\textheight] \eTH
            \bTH[width=0.17\textheight] single-step \times 2 \eTH
            \bTH[width=0.17\textheight] two-step \eTH
            \bTH[width=0.17\textheight] two-step \par no small \eTH
            \bTH[width=0.17\textheight] layers \par four-step \par no small \eTH
            \bTH[width=0.17\textheight] shrunk layers \par four-step \par no small \eTH
        \eTR
        \bTR[topframe=off]
            \bTH Iter. \eTH
            \bTH \eTH
            \bTD min / avg / max \eTD
            \bTD min / avg / max \eTD
            \bTD min / avg / max \eTD
            \bTD min / avg / max \eTD
            \bTD min / avg / max \eTD
        \eTR
        \bTR[topframe=on]
            \bTH[nr=3] 1-4 \par neg. \eTH
            \bTD polytopes \eTD
            \bTD[nc=5] 21 / 21 / 21 \eTD
        \eTR
        \bTR
            \bTD elapsed \eTD
            \bTD[nc=5] 0:01 / 0:01 / 0:01 \eTD
        \eTR
        \bTR
            \bTD \% decided \eTD
            \bTD[nc=5] 23.8 / 23.8 / 23.8 \eTD
        \eTR
        \bTR[topframe=on]
            \bTH[nr=3] 5 \eTH
            \bTD polytopes \eTD
            \bTD 48 / 50 / 53 \eTD
            \bTD 68 / 85 / 97 \eTD
            \bTD 49 / 54 / 60 \eTD
            \bTD 165 / 168 / 171 \eTD
            \bTD 128 / 130 / 133 \eTD
        \eTR
        \bTR
            \bTD elapsed \eTD
            \bTD 0:04 / 0:04 / 0:05 \eTD
            \bTD 0:11 / 0:21 / 0:36 \eTD
            \bTD 0:04 / 0:05 / 0:07 \eTD
            \bTD 1:19 / 1:24 / 1:27 \eTD
            \bTD 0:28 / 0:29 / 0:31 \eTD
        \eTR
        \bTR
            \bTD \% decided \eTD
            \bTD 30.7 / 31.0 / 31.4 \eTD
            \bTD 35.5 / 38.4 / 39.6 \eTD
            \bTD 35.5 / 36.7 / 39.2 \eTD
            \bTD 93.4 / 95.0 / 96.7 \eTD
            \bTD 99.0 / 99.0 / 99.0 \eTD
        \eTR
        \bTR[topframe=on]
            \bTH[nr=3] 6 \eTH
            \bTD polytopes \eTD
            \bTD 166 / 194 / 239 \eTD
            \bTD 227 / 277 / 317 \eTD
            \bTD 83 / 99 / 114 \eTD
            \bTD  \eTD
            \bTD  \eTD
        \eTR
        \bTR
            \bTD elapsed \eTD
            \bTD 1:55 / 4:00 / 11:56 \eTD
            \bTD 2:16 / 6:06 / 16:36 \eTD
            \bTD 0:18 / 0:25 / 0:36 \eTD
            \bTD - \eTD
            \bTD - \eTD
        \eTR
        \bTR
            \bTD \% decided \eTD
            \bTD 53.6 / 55.8 / 58.5 \eTD
            \bTD 70.7 / 74.8 / 79.4 \eTD
            \bTD 53.6 / 61.8 / 71.0 \eTD
            \bTD  \eTD
            \bTD  \eTD
        \eTR
        \bTR[topframe=on]
            \bTH[nr=3] 7 \eTH
            \bTD polytopes \eTD
            \bTD 239 / 280 / 321 \eTD
            \bTD 300 / 353 / 394 \eTD
            \bTD 124 / 132 / 137 \eTD
            \bTD  \eTD
            \bTD  \eTD
        \eTR
        \bTR
            \bTD elapsed \eTD
            \bTD 4:04 / 7:14 / 17:13 \eTD
            \bTD 3:10 / 7:24 / 17:59 \eTD
            \bTD 0:26 / 0:41 / 0:50 \eTD
            \bTD - \eTD
            \bTD - \eTD
        \eTR
        \bTR
            \bTD \% decided \eTD
            \bTD 69.3 / 72.0 / 73.3 \eTD
            \bTD 87.8 / 95.6 / 99.5 \eTD
            \bTD 76.2 / 83.8 / 89.8 \eTD
            \bTD  \eTD
            \bTD  \eTD
        \eTR
        \bTR[topframe=on]
            \bTH[nr=3,bottomframe=on] 8 \eTH
            \bTD polytopes \eTD
            \bTD 296 / 351 / 408 \eTD
            \bTD 333 / 389 / 435 \eTD
            \bTD 141 / 151 / 163 \eTD
            \bTD  \eTD
            \bTD  \eTD
        \eTR
        \bTR
            \bTD elapsed \eTD
            \bTD 6:15 / 12:35 / 27:24 \eTD
            \bTD 3:26 / 8:53 / 18:07 \eTD
            \bTD 0:31 / 0:47 / 0:58 \eTD
            \bTD - \eTD
            \bTD - \eTD
        \eTR
        \bTR
            \bTD \% decided \eTD
            \bTD 81.6 / 84.3 / 85.6 \eTD
            \bTD 95.5 / 99.2 / 99.8 \eTD
            \bTD 89.9 / 96.5 / 99.3 \eTD
            \bTD  \eTD
            \bTD  \eTD
        \eTR
    \eTABLE
\stopbuffer

\startsubsection[title={Results},reference=sec:cases-integrator-results]

    % Rotation value has to be flipped for some reason (maybe because it refers
    % to the page that includes the figure, not the figure itself?
    % % TODO check rotation in final document
    \placetable[here,\doifoddpageelse{270}{90}][tab:cases-integrator-results-statistics]{
        TODO
    }{
        \startframedtext[width=\textheight,offset=0mm,frame=off,topframe=off,bottomframe=off]
            \getbuffer[buf:cases-integrator-results-statistics]
        \stopframedtext
    }

    Show table with average numbers for all problems.
    Talk about randomization.
    Give numbers as percent progress per second?
    Layered refinement performs best, importance of small-state suppression (combats over-refinement) and problems with jaggedness (which just causes more jaggedness).
    Plot number of state space partition elements vs. game size for full analysis.
    Show how product game simplification can save time.
    Illustrated also by 2-layers at a time experiment.

\stopsubsection


            \stopsection

            \startsection[title={Corridor},reference=sec:cases-corridor]
                Reachability/Avoidance through a corridor.

\startformula
    \VecX_{t+1} = \TwoByTwo{1}{0}{0}{1} \VecState_{t} + \TwoByTwo{1}{0}{0}{1} \VecControl_{t} + \VecRandom_{t} \EndComma
\stopformula

where $\VecState_{t} \in \StateSpace = \ClosedInterval{0}{4} \times \ClosedInterval{0}{3}$, $\VecControl_{t} \in \ControlSpace = \ClosedInterval{-0.5}{0.5}^2$ and $\VecRandom_{t} \in \RandomSpace = \ClosedInterval{-0.1}{0.1}^2$.

\startsubsection[title={Reachability Analysis},reference=sec:cases-corridor-reachability]

    \placetable[top][tab:cases-corridor-reachability]{
        TODO
        See \in{section}[sec:cases-corridor-reachability] for discussion.
    }{
        \RefinementTable{
            \RefinementTableRow[iteration={},polys=65,onestates=140,oneactions=6856,twostates=6778,twoactions=85804,
                                total={1:35},refinement={0:02},gamegraph={1:29},analysis={0:04},
                                yes=69.7,no=30.3,maybe=0.0,figure=cases-corridor-reachability]
        }
    }

    First look at objective avoidance and reachability objective

    \startformula
        ( \neg \pi ) \Until \varphi
    \stopformula

    co-safe interpretation.
    Narrow passage between two rooms, challenge is the relatively precise control required to enter the corridor.

    Solve with multi-step robust refinement.
    Show problem with long thin states and skipping of small states.
    Show variations in progress due to randomization.

    Layered refinement.
    Demonstrate how variability of solution due to randomized $\RefinePos$ is stabilized.
    Note that refinement around corridor entrance is finer that in rest of the room.

    2-dimensional control space.
    Compare game complexity to double integrator with 1D control space.

\stopsubsection


\startsubsection[title={Reachability Controller}]

    \placefigure[top][fig:cases-corridor-reachability-trace]{
        TODO
    }{
        \startcombination[nx=2,ny=1,distance=10mm]
            {\externalfigure[cases-corridor-reachability-trace-robin][width=0.49\textwidth]}{}
            {\externalfigure[cases-corridor-reachability-trace-layers][width=0.49\textwidth]}{}
        \stopcombination
    }

    Problematic performance of round-robin controller which does not violate avoidance objective but does not want to enter the corridor.
    Apply layered approach to reachability.
    Show off reordered round-robin controller for reachability that does better job of guiding the trace through the corridor.

\stopsubsection


\startreusableMPgraphic{cases-corridor-recurrence-automaton}
    beginfig(0);
        with spacing((17,15)) matrix.a(9,9);
        node_double.a[1][1](btex $q_0$ etex);
        node_dash.a[1][8](btex $q_1$ etex);
        node_dash.a[8][1](btex $q_2$ etex);
        % Outgoing transitions of q0
        incoming(0, "") (a[1][1]) 180;
        loop.top(.4, btex \small \;$ \neg \theta \wedge \phi \wedge \mu$ etex) (a[1][1]) 90;
        arrow.top(.5, btex \small $ \neg \theta \wedge \phi $ etex) (a[1][1],a[1][8]) a[1][1].c..a[0][4].c..a[0][5].c..a[1][8].c;
        arrow.rt(.5, btex \small $ \neg (\theta \vee \phi)$ etex) (a[1][1],a[8][1]) a[1][1].c..a[4][2].c..a[5][2].c..a[8][1].c;
        % Outgoing transitions of q1
        loop.top(.4, btex \small \;$ \neg (\theta \vee \phi) $ etex) (a[1][8]) 90;
        arrow.bot(.5, btex \small $ \neg \theta \wedge \phi \wedge \mu $ etex) (a[1][8],a[1][1]) a[1][8].c..a[2][5].c..a[2][4].c..a[1][1].c;
        arrow.rt(.4, btex \small \qquad$ \neg (\theta \vee \mu) \wedge \phi $ etex) (a[1][8],a[8][1]) a[1][8].c..a[2][8].c..a[8][2].c..a[8][1].c;
        % Outgoing transitions of q2
        loop.bot(.5, btex \small \;$ \neg (\theta \vee \mu) $ etex) (a[8][1]) 270;
        arrow.lft(.5, btex \small $ \neg \theta \wedge \mu $ etex) (a[8][1],a[1][1]) a[8][1].c..a[5][0].c..a[4][0].c..a[1][1].c;
    endfig;
\stopreusableMPgraphic

\startreusableMPgraphic{cases-corridor-recurrence-automaton-pruned}
    beginfig(0);
        with spacing((17,15)) matrix.a(9,9);
        node_double.a[1][1](btex $q_0$ etex);
        node_dash.a[1][8](btex $q_1$ etex);
        node_dash.a[8][1](btex $q_2$ etex);
        % Outgoing transitions of q0
        incoming(0, "") (a[1][1]) 180;
        arrow.top(.5, btex \small $ \neg \theta \wedge \phi $ etex) (a[1][1],a[1][8]) a[1][1].c..a[0][4].c..a[0][5].c..a[1][8].c;
        arrow.rt(.5, btex \small $ \neg (\theta \vee \phi)$ etex) (a[1][1],a[8][1]) a[1][1].c..a[4][2].c..a[5][2].c..a[8][1].c;
        % Outgoing transitions of q1
        loop.top(.4, btex \small \;$ \neg (\theta \vee \phi) $ etex) (a[1][8]) 90;
        arrow.rt(.4, btex \small \qquad$ \neg (\theta \vee \mu) \wedge \phi $ etex) (a[1][8],a[8][1]) a[1][8].c..a[2][8].c..a[8][2].c..a[8][1].c;
        % Outgoing transitions of q2
        loop.bot(.5, btex \small \;$ \neg (\theta \vee \mu) $ etex) (a[8][1]) 270;
        arrow.lft(.5, btex \small $ \neg \theta \wedge \mu $ etex) (a[8][1],a[1][1]) a[8][1].c..a[5][0].c..a[4][0].c..a[1][1].c;
    endfig;
\stopreusableMPgraphic

\startsubsection[title={2-Recurrence and Safety},reference=sec:cases-corridor-recurrence]

    \placefigure[top][fig:cases-corridor-recurrence-automaton]{
        TODO
    }{
        \startcombination[nx=2,ny=1,distance=10mm]
            {\reuseMPgraphic{cases-corridor-recurrence-automaton}}{}
            {\reuseMPgraphic{cases-corridor-recurrence-automaton-pruned}}{}
        \stopcombination
    }

    Finally consider a sighly more complex objective that requires traces to visit both rooms over and over again while avoiding the walls of the corridor

    \startformula
        \Globally (\neg a \wedge \Finally b \wedge \Finally c )
    \stopformula

    No co-safe interpretation.

    Explain repeated (q0 q1 q2) trace as decomposition targets.
    Mention that refinement for one direction only does not lead to satisfaction for entire objective.

    \placetable[top][tab:cases-corridor-recurrence]{
        TODO
        See \in{section}[sec:cases-corridor-recurrence] for discussion.
    }{
        \RefinementTable{
            \RefinementTableRow[iteration={},polys=122,onestates=339,oneactions=46768,twostates=25139,twoactions=610511,
                                total={3:11},refinement={0:02},gamegraph={2:09},analysis={1:00},
                                yes=69.7,no=30.3,maybe=0.0,figure=cases-corridor-recurrence]
        }
    }

    Apply layered refinement for both transitions.
    Note that overrefinement in corridor occurs due to fixed layer structure.
    Mention blow up of state space and that no progress guarantee can help here as nothing is in yes-region at the beginning.

    \placefigure[top][fig:cases-corridor-recurrence-trace]{
        TODO
    }{
        \externalfigure[cases-corridor-recurrence-trace][width=0.8\textwidth]
    }
    
    Use layered approach again for controller synthesis.

\stopsubsection


            \stopsection

            \startsection[title={Inverted Pendulum},reference=sec:cases-pendulum]
                Non-co-safe objective: $\Globally \Finally \varphi$.
Practical application (link yt-video?), with real world problems (e.g. time discretization, linearization, numerical errors).
Demonstration of the synthesized controller in a real-time scenario.


\startsubsection[title={Derivation}]

    Equation of motion: Angular momentum changes when torque is applied

    \startformula
        \dot\AngularMomentum = \sum_{i} \Torque_i = \sum_{i} \Length_i \Force_i
    \stopformula

    First torque from gravitational acceleration $ \Force_g = \Mass \Gravity \sin(\Angle) $.
    Second torque from friction, here friction force assumed to be proportional to the velocity of the swinging mass which is $\Length \dot\Angle$: $\Force_f = \Friction \Length \dot\Angle$.
    Third torque is control torque, applied as a force $\Force_c = \MaxForce \Thrust$ at the end of the pendulum.
    Using $\AngularMomentum = \MomentOfInertia \dot\Angle$ where $\MomentOfInertia = \Length^2 \Mass$ is the moment of inertia, and $\Torque = \Length \Force$:

    \startformula
        \startalign[n=3,align={middle,right,left}]
            \NC
            \NC \dot\AngularMomentum =
            \NC \sum_{i} \Length_i \Force_i
            \NR
            \NC \Leftrightarrow \quad
            \NC \Length^2 \Mass \ddot\Angle =
            \NC \Length (\Force_g + \Force_f + \Force_c)
            \NR
            \NC \Leftrightarrow \quad
            \NC \Length \Mass \ddot\Angle =
            \NC \Mass \Gravity \sin(\Angle) + \Friction \Length \dot\Angle + \MaxForce \Thrust
            \NR
            \NC \Leftrightarrow \quad
            \NC \ddot\Angle =
            \NC \frac{\Gravity}{\Length} \sin(\Angle) + \frac{\Friction}{\Mass} \dot\Angle + \frac{\MaxForce}{\Length \Mass} \Thrust
            \NR
        \stopalign
    \stopformula

    Linearization with small angle approximation $\sin(\Angle) \approx \Angle$.
    Discuss error size.

    \startformula
        \ddot\Angle = \frac{\Gravity}{\Length} \Angle + \frac{\Friction}{\Mass} \dot\Angle + \frac{1}{\Length \Mass} \Force_c
    \stopformula

    Rewrite as first-order system

    \startformula
        \DDt \TwoByOne{\Angle}{\dot\Angle}
        = \TwoByTwo{0}{1}{\frac{\Gravity}{\Length}}{\frac{\Friction}{\Mass}} \TwoByOne{\Angle}{\dot\Angle}
        + \TwoByOne{0}{\frac{\MaxForce}{\Length \Mass} } u
    \stopformula

    Introduce $\VecState_t = \TwoByOne{\Angle(t)}{\dot\Angle(t)}$ and $\VecControl_t = \OneByOne{u(t)}$ and discretize with an Euler-forward step:

    \startformula
        \startalign[n=3,align={middle,right,left}]
            \NC 
            \NC \DDt \TwoByOne{\Angle(t)}{\dot\Angle(t)} = 
            \NC \frac{\VecState_{t+1} - \VecState_{t}}{\Deltat} + \BigO(\Deltat)
            \NR
            \NC \Rightarrow \quad
            \NC \VecState_{t+1} =
            \NC \VecState_{t}
                + \DDt \TwoByOne{\Angle(t)}{\dot\Angle(t)} \Deltat
                + \BigO(\Deltat^2)
            \NR
            \NC \Rightarrow \quad
            \NC \VecState_{t+1} =
            \NC \VecState_{t}
                + \Deltat \TwoByTwo{0}{1}{\frac{\Gravity}{\Length}}{\frac{\Friction}{\Mass}} \VecState_t
                + \Deltat \TwoByOne{0}{\frac{\MaxForce}{\Length \Mass} } \VecControl_t
                + \BigO(\Deltat^2)
            \NR
            \NC \Rightarrow \quad
            \NC \VecState_{t+1} =
            \NC \TwoByTwo{1}{1}{\frac{\Gravity}{\Length} \Deltat}{1 + \frac{\Friction}{\Mass} \Deltat} \VecState_t
                + \TwoByOne{0}{\frac{\MaxForce}{\Length \Mass} \Deltat } \VecControl_t
                + \VecRandom_t
            \NR
        \stopalign
    \stopformula

    Where the discretization error term $\BigO(\Deltat^2)$ has been approximated by

    \startformula
        \VecRandom_t \in \ClosedInterval{?}{?} \times \ClosedInterval{?}{?}
    \stopformula

    based on the second order derivative ...

\stopsubsection


\startsubsection[title={Solution}]

    Implementation of interactive application.
    1st-order symplectic integrator (good conservation of energy) of non-approximated pendulum equation.
    Solution of analysis problem with layered positive refinement (?)
    Controller synthesized and embedded into real-time simulation.

    Show plots of a few angle/input vs. time graphs.
    Bring controller to its limits when physics approximation breaks down or time step is chosen too big (invalidate probabilistic perturbation assumption).

\stopsubsection


            \stopsection

        \stopchapter

        \startchapter[title={Discussion}]

            Wrap up:
Quickly summarize previous chapters and prepare outlook.



            \startsection[title={Summary of Contributions},reference=sec:discussion-contributions]
                List of innovative things introduced in the thesis.


            \stopsection

            \startsection[title={Conclusions},reference=sec:discussion-conclusions]
                Achievements and how they relate to goals set out at the beginning.
Review improvements and failings:
Addition and investigation of new refinement methods, particularly layered refinement which performs well for the shown example problems.
Demonstrate applicability for problems other than reachability.
Implementation of an interactive, educational tool for problem exploration in 1D and 2D.


            \stopsection

            \startsection[title={Outlook},reference=sec:discussion-outlook]
                Where to go next?  What doesn't work yet?
More complicated, multi-stage objectives.
Potent game-solver for objectives not expressible with parity 3 (nothing stands in the way of full LTL).
Better, automated selection of transitions for refinement targets and controller synthesis.
Higher dimensions in practice.
Refinement that takes probabilistic aspects of system fully into account.

Where are limitations and what seems possible?
Expectation is that higher dimensions quickly eat up improvements in refinement efficiency (see corridor vs double integrator game size with only one dimension difference in control space).
Is approach compatible with quantitative analysis?


            \stopsection

        \stopchapter

    \stopbodymatter

    \startbackmatter
        \startchapter[title=References]
            \placepublications
            \break % makes footer work properly
        \stopchapter
    \stopbackmatter

    \startsectionblock[declaration]
        \startsubject[title=Erklärung]

Ich versichere hiermit, dass ich die von mir eingereichte Abschlussarbeit
selbstständig verfasst und keine anderen als die angegebenen Quellen und
Hilfsmittel benutzt habe.

\blank[20mm]

\blackrule[width=12cm,height=0.2mm]
\hbox{\raise2mm\hbox{\small Ort, Datum, Unterschrift}}

\stopsubject


    \stopsectionblock

\stoptext

