Both the verification and synthesis probles are posed for the linear stochastic system and the specification.


\startsubsection[title={Satisfiability Analysis},reference=sec:problem-statement-analysis]

    First, the verification of system (\in[fml:problem-setup-lsseq]) with respect to a temporal logic objective $\varphi$ is considered.
    The goal is to determine for which initial states of the system a control strategy exists such that the objective $\varphi$ can be fulfilled with probability $1$.
    The result of this almost-sure analysis is the computed set of initial states $\InitialStates$.

    The specification $\varphi$ uses the linear predicates $\Predicates$ as atomic propositions.
    Objectives are limited to the extended GR(1) fragment of LTL, i.e. all formulas that can be translated to a deterministic \omega-automaton with a one-pair Streett acceptance condition.
    If a formula allows a co-safe interpretation in addition to an infinite interpretation, both may be considered.
    In the infinite interpretation the trace is never allowed to leave the state space, whereas in the co-safe interpretation the trace can go anywhere once the objective has been satisfied.

    A notable special case is reachability analysis with the co-safe objective $\varphi = \Finally \phi$, where $\phi$ is a propositional formula over $\Predicates$.
    This specification is satisfied when a trace reaches the region defined by $\phi$.
    Reachability analysis for the given problem setup was studied in detail by \cite[Svorenova2017] and is of great importance here as well.
    As shown in \in{Section}[sec:refinement-transition-decomposition], reachability is a fundamental building block of refinement procedures for more complex problems.

\stopsubsection


\startsubsection[title={Controller Synthesis},reference=sec:problem-statement-synthesis]

    The second problem considered is controller synthesis.
    The analysis identifies initial states from which traces can be controlled such that the given objective is fulfilled with probability 1.
    The aim of controller synthesis is to construct a controller that achieves this.

\stopsubsection

