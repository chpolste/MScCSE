To implement an interactive system exploration tool, the web browser was chosen as a platform.
Browsers are taylored to the display of visual interfaces and have built-in vector graphics capabilites.
They are arguably the most accessible platform of the modern age, as browsers are available for virtually device with a screen.
Interactivity is achieved through the JavaScript programming language which is natively tied to the browser's interface representation, the document object model.
JavaScript is an easy-to-use language which offers good performance due to JIT compilation.
Debugging tools are shipped with every modern browser.
Using node.js, JavaScript can also be run outside of the browser.

However, for the purposes of this work, there was also an important disadvantage associated with the choice of the browser.
Native browser applications are restricted to JavaScript without the possibility to interface with other programming languages via direct-call APIs or wrappers.
The only way to connect JavaScript to the outside world from within the browser is through a client-server infrastructure.
This was particularly an issue given that JavaScript libraries for control problems of hybrid systems could not be found.
E.g.\ MPT \cite[authoryears][Herceg2013] is written in MATLAB and TuLiP \cite[authoryears][Filippidis2016] in Python, so neither is directly accessible from within the browser.
While node.js can resolve these issues, it limits the available graphical capabilities and accessibility.

Nevertheless, the browser was still chosen.
The range of problems under consideration is limited to relatively simple, low-dimensional problems due to the representation- and responsiveness requirements of interactive visualization.
Time was invested to implement the required operations from convex geometry as well as game- and automaton data structures.
This also provided a valuable learning opportunity for the author of this thesis and has deepened his understanding of the given problem and its solution procedure.

