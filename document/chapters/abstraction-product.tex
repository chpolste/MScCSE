The game graph constructed in previous section has no winning condition, but in order to analyse the game with respect to a temporal logic specification one is required.
Translating the specifiction into an \omega-automaton yields an acceptance condition for the automaton which can be transfered to the game through construction of their synchronous product.
The resulting product game enforces the synchronized evaluation of the objective automaton during plays on the game graph and can be given to a game solver for analysis purposes.


\startsubsection[title={Objective Automaton}]

    The GR(1) objective formula $\varphi$ from section \in[sec:problem-statement] is formulated over a set of linear predicates $\Predicates$.
    The initial decomposition of the state space (\in[fml:abstraction-graph-decomposition]) uses an equivalence relation based on these linear predicates, so every state of a given $\State{i}$ fulfills and rejects exactly the same set of linear predicates, $\PredicatesOf{\State{i}} \in 2^\Predicates$.
    A play of $\GameGraph$ therefore induces a word over the alphabet $2^\Predicates$.
    Using this alphabet, an \omega-automaton

    \startformula
        \Automaton = (Q,\, 2^\Predicates,\, \Transition_\Automaton,\, q_0,\, \Condition_\Automaton)
    \stopformula

    corresponding to $\varphi$ can be constructed.
    The restriction to GR(1) formulas allows expression of the acceptance condition with a Streett pair $\Transition_\Automaton = \Tuple{E_\Automaton}{F_\Automaton}$ (see section \in[sec:theory-automata]).

\stopsubsection


\startsubsection[title={Synchronized Product}]

    To enforce the evaluation of the automaton synchronized to plays from the game, their synchronous product

    \startformula
        \ProductGame = ( P_1, P_2, Act, \Transition, \Condition )
    \stopformula

    is constructed.
    $\ProductGame$ is a 2½-player game with a one-pair Streett winning condition $\Condition$ modeled after $\Condition_\Automaton$.
    Player states are $P_1 = G_1 \ftimes Q$ and $P_2 = G_2 \ftimes Q$, where $\ftimes$ is the normal cartesian product but with flattened result to reduce visual clutter.
    The set of action is taken directly from the game graph $\GameGraph$.
    The synchronous evolution is upheld through the transition relation

    \startformula
        \Transition
            \Big( \Tuple{\State{i}}{q}, \PlayerOneAction{i}{J} \Big)
            \Big( \Triple{\State{i}}{J}{q'} \Big)
        = \startmathcases
            \NC \Transition_\GameGraph
            \Big( \State{i}, \PlayerOneAction{i}{J} \Big)
            \Big( \Tuple{\State{i}}{J} \Big)
            \MC \StartIf \Transition_\Automaton(q, \PredicatesOf{\State{i}}) = q'
            \NR
            \NC 0
            \NC otherwise
            \NR
        \stopmathcases
    \stopformula

    for player 1 and

    \startformula
        \Transition
            \Big( \Triple{\State{i}}{J}{q}, \PlayerTwoAction{i}{J}{K} \Big)
            \Big( \Tuple{\State{k}}{q'} \Big)
        = \startmathcases
            \NC \Transition_\GameGraph
            \Big( \Tuple{\State{i}}{J}, \PlayerTwoAction{i}{J}{K} \Big)
            \Big( \State{k} \Big)
            \MC \StartIf q = q'
            \NR
            \NC 0
            \NC otherwise
            \NR
        \stopmathcases
    \stopformula

    for player 2.
    The Streett pair is given by $\Condition = \Tuple{E_\Automaton}{F_\Automaton}$ with $E_\Automaton = (G_1 \cup G_2) \ftimes E_\Automaton$ and $F_\Automaton = (G_1 \cup G_2) \ftimes F_\Automaton$.
    Note that automaton transitions occur only after player 1 actions.

\stopsubsection


\startreusableMPgraphic{mp:abstraction-product-endloop}
    with spacing((30,30)) matrix.a(9,10);
    movepos.a[2][0](-25,0);
    % States
    with fixedboxwidth(50) with fixedboxheight(30) with shape(fixedbox) node.a[2][0](btex $\Tuple{\State{i}}{q}$ etex);
    with fixedboxwidth(50) with fixedboxheight(30) with shape(fixedbox) node.a[2][9](btex $\State{\DeadEnd1}$ etex);
    with fixedboxwidth(50) with fixedboxheight(30) with shape(fixedbox) with filling(solid) with fillingcolor(lightgray) node.a[2][5](btex $\State{\DeadEnd2}$ etex);
    % Actions
    with shape(circle) with size(35) node.a[4][7](btex \ssd $\DeadEnd$ etex);
    with shape(circle) with size(35) node.a[2][2](btex \ssd $\DeadEnd$ etex);
    with shape(circle) with filling(solid) with fillingcolor(lightgray) with size(35) node.a[0][7](btex \ssd $\DeadEnd$ etex);
    % Arrows
    with tipsize(0) arrow.rt(.9, "") (a[2][9],a[4][7]) a[2][9].c..a[3][9].c..a[4][8].c..a[4][7].c;
    arrow.llft(.5, "1") (a[4][7],a[2][5]) a[4][7].c..a[4][6].c..a[3][5].c..a[2][5].c;
    with tipsize(0) arrow.rt(.9, "") (a[2][5],a[0][7]) a[2][5].c..a[1][5].c..a[0][6].c..a[0][7].c;
    arrow.urt(.5, "1") (a[0][7],a[2][9]) a[0][7].c..a[0][8].c..a[1][9].c..a[2][9].c;
    with tipsize(0) arrow.rt(.9, "") (a[2][0],a[2][2]) a[2][0].c..a[2][2].c;
    arrow.top(.5, "1") (a[2][2],a[2][5]) a[2][2].c..a[2][5].c;
\stopreusableMPgraphic

\startsubsection[title={Dead-end States}]

    \placefigure[top][fig:abstraction-product-endloop]{
        A player 1 state $\Tuple{\State{i}}{q}$ of the product game $\ProductGame$, connected to the dead-end loop.
    }{
        \framed[width=\textwidth,frame=off]{\reuseMPgraphic{mp:abstraction-product-endloop}}
    }

    If $\Automaton$ is an incomplete \omega-automaton, $\ProductGame$ will contain dead-end states (i.e.\ states without any outgoing actions) in $P_1$.
    Because a word is rejected immediately by an incomplete automaton when a non-existing transition is found, these dead-end states can easily be removed from the product game:
    A player 1 state $\State{\DeadEnd1}$, a player 2 state $\State{\DeadEnd2}$ and an action $\DeadEnd$ are added to $P_1$, $P_2$ and $Act$, respectively.
    States $\State{\DeadEnd1}, \State{\DeadEnd2}$ form a deterministic loop that traps any entering plays:

    \startformula
        \startalign[n=2,align={right,left}]
            \NC \Transition(\State{\DeadEnd1}, \DeadEnd)(s) =
            \NC \startmathcases
                    \NC 1
                    \MC \StartIf s = \State{\DeadEnd2}
                    \NR
                    \NC 0
                    \NC otherwise
                    \NR
                \stopmathcases
            \NR
            \NC \Transition(\State{\DeadEnd2}, \DeadEnd)(s) =
            \NC \startmathcases
                    \NC 1
                    \MC \StartIf s = \State{\DeadEnd1}
                    \NR
                    \NC 0
                    \NC otherwise
                    \NR
                \stopmathcases
            \NR
        \stopalign
    \stopformula

    Any dead-end state $\State{i}$ is then deterministically connected to this loop using the action $\DeadEnd$ as depicted in Figure \in[fig:abstraction-product-endloop].
    To ensure that a play stuck in the loop will never lead to player 1 winning the game, states $\State{\DeadEnd1}$ and $\State{\DeadEnd2}$ are added to the set $E_\Automaton$ of the one-pair Streett acceptance condition $\Condition$.

    Dead-end states will also occur due to the outer states from $\ExtendedStateSpace \setminus \StateSpace$ which were explicitly constructed without actions in the game graph $\GameGraph$.
    These are connected to the same dead-end loop used for handling automaton dead ends, as traces that leave the state space immediately violate the specification.

\stopsubsection


\startsubsection[title={Co-safe Interpretation}]

    There is an exception to the last statement of the previous section:
    In the co-safe interpretation, traces that have fulfilled their objective are free to go anywhere, even outside of the state space.
    It is possible to accomodate this interpretation in the product game construction with a postprocessing step analogous to the one that resolves dead-end states.
    A new player 1 state $\State{\SatEnd1}$ and player 2 state $\State{\SatEnd2}$ is introduced together with an action $\SatEnd$.
    Final states, i.e.\ states triggering satisfaction of the co-safe objective, are stripped of their actions and then deterministically redirected into a loop constructed exactly as in Figure \in[fig:abstraction-product-endloop] but using $\State{\SatEnd1}$ instead of $\State{\DeadEnd1}$, $\State{\SatEnd2}$ instead of $\State{\DeadEnd2}$ and $\SatEnd$ instead of $\DeadEnd$.
    The only states occuring infinitely often in a trace trapped in this loop are $\State{\SatEnd1}$ and $\State{\SatEnd2}$ which are not included in either of the acceptance sets of the one-pair Streett condition.
    Trapped traces will therefore guarantee that player 1 wins the game no matter what happens after the objective has been satisfied.

    The following requirements are expected for one-pair Streett acceptance conditions $\Tuple{E}{F}$ with co-safe interpretation:
    $F$ contains all final states of the automaton or game and must not be empty.
    All other states are members of $E$.
    Using this convention, automata and games with the these properties can be interpreted in both the infinite framework, where traces must never leave the state space, and the co-safe framework, where this requirement is lifted once a final state has been reached.

\stopsubsection

