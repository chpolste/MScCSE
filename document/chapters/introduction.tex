When model checking was introduced by Edmund M. Clarke and E. Allen Emerson in the early 1980s, it provided an automated and practical alternative to the manual and proof-theoretic reasoning methods used to verify software until then \cite[authoryears][Emerson2008].
The novel combination of finite, state-based models and specifications given in a temporal logic enabled the verification of concurrent, \quotation{reactive} programs.
Today, model checking is a mature formal verification procedure, although its industrial adoption is slower than the pace of research activity surrounding it \cite[authoryears][Bennion2014].
The commonly encountered state space explosion continues to be an issue but increasing computing power as well as the development of simplification techniques have brought even large and complex systems into reach.
Because of its general and therefore flexible approach, model checking has found new areas of application since its inception.
Its correct-by-design philosophy and synthesis capabilities in particular are relied upon by the computer science and control engineering communities for the verification of cyber-physical systems and construction of controllers \cite[authoryears][Ehlers2017,Balkan2018].

