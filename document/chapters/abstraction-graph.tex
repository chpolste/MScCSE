Based on the state space decomposition from \in[sec:abstraction-decomposition] and dynamics operators from \in[sec:abstraction-operators], a game graph $\GameGraph = (G_1, G_2, Act, \Transition_\GameGraph)$ for a 2½-player game is build.
As a reminder, the goal of the game abstraction is to replace the continuous state- and control space by discrete counterparts.
The construction is that of \cite[Svorenova2017]\footnote{The intermediate non-deterministic transition system is omitted here and the game instead constructed directly from the LSS.}, who attribute their approach to \cite[Yordanov2012].

Player 1 states are the parts of the disjunct state space decomposition

\startformula
    G_1 = \IndexedStates{i}{I} \EndPeriod
\stopformula

The states are identified by their corresponding polytopes and it should always be clear from context if a $\State{i}$ refers to the polytope or the associated game state.
Since $\StateIndices = \StateIndicesInner \cup \StateIndicesOuter$, the set of outer states from $\ExtendedStateSpace \setminus \StateSpace$ is included in $G_1$ as well, simplifying the handling of transitions out of the state space.

For any given state vector $\VecState$ of the state space, the partition induces an equivalence relation $\sim_\VecState$ over the control space

\startformula
    \VecControl \sim_\VecState \VecControl' \;\Longleftrightarrow\;
    \forall j \in \StateIndices: \Big(
        ( \Posterior{\VecState}{\VecControl} \cap \State{j} = \emptyset ) \,\leftrightarrow\,
        ( \Posterior{\VecState}{\VecControl'} \cap \State{j} = \emptyset )
    \Big) \EndComma
\stopformula

i.e.\ two control vectors are equivalent if and only if the same set of state space parts is reachable after one step of system evolution.
In order to obtain actions for a state $\State{i}$ based on this relation it must be extended to the entire polytope.
However, not all states in $\State{i}$ necessarily produce the same relation, so a common $\sim_{\State{i}}$ does not generally exist.
Instead, it is defined as

\startformula
    \VecControl \sim_{\State{i}} \VecControl' \;\Longleftrightarrow\;
    \forall j \in \StateIndices: \Big(
        ( \Posterior{\State{i}}{\VecControl} \cap \State{j} = \emptyset ) \,\leftrightarrow\,
        ( \Posterior{\State{i}}{\VecControl'} \cap \State{j} = \emptyset )
    \Big)
\stopformula

and used to define the player one actions

\startformula
    Act_1 = \BigSet{ \PlayerOneAction{i}{J} \Bigmid i \in I \MidComma J \subseteq I } \EndComma
\stopformula

named after the origin state $\State{i}$ and the set of reachable target states $\IndexedStates{j}{J}$ goverened by $\sim_{\State{i}}$.
No actions are defined for the outer states $\IndexedStates{i}{\StateIndicesOuter}$, which are only used as transition targets for the inner states $\IndexedStates{i}{\StateIndicesInner}$.
Note that every possible combination of reachable target states has a unique associated action, therefore transitions out of player 1 states are deterministic under these actions.

Because of a possible mismatch between $\sim_\VecState$ and $\sim_{\State{i}}$, not every state in the action's target set may be reachable for every $\VecState \in \State{i}$.
For a trace that is currently somewhere in $\State{i}$ this means the exact set of reachable target states can only be determined after the trace has been localized in $\State{i}$.
In the abstraction, this localization is up to the other player of the game.
Player 2 represents the unknown real state of a trace in the system due to the state space abstraction and has actions that select subsets of $\State{i}$ from which the trace is continued in each step.
The game then transitions probabilistically to any of the states in the resulting reachable state set.
First, due to the deterministic player 1 actions, player 2 states can be defined as tuples of a player 1 state and action

\startformula
    G_2 = \BigSet{\Tuple{\State{i}}{J} \Bigmid i \in I \MidAnd J \subseteq I} \EndPeriod
\stopformula

The player 2 actions are denoted by

\startformula
    Act_2 = \BigSet{ \PlayerTwoAction{i}{J}{K} \Bigmid i \in I \MidComma K \subseteq J \subseteq I } \EndComma
\stopformula

where $K$ is the set of target state indices that is actually reachable from the selected subset.

The complete set of player actions is then

\startformula
    Act = Act_1 \cup Act_2 \EndPeriod
\stopformula

With states and actions identified, the transition relation $\Transition_\GameGraph$ can be defined.
Here, the connection between the player actions of the game and the system dynamics that was considered above is formalized using operators from \in[sec:abstraction-operators].
Player 1 actions are deterministic, therefore $\Transition_\GameGraph$ takes the form

\startformula
    \Transition_\GameGraph
        \Big( \State{i}, \PlayerOneAction{i'}{J'} \Big)
        \Big( \Tuple{\State{i}}{J} \Big)
    = \startmathcases
        \NC 1
        \MC \startgathered
                \NC \StartIf i = i' \MidAnd J = J'
                \NR
                \NC \quad \MidAnd \ConcreteAction{\State{i}}{\IndexedStates{j}{J}} \neq \emptyset
                \NR
            \stopgathered
        \NR
        \NC 0
        \NC otherwise \EndComma
        \NR
    \stopmathcases
\stopformula

with exactly one successor for each combination of player 1 state and action.
The connecting operator is the concrete action which reflects the action-generating relation $\sim_{\State {i}}$.
Player 2 actions are probabilistic and given by

\startformula
    \Transition_\GameGraph
        \Big( \Tuple{\State{i}}{J}, \PlayerTwoAction{i'}{J'}{K} \Big)
        \Big( \State{k} \Big)
    = \startmathcases
        \NC \displaystyle\frac{1}{|K|}
        \MC \startgathered
                \NC \StartIf i = i' \MidAnd J = J' \MidAnd k \in K
                \NR
                \NC \quad \MidAnd \PrecisePredecessor{\State{i}}{U_i^J}{\IndexedStates{k}{K}} \neq \emptyset
                \NR
            \stopgathered
        \NR
        \NC 0
        \NC otherwise \EndComma
        \NR
    \stopmathcases
\stopformula

where $U_i^J = \ConcreteAction{\State{i}}{\IndexedStates{j}{J}}$ is the control input associated with the player 1 action $\PlayerOneAction{i}{J}$ that lead to the player 2 state from player 1 state $\State{i}$.
The connecting operator is the precise predecessor, which provides the required origin subsets in $\State{i}$ with different reachable target state sets $\IndexedStates{k}{K}$.

In other words, player 1 chooses a set of (reachable) states, one of which will be randomly selected as the successor state.
But because of the real state of a trace is unknown due to the state space discretization, the support of the distribution from which the successor is sampled is not unique and must be chosen by player 2.

Simplified probability distribution: don't care about the actual transition probabilities since only almost sure analysis is considered here, where all non-zero probabilities are equivalent.

