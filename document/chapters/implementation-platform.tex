Platform chosen for the implementation was the browser.
This choice came with both advantages and challenges.

The biggest argument for the browser is its availability and graphical capabilities.
Browsers have been ported to virtually all platforms that have a screen, from desktop and laptop computers to tablets, mobile phones and even televisions.
Modern browsers (Firefox, Chromium and its derivatives) hav good support for both pixel-based and vector graphics.
SVG is natively implemented and can be accessed with the same tools as any other elements.

The language of the browser is JavaScript a simple, but flexible language that has come a long way from its beginnings due to the massive investments in web technologies that rely on it.
It allows interactive development, is interpreted and excellent debugging tools are available out-of-the-box in current browsers.
Modern javascript interpreters are supported by sophisticated just-in-time compilers, which are able to produce highly optimized machine code without the need for interfaces to C-extensions.
Using WebWorkers, even parallel processing is possible with some restrictions.
However, the inability to connect to other languages is also a problem.
Interfacing with existing non-JS code is only possible by translating that code into JavaScript as nothing else runs in the browser.
Even with automated tools such as emscripten, this is not trivial.
For this work, an entire convex geometry library had to be implemented because no mature library could be found.
This took substantial resources even though the library was restricted to 1- and 2-dimensional problems.

The interfacing problem of the browser can be overcome through the use of node.js, a javascript runtime based on chromium which allows execution of code outside of the browser.
API for extensions and possibility of interfacing via the shell exist.

