The game graph constructed in previous section has no winning condition, but in order to analyse the game with respect to a temporal logic specification one is required.
Translating the specifiction into an \omega-automaton yields an acceptance condition for the automaton which can be transfered to the game through construction of their synchronous product.
The resulting product game enforces the synchronized evaluation of the objective automaton during plays on the game graph and can be given to a game solver for analysis purposes.

The GR(1) objective formula $\varphi$ from section \in[sec:problem-statement] is formulated over a set of linear predicates $\Predicates$.
The initial decomposition of the state space (\in[fml:abstraction-graph-decomposition]) uses an equivalence relation based on these linear predicates, so every state of a given $\State{i}$ fulfills and rejects exactly the same set of linear predicates, $\PredicatesOf{\State{i}} \in 2^\Predicates$.
A play of $\GameGraph$ therefore induces a word over the alphabet $2^\Predicates$.
Using this alphabet, an \omega-automaton

\startformula
    \Automaton = (Q,\, 2^\Predicates,\, \Transition_\Automaton,\, q_0,\, \Condition_\Automaton)
\stopformula

corresponding to $\varphi$ can be constructed.
The restriction to GR(1) formulas allows expression of the acceptance condition with a Streett pair $\Transition_\Automaton = \Tuple{E_\Automaton}{F_\Automaton}$ (see section \in[sec:theory-automata]).

To enforce the evaluation of the automaton synchronized to plays from the game, their synchronous product

\startformula
    \ProductGame = ( P_1, P_2, Act, \Transition, \Condition )
\stopformula

is constructed.
$\ProductGame$ is a 2½-player game with a one-pair Streett winning condition $\Condition$ modeled after $\Condition_\Automaton$.
Player states are $P_1 = G_1 \ftimes Q$ and $P_2 = G_2 \ftimes Q$, where $\ftimes$ is the normal cartesian product but with flattened result to reduce visual clutter.
The set of action is taken directly from the game graph $\GameGraph$.
The synchronous evolution is upheld through the transition relation

\startformula
    \Transition
        \Big( \Tuple{\State{i}}{q}, \PlayerOneAction{i}{J} \Big)
        \Big( \Triple{\State{i}}{J}{q'} \Big)
    = \startmathcases
        \NC \Transition_\GameGraph
        \Big( \State{i}, \PlayerOneAction{i}{J} \Big)
        \Big( \Tuple{\State{i}}{J} \Big)
        \MC \StartIf \Transition_\Automaton(q, \PredicatesOf{\State{i}}) = q'
        \NR
        \NC 0
        \NC otherwise
        \NR
    \stopmathcases
\stopformula

for player 1 and

\startformula
    \Transition
        \Big( \Triple{\State{i}}{J}{q}, \PlayerTwoAction{i}{J}{K} \Big)
        \Big( \Tuple{\State{k}}{q'} \Big)
    = \startmathcases
        \NC \Transition_\GameGraph
        \Big( \Tuple{\State{i}}{J}, \PlayerTwoAction{i}{J}{K} \Big)
        \Big( \State{k} \Big)
        \MC \StartIf q = q'
        \NR
        \NC 0
        \NC otherwise
        \NR
    \stopmathcases
\stopformula

for player 2.
The Streett pair is given by $\Transition = \Tuple{E_\Automaton}{F_\Automaton}$ with $E_\Automaton = (G_1 \cup G_2) \ftimes E_\Automaton$ and $F_\Automaton = (G_1 \cup G_2) \ftimes F_\Automaton$.
Note that automaton transitions occur only with player 1 actions.

Handling of deadlock states (outer states, incomplete automaton).

Switching bewteen infinite and co-safe interpretation.
If co-safe objective interpretation is desired, redirect to dedicated sat state once terminal state is reached.

Product game simplification strategies:
Only construct reachable states.
If partial results are already available stop once decided or undecided state is reached.

Finally, repeat traces.
Abstraction can do everything that LSS can do, but also more.

