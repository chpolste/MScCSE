Interactive web application for system exploration, educational use, quick problem setup in 1D and 2D.
Describe features, show 2-3 screenshots.

Usefulness for development of new refinement procedures.
Make use of human ability to recognize patterns for refinement.
Product game graph too complicated to be looked at and understood due to high complexity, hundreds of states and actions even for simple systems.
Furthermore, connection to dynamics vanishes once game graph has been created and changes to the graph for some refinement are hard to estimate.
Since the LSS and player 1 states have an obvious geometric interpretation they can meaningfully be displayed for low dimensions (1D, 2D and somewhat also 3D).
This creates a direct connection between the geometry, dynamics and game which helps in understanding patterns.
Because refinement relies on a combination of these components, one can reasonably expect that the visualization will assist in the design of refinement procedures.
Even for refinement procedures based on theoretical principles, an easily accessible and interactive implementation will help in testing new procedures and ironing out practical problems.

The restriction to low dimensionality is not seen as a problem.
Linear dynamics offers few possible surprises, so dynamical results should be transferrable to higher dimensions.
When looking at the game graph, geometric dimensionality is non-existent, patterns seen in 1D and 2D will therefore not be different than those emerging in higher dimensions.

Finally, besides the applications in research it is also useful as a standalone educational tool.
Easy-to-access implementation, direct visual feedback, interactivity, inclusion of inline help support people new to the field as was the case for the author of this thesis.
Builds intuition.

Why JavaScript?
Biggest argument is availability and simplicity.
Browsers are available for virtually every platform that has a screen from desktop and laptop computers, to tablets, mobile phones and even televisions.
Node.js allows use on the command-line natively.
Easy to use language with an interpreter, which allows for quick and interactive development.
Modern javascript interpreters have sophisticated just-in-time compilers, which are able to produce highly optimized machine code without the need for C-extensions as is the case e.g. for python.
Browsers have sophisticated built-in graphical capabilities that natively tie into the language.

