In the early 1980's model checking was introduced by E. M. Clarke and E. A. Emerson and started to establish itself as an automated and practical alternative to the manual and proof-theoretic reasoning methods used to verify software until then \cite[authoryears][Emerson2008].
The novel combination of finite, state-based models and specifications given in a temporal logic had allowed the verification of concurrent, \quotation{reactive} programs.
Exponentially increasing computing power due to Moore's law as well as the development of abstraction techniques has helped to reduce the complications of the state space explosion problem.
Today, model checking is established as a mature procedure in the formal verification community, although it's industrial adoption is lacking behind the research activity \cite[authoryears][Bennion2014].
Due to its flexible approach, model checking has found new areas of application since its inception.
Its correct-by-design philosopy and synthesis capabilities are enjoyed by the computer science and control engineering communities for the verification of cyber-physical systems and their controllers \cite[authoryears][Ehlers2017,Balkan2018].

