Decompose product game into a series of co-safe reachability problems that, when solved, can be combined into a solution for the product game.
$\ProductGame$ has copy of the LSS game graph for every automaton state.
A trace through the system must reach regions associated with automaton transitions in an order that satisfies the specification.
Therefore, each transition individually has a corresponding co-safe reachability problem in the LSS.
If the satisfying regions of every one of these almost-sure reachability problems can be determined, a solution for the original problem emerges by piecing together satisfying strategies from the reachability problems, if they exist.
Because the reachability problems are co-safe, winning strategies will satisfy lead to the required automaton transitions in finite time and the composite strategy cannot get stuck in any of the individual reachability tasks.
% TODO formalize the strategy stitch-up, handle next, which requires one-step reachability but is not included in GR(1) anyhow

To construct a reachability system for a transition from $q$ to $q'$ in the automaton:
Same LSS $\LSS$, start with the same partition as current state-space abstraction but only a simple co-safe reachability/avoidance winning condition.
Elements of the current state space partition are therefore separated into 3 sets:
First,

\startformula
    \ReachStates{q}{q'} = \Set{ \State{i} \mid \Tuple{\State{i}}{q} \in P_1 \MidAnd \State{i} \notin \NoStates{q} \MidAnd ( \QNext{i}{q} = q' \MidOr \YesStates{q} ) } \EndComma
\stopformula

the elements whose union has to be reached.
These are parts where a transition to the transition target automaton state happens with any of the next player 1 actions in the product game.
It also includes all parts that have already been recognized as satisfying for the origin $q$ by a previous analysis.
Second,

\startformula
    \RefineStates{q}{q'} = \Set{ \State{i} \mid \Tuple{\State{i}}{q} \in P_1 \MidAnd \State{i} \notin \NoStates{q} \MidAnd \QNext{i}{q} = q } \EndComma
\stopformula

the elements which do not trigger an automaton transition with their player 1 actions.
These elements will be refined.
And finally,

\startformula
    \AvoidStates{q}{q'} = \IndexedStates{i}{I} \setminus \left( \ReachStates{q}{q'} \cup \RefineStates{q}{q'} \right) \EndComma
\stopformula

the elements where a transition to any other automaton state happens with the next player 1 action, as well as all elements from the no-set of the previous analysis.
These sets are disjunct except for the special case $q = q'$, where $\ReachStates{q}{q'} \cap \RefineStates{q}{q'} \ne \emptyset$.

The state space partition of the product system is shared between all copies for the automaton states.
The individual reachability problems for each transition are therefore weakly linked, i.e.\ they can all be solved independently in parallel and then combined into one partition or they can be solved sequentially such that the partitions from earlier refinements potentially provide partial solutions to later reachability problems.
Finding a \quotation{best} order of these refinements is again a non-trivial task.
For co-safe objectives, one should work backwards from the final states, which are guaranteed to be satisfying.
For infinite objectives a best path through the automaton is not obvious and multiple paths may even be required to find all satisfying initial states.

