Demonstrate the presented refinement procedures in 3 case studies.
Goal is to compare performance and illustrate challenges.
In particular, a double integrator problem that was examined in detail by \cite[Svorenova2017] is considered again in the first case study, offering the possibility to compare directly with their results.
The second case study, a simple corridor between two rooms has to be passed.
Finally, a practical problem is considered in form of an inverted pendulum which has to be stabilized in an upright position.
It is shown how a linear stochastic system can be derived from the pendulum's equation and a controller is synthesized.

Case studies produced from the implementation presented in the previous chapter.
All problems have 2-dimensional state spaces which is the highest dimension supported by the implementation and convenient for representation in this document.
The dimensionality is not restricted by the procedures, which are dimension-independent and, as argued in section \in[sec:implementation-exploration], it is not expected that higher-dimensional problems lead to fundamentally different behaviour and challenges, other than the increasing computational demands.

