% SVG conversion patch: because my svg exports can contain objects outside of
% the visible area, inkscape extends the size of the page when converting to
% pdf such that the extent of these objects is included in the view (even
% though they are not drawn). With the parameter --export-area-page, this
% behaviour can be changed to match the expected output (no empty borders).
% However, the parameters given to inkscape (which context calls for the
% conversion) are hard-coded into the lua code, so the lua code has to be
% patched.
% The following is a patch for the file
% context/tex/texmf-context/tex/context/base/mkiv/grph-con.lua
% and overrides the svg converter defined there.
\startluacode
do

    local longtostring      = string.longtostring
    local expandfilename    = dir.expandname
    local figures           = figures
    local converters        = figures.converters
    local programs          = figures.programs

    local svgconverter = converters.svg
    converters.svgz    = svgconverter

    local runner = sandbox.registerrunner {
        name     = "custom svg to something",
        program  = "inkscape",
        template = longtostring [[
            "%oldname%"
            --export-area-page
            --export-dpi=%resolution%
            --export-%format%="%newname%"
        ]],
        checkers = {
            oldname    = "readable",
            newname    = "writable",
            format     = "string",
            resolution = "string",
        },
        defaults = {
            format     = "pdf",
            resolution = "600",
        }
    }

    programs.inkscape = {
        runner = runner,
    }

    function svgconverter.pdf(oldname,newname)
        runner {
            format     = "pdf",
            resolution = "600",
            newname    = expandfilename(newname),
            oldname    = expandfilename(oldname),
        }
    end

    function svgconverter.png(oldname,newname)
        runner {
            format     = "png",
            resolution = "600",
            newname    = expandfilename(newname),
            oldname    = expandfilename(oldname),
        }
    end

    svgconverter.default = svgconverter.pdf

end
\stopluacode


% Fix for mathmatrix/mathalign problem:
% https://mailman.ntg.nl/pipermail/ntg-context/2019/094460.html
% https://tex.stackexchange.com/questions/481411:
\unprotect
\newcount\c_math_eqalign_column_saved
\newcount\c_math_eqalign_first_saved

\unexpanded\def\math_matrix_start#1%
   {\begingroup
    \globalpushmacro\c_math_matrix_first
    \c_math_eqalign_column_saved\c_math_eqalign_column
    \c_math_eqalign_first_saved \c_math_eqalign_first
    \edef\currentmathmatrix{#1}%
    \dosingleempty\math_matrix_start_indeed}

\def\math_matrix_stop
   {\math_matrix_stop_processing
    \global\c_math_eqalign_column\c_math_eqalign_column_saved
    \global\c_math_eqalign_first\c_math_eqalign_first_saved
    \globalpopmacro\c_math_matrix_first
    \endgroup}
\protect

% Vertical position of default coloncolon command is not right
\definemathcommand[coloncolon][rel]{\colon\!\colon}

