In order to describe discrete events in and generate discrete abstractions of a continuous space, a discrete representation of that space is necessary.
Polygonal descriptions in the framework of convex geometry are a popular choice.
Convex geometry can be applied to problems of any dimension, has well understood computational properties due to its roots in linear optimization and available libraries such as MPT \cite[authoryears][Herceg2013].
Furthermore, the concept of halfspaces that is central to the representation of convex polytopes works particularly well in the context of hybrid systems with linear properties.


\startsubsection[title={Polytope Representations}]

    A (closed) halfspace $H \subset \reals^n$ is the set of points

    \startformula
        H = \Set{ \Vec{x} \in \reals^n \mid \VecU \cdot \VecX \leq c }
    \stopformula

    that fulfill a linear inequality defined by the normal vector $\VecU \in \reals^n$, $\VecU \neq \Vec{0}$ and an offset $c \in \reals$.
    The normal vector $\VecU$ is pointing away from the halfspace and it is always assumed that its length is normalized, i.e. $\TwoNorm{\VecU} = 1$.

    H-representation: a bounded intersection of halfspaces $\IndexedSet{H_j}{j \in J}$ is a convex polytope $P$ and can be written as

    \startformula
        P = \bigcup_{j \in J} H_j = \Set{\VecX \in \reals^n \mid \MatU \VecX \leq \VecC } \EndComma
    \stopformula

    where $\Mat{U}$ is the stack of transposed normal vectors $\VecU_j$ of the halfspaces, $\Mat{C}$ the corresponding stack of offset values $c_j$ and the inequation holds component-wise.

    V-representation: a convex polytope can equivalently be defined as the convex hull

    \startformula
        P = \Hull(X) = \BigSet{ \sum_{i \in I} \lambda_i x_i \Bigmid \forall i : \lambda_i \in \ClosedInterval{0}{1}, \sum_{i \in I} \lambda_i = 1 }
    \stopformula

    of a finite set of points $X = \IndexedSet{x_i}{i \in I} \subset \reals^n$.
    The vertices of $P$ are the minimum set of points $\Vertices(P) \subset \reals^n$ such that $P = \Hull({\Vertices(P)})$ and uniquely define the convex polytope.

\stopsubsection


\startsubsection[title={Properties of Convex Polytopes}]

    Convex polytopes have advantages properties for practical applications, that make them popular for computational problems in geometry.
    The intersection of two convex polytopes is always a convex polytope and can easily be computed in H-representation by merging the sets of bounding halfspaces of the polytopes that are intersected.
    Linear transformations of convex polytopes again yield convex polytopes, see section \in[sec:theory-geometry-operations].
    Every non-convex polytopic region can be decomposed into a set of convex polytopes, e.g. by triangulation.
    Results based on convex polytopic geometry are therefore applicable to non-convex polytopic geometry after decomposition.

    In this work all polytopes are convex polytopes and the word \quotation{polytope} will always refer to a convex polytope.
    Furthermore, only full-dimensional polytopes are considered here, i.e. polytopes in which can fit a non-empty ball of the dimension that the polytope is embedded in.
    All other polytopes are treated as empty.
    Due to limitations imposed by computational geometry with floating point numbers, no distinction between closed and open polytopes/halfspaces is made and the closed form will be favoured in the text.

\stopsubsection


\startsubsection[title={Operations on Convex Polytopes},reference=sec:theory-geometry-operations]

    Linear transformations of a convex polytope $X \in \reals^n$ are defined as:
    application of a matrix $\MatA \in \reals^{m \times n}$ from the left and translation by a vector $\VecV \in \reals^m$

    \startformula
        \MatA X + \VecV \colonequals \Set{ \VecY \in \reals^m \mid \exists \VecX \in X : \VecY = \MatA\VecX + \VecV }
    \stopformula

    and application of a matrix $B \in \reals^{n \times n}$ from the right

    \startformula
        X \MatB \colonequals \Set{ \VecY \in \reals^m \mid \MatU \MatB \VecY \leq \VecC } \EndPeriod
    \stopformula

    The identity $X\MatB = \MatB^{-1}X$ holds for all invertible matrices $\MatB$.
    Note that application of a matrix may change the dimension of the space the polytope is embedded in.

    Aside from the usual set operations like intersection, union and difference which can be applied to convex polytopes (convex sets of points in the embedding vector space), two additional binary operations are defined for polytopes $X, Y \in \reals^n$. The Minkowski sum

    \startformula
        X \oplus Y \colonequals \Set{ \VecZ \in \reals^n \mid \exists \VecX \in X, \exists \VecY \in Y : \VecZ = \VecX + \VecY}
    \stopformula

    can be computed by translating every vertex of $X$ with every vertex of $Y$ and then taking the convex hull of the resulting set of points.
    The Pontryagin difference

    \startformula
        X \ominus Y \colonequals \Set{ \VecZ \in \reals^n \mid \forall \VecY \in Y : \VecZ + \VecY \in X }
    \stopformula

    can be computed by translating every halfspace of X by every vertex of Y and then taking the intersection of these halfspaces.
    While the Minkowski sum is commutative, the Pontryagin difference is not.
    Minkowski sum and Pontryagin difference are not inverse operations.
    In general it only holds that

    \startformula
        (X \ominus Y) \oplus Y \subseteq X
    \stopformula

    \cite[authoryears][Baotic2009].
    This is illustrated in Figure \in[fig:theory-geometry-operations], where a concrete counterexample is provided showing that Minkowski sum cannot generally invert a Pontryagin difference operation.
    Furthermore,

    \startformula
        X \ominus Y \neq X \oplus (-Y)
    \stopformula

    in general.

    \placefigure[top][fig:theory-geometry-operations]{
        Illustration of the Pontryagin difference (center) and Minkowski sum (right).
        Note that $(X \ominus Y) \oplus Y = Z \oplus Y \neq X$.
        Adapted from Figures 7 and 8 of \cite[Baotic2009].
    }{
        \startcombination[3*1]
            {\externalfigure[theory-geometry-operand][width=0.32\textwidth]}{}
            {\externalfigure[theory-geometry-pontryagin][width=0.32\textwidth]}{}
            {\externalfigure[theory-geometry-minkowski][width=0.32\textwidth]}{}
        \stopcombination
    }

\stopsubsection

