Game-graph construction expensive, independent of objective.
Observation: if a satisfying strategy can be found on just a subset of the complete product game graph, this strategy also exists in the full game graph.
Positive refinement based on only a subset of the game graph therefore has the potential to enable winning strategies for player 1 and comes at potentially much lower cost due to the reduced number of states and/or actions that have to be considered.
Transferability of negative refinement generally not possible as full game can do more than any of its subsets.

One way of selecting a subset of the game is by choosing a fixed control input region for every state such that player 1 has only a single action in any state.
The dynamics of the system is then piecewise and player 1 can be removed, which reduces the game to a 1½-player game, i.e.\ an MDP.
This makes construction and analysis less complex and allows deeper insight into the remaining graph at relatively low computational cost.
However, applying positive refinement in a piecewise dynamics is likely to lead to jaggedness and too much non-convexness is again computationally demanding, possibly offsetting gains due the reduction.
Fixed dynamics without adjustments will always have its limits after some steps.
Additionally, choosing the control regions during the reduction is non-trivial task.

Instead of reducing the number of actions one can also reduce the number of states by selecting some associated subset of game states.
This is possible when the objective is made up of separable sub-objectives, e.g. a safety property in conjunction with something else.
The idea is that solving each of the sub-objectives individually will lead to a solution for the composite objective.
Advantages: sub-objectives are often simpler (e.g.\ reachability, safety), feedback between refinement and analysis is quicker and problems can be solved in parallel.

There is actually a systematic way to decompose any objective into a series of co-safe reachability problems that, when solved, can be combined into a solution for the product game.
$\ProductGame$ has copy of the LSS game graph for every automaton state.
A trace through the system must reach regions associated with automaton transitions in an order that satisfies the specification.
Therefore, each transition individually has a corresponding co-safe reachability problem in the LSS.
If the satisfying regions of every one of these almost-sure reachability problems can be determined, a solution for the original problem emerges by piecing together satisfying strategies from the reachability problems, if they exist.
Because the reachability problems are co-safe, winning strategies will satisfy lead to the required automaton transitions in finite time and the composite strategy cannot get stuck in any of the individual reachability tasks.
% TODO formalize the strategy stitch-up, handle next, which requires one-step reachability but is not included in GR(1) anyhow
Based on this decomposition another refinement procedure is proposed in the following section.

