This work has been concerned with the computation of the set of initial states of a linear stochastic system such that there exists a control strategy to ensure an objective specified in the extended GR(1) fragment of linear temporal logic with probability 1.
The problem was previously posed and solved by \cite[Svorenova2017], who devised an iterative abstraction-analysis-refinement procedure:
A 2½-player game serves as the abstraction model and is constructed based on the dynamics of the LSS.
It is combined with the temporal logic specification by translating the latter into a deterministic \omega-automaton.
Analysing the synchronized product of game and automaton in an adversarial and cooperative setting identifies subsets of the state space for which satisfying strategies exist or cannot exist.
If the abstraction is too coarse, state space regions remain undecided and are refined based on information from the system dynamics and analysis.
The procedure generates correct solutions at every step but is not guaranteed to terminate.

Building on these ideas laid out by \cite[Svorenova2017], new refinement procedures were presented in this work.
The category of neutral refinement was introduced and filled with safety and self-loop removal refinement procedures.
The positive refinement approach was evolved into a more general robust refinement framework and extended to solve multi-step reachability problems.
A decomposition of the product game induced by transitions of the objective automaton allows to apply these methods for reachability to problems verified against arbitrary GR(1) specifications.
The effectiveness of these refinement procedures was assessed in two case studies.
Layered multi-step refinement approaches showed the best and most consistent performance as they reduce jaggedness in the state space partition and thereby stabilize and suppress the state space explosion.

The abstraction-analysis-refinement procedure together with all presented refinement methods was implemented as an interactive application for problems in 1 and 2 dimensions.
Its platform is the browser, making it very accessible and therefore suitable for educational and experimentation purposes.

