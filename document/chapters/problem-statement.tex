Two related questions are posed about the problem setup.
The first asks for the existence of controllers that ensure some specified behaviour of traces in the system.
The second requires the construction of such controllers.
These problems were asked before by \cite[Svorenova2017] and are repeated here.


\startsubsection[title={Satisfiability Analysis},reference=sec:problem-statement-analysis]

    The first problem considered is an analysis of the system \in[fml:problem-setup-lsseq] with a temporal logic objective $\varphi$.
    The aim of the analysis is to identify the set of initial states $\InitialStates$ for which a control strategy exists such that the objective $\varphi$ can be fulfilled with probability $1$ when starting a trace from within $\InitialStates$.
    This is called almost-sure analysis.
    Some authors \cite[alternative=authoryears,left={(e.g.\ }][Lahijanian2015] use the term \quotation{verification} instead of \quotation{analysis}.

    The specification $\varphi$ is formulated using the linear predicates in $\Predicates$ as atomic propositions.
    Here, objectives are limited to formulas from the extended GR(1) fragment of LTL, i.e. all objectives that can be expressed with an \omega-automaton with one-pair Streett acceptance condition (section \in[sec:theory-logic-fragments]).
    If a formula allows a co-safe interpretation in addition to an infinite interpretation, both are considered.
    In the infinite interpretation the trace is never allowed to leave the state space, whereas in the co-safe interpretation the trace can go anywhere once the objective has been satisfied.

    A notable special case is reachability analysis using the co-safe objective $\varphi = \Finally \phi$, where $\phi$ is a proposition over $\Predicates$.
    The specification is fulfilled when a trace reaches the region specified by $\phi$.
    Reachability analysis was studied in-depth by \cite[Svorenova2017] and is of great importance here as well.
    As shown in section \in[sec:refinement-reachability], reachability is the central building block of solutions for more complex problems.

\stopsubsection


\startsubsection[title={Controller Synthesis},reference=sec:problem-statement-synthesis]

    The second problem considered is the immediate follow-up problem to an almost-sure analysis: controller synthesis.
    In the analysis, initial states are identified from which traces can be controlled such that the objective is fulfilled with probability 1.
    The aim of controller synthesis is to constructively find such a controller.

\stopsubsection

