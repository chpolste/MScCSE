This work has been concerned with the computation of the set of initial states of a linear stochastic system, such there exists a control strategy to ensure an objective specified in the extended GR(1) frament of linear temporal logic with probability 1.
The problem was previously posed by \cite[Svorenova2017], who also devised an iterative abstraction-analysis-refinement procedure for its solution.
A 2½-player game serves as the abstraction model and is constructed based on the dynamics of the LSS.
It is combined with the temporal logic specification by translating the latter into an \omega-automaton and taking the synchronized product of game and automaton.
Analysing the resulting product game in both an adversarial and cooperative setting allows to classify parts of the abstraction as members of the sets of satisfying and non-satisfying initial states.
Due to coarse abstraction, membership cannot be decided for some states.
Therefore, refinement is applied to the abstraction based on information from the system dynamics, and game analysis.
The procedure generates correct (partial) solutions at every step but is not guaranteed to terminate.

Building on the ideas layed out by \cite[Svorenova2017], new refinement procedures were presented in this work.
The positive refinement approach has been evolved into a more general robust refinement scheme and the layered refinement technique extended to general objectives through a product game decomposition based on automaton transitions.
The new category of neutral refinement methods has been introduced together with two procedures called safety refinement and self-loop removal.
Their merits were discussed in the context of stutter-equivalence in next-free LTL.

The abstraction and analysis procedure together will all presented refinement methods was implemented as an interactive application for problems in 1 and 2 dimensions.
Significant improvements in the refinement performance in terms of time to solution and the number of abstraction states required for a solution have been demonstrated.
The layered decomposition approach in particularly has shown promising performance in the case studies.

