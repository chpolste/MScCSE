First set of refinement procedures are called holistic, as they consider the entire product game.
General principle of operation: seek pattern based on dynamics and game graph and then break it up.


\startbuffer[buf:refinement-holistic-positive-algorithm]
    \startalgorithmic[numbering=no,margin=0em]
        \INPUT{Element of state space partition $\State{i}$ }
        \OUTPUT{Partition of $\State{i}$}
    \stopalgorithmic
    \startalgorithmic
        \STATE{$Y \leftarrow \Set{\State{i}}$}
        \FORALL{$q \in Q$}
            \IF{$\State{i} \in \MaybeStates{q}$}
                \STATE{$q' \leftarrow \QNext{i}{q}$}
                \STATE{$Y' \leftarrow \emptyset$}
                \FORALL{$Y_n \in Y$}
                    \STATE{$Y' \leftarrow Y' \cup \RefinePositive{Y_n}{\YesStates{q'}}$}
                \ENDFOR
                \STATE{$Y \leftarrow Y'$}
            \ENDIF
        \ENDFOR
        \RETURN{Y}
    \stopalgorithmic
\stopbuffer

\startsubsection[title={Robust Positive Refinement},reference=sec:refinement-holistic-positive]

    \placealgorithm[top][alg:refinement-holistic-positive]{
        Robust positive refinement for an element $\State{i}$ of the state space partition.
    }{
        \startframedtext[width=\textwidth,frame=off]
            \getbuffer[buf:refinement-holistic-positive-algorithm]
        \stopframedtext
    }

    Implementation of case 1 of the positive refinement by \cite[Svorenova2017], adapted to use $\RefinePos$.
    Targets already recognized yes-states and refines robustly, therefore it has a progress guarantee.
    Algorithm \in[alg:refinement-holistic-positive] is an implementation.
    Note that the procedure does not require the availability of player actions, relying only on geometric operations and knowledge of the objective automaton and analysis results for the refinement.
    It can therefore easily be chained with other methods without the need for a game graph recomputation.

    Limitation: if yes-region is empty, no refinement can be made.
    This is not a problem for co-safe specifications where the final states are always recognized as satisfying states in the first analysis.
    For other objectives the initial partition will generally not yield any yes-states after the first analysis.
    This issue will be addressed in section \in[sec:refinement-transition], where the robust positive refinement will appear again in context of a reachability decomposition approach.

\stopsubsection


\startbuffer[buf:refinement-holistic-negattr-algorithm]
    \startalgorithmic[numbering=no,margin=0em]
        \INPUT{Element of state space partition $\State{i}$ }
        \OUTPUT{Partition of $\State{i}$}
    \stopalgorithmic
    \startalgorithmic
        \STATE{$Y \leftarrow \Set{\State{i}}$}
        \FORALL{$q \in Q$}
            \IF{$\State{i} \in \MaybeStates{q}$}
                \STATE{$q' \leftarrow \QNext{i}{q}$}
                \STATE{$Y' \leftarrow \emptyset$}
                \FORALL{$Y_n \in Y$}
                    \STATE{$A \leftarrow \Attractor{Y_n}{\ControlSpace}{\NoStates{q'}}$}
                    \STATE{$Y' \leftarrow Y' \cup \Convexify(A) \cup \Convexify(Y_n \setminus A)$}
                \ENDFOR
                \STATE{$Y \leftarrow Y'$}
            \ENDIF
        \ENDFOR
        \RETURN{Y}
    \stopalgorithmic
\stopbuffer

\startsubsection[title={Negative Attractor},reference=sec:refinement-holistic-negattr]

    \placealgorithm[top][alg:refinement-holistic-negattr]{
        Negative Attractor refinement for an element $\State{i}$ of the state space partition.
    }{
        \startframedtext[width=\textwidth,frame=off]
            \getbuffer[buf:refinement-holistic-negattr-algorithm]
        \stopframedtext
    }

    \cite[Svorenova2017] also identify the problematic pattern 

    \placeformula[fml:refinement-holistic-negattr-condition]
    \startformula
        \startalign[n=2,align={right,left}]
            \NC \empty
            \NC \forall (\PlayerOneAction{i}{J}) \in Act \;\, \exists (\PlayerTwoAction{i}{J}{K}) \in Act \;\, \exists k \in K : \State{k} \in \NoStates{q'}
            \NR
            \NC \Leftrightarrow \quad
            \NC \forall (\PlayerOneAction{i}{J}) \in Act \;\, \exists j \in J : \State{j} \in \NoStates{q'} \EndComma
            \NR
        \stopalign
    \stopformula

    where $q' = \QNext{i}{q}$, associated with a player 1 state $\Tuple{\State{i}}{q}$ and propose a negative refinement procedure to break it up.
    The condition matches player 1 states in which player 2 can win the game with non-zero probability independent of the action that player 1 chooses.
    Their refinement procedure partitions the associated polytopes of such states based on the set $\Attractor{\State{i}}{\ControlSpace}{\NoStates{q'}}$.
    Plays originating in this attractor region lead to player 2 winning with non-zero probability even if player 2 plays the game cooperatively.
    The removed attractor region is therefore be guaranteed to be recognized as non-satisfying in the next analysis.

    A procedure based on the negative refinement of \cite[Svorenova2017] that refines an element of the state space partition with respect to the attractor region, taking into account every associated product state is given by Algorithm \in[alg:refinement-holistic-attractor].
    Note that the basic structure of the procedure is the same as that of Algorithm \in[alg:refinement-holistic-positive] with only the inner partitioning exchanged.

\stopsubsection


\startbuffer[buf:refinement-holistic-safety-algorithm]
    \startalgorithmic[numbering=no,margin=0em]
        \INPUT{Element of state space partition $\State{i}$}
        \OUTPUT{Partition of $\State{i}$}
    \stopalgorithmic
    \startalgorithmic
        \STATE{$Y \leftarrow \Set{\State{i}}$}
        \FORALL{$q \in Q$}
            \IF{$\State{i} \in \MaybeStates{q}$}
                \STATE{$q' \leftarrow \QNext{i}{q}$}
                \STATE{$Y' \leftarrow \emptyset$}
                \FORALL{$Y_n \in Y$}
                    \STATE{$Y' \leftarrow Y' \cup \RefinePositive{Y_n}{\StateSpace \setminus \NoStates{q'}}$}
                \ENDFOR
                \STATE{$Y \leftarrow Y'$}
            \ENDIF
        \ENDFOR
        \RETURN{$Y$}
    \stopalgorithmic
\stopbuffer

\startsubsection[title={Safety},reference=sec:refinement-holistic-safety]

    \placealgorithm[top][alg:refinement-holistic-safety]{
        Safety refinement for an element $\State{i}$ of the state space partition.
    }{
        \startframedtext[width=\textwidth,frame=off]
            \getbuffer[buf:refinement-holistic-safety-algorithm]
        \stopframedtext
    }

    The negative attractor refinement from the previous section generally does not resolve all occurences of the problematic pattern (\in[fml:refinement-holistic-negattr-condition]).
    It only identifies and refines regions where player 2 wins inevitably with non-zero probability.
    However, a state $\Tuple{\State{i}}{q}$ of the product game might satisfy the condition (\in[fml:refinement-holistic-negattr-condition]) and have a player 1 action after whose selection states from $\NoStates{q'}$ with $q' = \QNext{i}{q}$ can be avoided almost-surely if player 2 cooperates with player 1, resulting in a safe transition from the perspective of player 1.
    In such cases, refinement should be applied that allows player 1 to enforce this safe transition even if player 2 plays as an adversary.
    A procedure of this kind belongs in the category of neutral refinement.
    It neither aims to enlarge the set of yes-states nor the set of no-states, but it does remove opportunities for player 2 to win with non-zero probability from the game graph by transfering additional control over the game to player 1.

    Algorithm \in[alg:refinement-holistic-safety] describes a procedure with the goal of making a state safe in the above sense by splitting the associated polytope of a player 1 state matching the pattern (\in[fml:refinement-holistic-negattr-condition]) into parts where player 1 actions exist that guarantee a safe evolution of the LSS.
    The procedure can only guarantee a safe player 1 action for some parts of the generated partition so multiple iterations may be necessary.
    It is also possible that the entire state cannot be made safe, e.g. due to the \epsilon-limit behaviour seen in section \in[sec:abstraction-analysis-correctness].
    Again note that the structure of Algorithm \in[alg:refinement-holistic-safety] mirrors that of Algorithms \in[alg:refinement-holistic-positive] and \in[alg:refinement-holistic-negattr].
    These methods all do not require the game graph to be available and are therefore trivially chainable.

\stopsubsection


\startbuffer[buf:refinement-holistic-loops-algorithm]
    \startalgorithmic[numbering=no,margin=0em]
        \INPUT{Element of state space partition $\State{i}$}
        \OUTPUT{Partition of $\State{i}$}
    \stopalgorithmic
    \startalgorithmic
        \FORALL{$q \in Q$}
            \IF{$\State{i} \in \MaybeStates{q} \MidAnd \QNext{i}{q} = q$}
                \IF{$\exists J \ne \Set{i} : ( \PlayerTwoAction{i}{J}{\Set{i}} \in Act \MidAnd \IndexedStates{j}{J} \cap \NoStates{q} = \emptyset )$}
                    \STATE{$J$: witness from previous condition}
                    \STATE{$\ControlSpace_i^J \leftarrow \ConcreteAction{\State{i}}{\IndexedStates{j}{J}}$}
                    \STATE{$P \leftarrow \PrecisePredecessor{\State{i}}{\ControlSpace_i^J}{\State{i}}$}
                    \RETURN{$\Convexify(P) \cup \Convexify(\State{i} \setminus P)$}
                \ENDIF
            \ENDIF
        \ENDFOR
        \RETURN{$\Set{\State{i}}$}
    \stopalgorithmic
\stopbuffer

\startsubsection[title={Loop Removal},reference={sec:refinement-holistic-loops}]

    \placealgorithm[top][alg:refinement-holistic-loops]{
        Self-loop removal refinement for an element $\State{i}$ of the state space partition.
    }{
        \startframedtext[width=\textwidth,frame=off]
            \getbuffer[buf:refinement-holistic-loops-algorithm]
        \stopframedtext
    }

    The observation behind neutral refinement is that for $\Next$-free LTL objectives, player 1 can take any finite amount of time to arrive at a good transition as long the environment is neutral.
    But if an adversarial player 2 can force a play into an infinite loop containing only neutral states, player 1 cannot win almost-surely.
    Such loops may involve multiple elements of the state space partition, requiring deep analysis of the game graph to be identified.
    Easier to identify are problematic self-loops, i.e.\ individual states of the game graph in which player 2 can trap a trace indefinitely.

    The same observation was made by \cite[Yordanov2012] in the context of non-probabilistic hybrid systems with piecewise linear dynamics and LTL objectives.
    They categorize self-loops into transient and non-transient loops, where transient means that it is possible to escape from the loop in finite time by repeated application of the same control input in the non-abstracted system.
    For non-transient loops they then introduce loop-breaking player 1 actions to the game abstraction to add representation of this behaviour.
    While an approach based on modification of the game graph could be considered here too, the procedure of \cite[Yordanov2012] did not include iterative refinement of the state space partition which can also be used to address non-transient loop behaviour.
    Instead of introducing special player 1 actions, non-transient loop states in the state space partition are refined such that player 1 has an action available after which player 2 is not able to enforce looping with probability 1.

    The refinement procedure in Algorithm \in[alg:refinement-holistic-loops] is proposed to remove potentially problematic self-loops in the game graph.
    Given an element $\State{i}$ of the state space partition, it searches for a player 1 state $\Tuple{\State{i}}{q} \in \MaybeStates{q}$ which has a safe player 1 action $\PlayerOneAction{i}{J}$ that leads to a player 2 state $\Triple{\State{i}}{\PlayerOneAction{i}{J}}{q}$ in which a player 2 action $\PlayerTwoAction{i}{J}{\Set{i}}$ exists whose only target is the origin player 1 state $\Tuple{\State{i}}{q}$ again.
    The procedure only looks at player 1 states whose outgoing actions don't change the automaton state in order to only remove self-loops in the graph of the product $\ProductGame$ and not generally self-loops from $\GameGraph$.
    Unsafe actions that lead to any state in the no-set with non-zero probability are ignored as they are not considered to be neutral.
    Self-loops initiated deterministically by player 1 with an action $\PlayerOneAction{i}{\Set{i}}$ can also be ignored since player 1 already has full control over the looping behaviour for these actions.

    If a player 2 action passes the proposed tests, the state space polytope $\State{i}$ is partitioned with the precise predecessor associated with the player 2 action.
    The precise predecessor set contains all states in $\State{i}$ from which a trace may enter the self-loop under the selected control input region, so after extracting these states the trace localization ability of player 2 is weakened and player 1 should have more control over the looping behaviour.
    The refinement does not guarantee that the resulting states have no new self-loops.
    Multiple applications of the procedure may be required to ensure this.

    Note that if a state space part $\State{i}$ fulfills

    \startformula
        (\Posterior{\State{i}}{\ControlSpace} \cap \State{i}) \ominus \RandomSpace = \emptyset \EndComma
    \stopformula

    no self-loop can exist since any precise predecessor of $\Posterior{\State{i}}{\ControlSpace} \cap \State{i}$ will be empty (see computation (\in[TODO])).
    This property can be used to filter candidate states for the procedure relatively cheaply when the game graph has not been computed yet, e.g.\ directly after some other refinement step.
    It also ensures that all self-loop behaviour is eliminated eventually, once the condition is true for all states of the state space partition.

\stopsubsection

