In this work, linear stochastic systems as defined in section \in[sec:theory-hybrids-lss] are considered.
Their definition is repeated here together with a few additional assumptions that are made.

An LSS $\LSS$ is a discrete-time, continuous-space dynamic system, in which traces $\VecState = \VecState_0 \VecState_1 ...$ evolve according to the equation

\placeformula[fml:problem-setup-lsseq]
\startformula
    \VecState_{t+1} = \MatA \VecState_t + \MatB \VecControl_t + \VecRandom_t \EndComma
\stopformula

with $\MatA \in \reals^{n \times n}$, $\MatB \in \reals^{n \times m}$, $\VecState_t \in \ControlSpace \subset \reals^n$, $\VecControl_t \in \ControlSpace \subset \reals^m$ and $\VecRandom \in \RandomSpace \subset \reals^n$.
For convenience, it is assumed that $\StateSpace$, $\ControlSpace$ and $\RandomSpace$ are convex polytopes.
The random vector distribution is uniform and non-zero in $\RandomSpace$.
A trace is controlled by a strategy $\Function{\Strategy{}{\LSS}}{\StateSpace^+}{\ControlSpace}$, mapping finite, fixed-sized prefixes of traces to control inputs at every timestep.
Note that such an LSS may be seen as a Markov decision process, with infinitely many states from $\StateSpace$ and infinitely many actions from $\ControlSpace$.

The LSS is equipped with a set of linear predicates over the state space.
Each linear predicate $\Predicate_k$ is associated with a halfspace $H_k = \Set{ \Vec{x} \in \reals^n \mid \VecU_k \cdot \VecX \leq c_k }$.
The set of all predicates is $\Predicates = \IndexedSet{\Predicate_k}{k \in K}$ where $K$ is an appropriately sized set of indices.
For a given state vector $\VecState$, the function $\Function{\Predicate}{\StateSpace}{2^\Predicates}$ returns the set of fulfilled predicates.
When applied to a trace, $\Predicate$ generates a word over the alphabet $\Predicates$ which can be interpreted by a temporal logic specification.

% TODO example?

