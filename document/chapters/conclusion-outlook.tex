In this work, only low-dimensional problems and specifications with small automaton sizes were considered.
While it was argued in \in{Section}[sec:implementation-exploration] that this restriction does not limit the applicability of obtained results to problems of higher dimensions and with more complex objectives, but it would still be interesting to see how the developed refinement schemes perform in more complex situations.

The limitation to full-dimensional polytopes has been identified as a source of non-termination.
It would be interesting to characterize the \epsilon-limit behaviour in more detail.
A rounding approach could be feasible to resolve the limit behaviour and therefore allow termination for a greater number of problems.

Positive refinement methods that take the probabilistic aspects of the LSS fully into account and do not treat the stochasticity as purely adversarial are still missing.

The relationship between refinement methods and controller synthesis could be further investigated.
The robust controller scheme from the corridor case study (\in{Section}[sec:cases-corridor-recurrence]) shows that concepts from one domain can be valuable for the other.

The double integrator case study was (partially) solvable through self-loop removal.
The alternative approach of \cite[Yordanov2012] to introduce additional actions based on stuttering equivalence however is more sophisticated and can potentially be transferred to the probabilistic setting.

