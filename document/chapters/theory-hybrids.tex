Hybrid systems present a framework in which discrete-event and continuous-variable dynamics are coupled to produce heterogeneous behaviour emerging from the interaction of both domains.
While \cite[VanDerSchaft2000] remarked at the start of the 21st century that the area of hybrid systems was \quotation{still largely unexplored}, two decades later \cite[Lin2014] introduce hybrid systems observing that they \quotation{have recently been at the center of intense research activity}.

The reason this interest has emerged is the broad, multidisciplinary application spectrum of the hybrid system framework and its many related theoretical research questions.
It has found use in computer-aided verification of programs interacting with continuous environments (embedded and cyber-physical systems) and is applied to control theory as well as robotics and artificial intelligence problems. %TODO: references!
Many examples of hybrid systems are found in the introductions by \cite[VanDerSchaft2000,Lin2014] and the references therein.
Probably the most often invoked example of a simple hybrid system is a heater controlled by a thermostat.
The heater works in a discrete domain, its state is either on or off.
The thermostat measures temperature, which is a continuous, real valued variable.
Both are coupled with discrete if-then-else rules: if the observed temperature falls below some threshold, the thermostat activates the heater.
Once the desired temperature is reached, the heater is turned off.

It is essential to every hybrid system that the discrete and continuous domains not just coexist but interact with each other.
The discrete domain is often governed by discrete events resulting in threshold or switching behaviour.
The continuous variables are usually associated with physical processes such as mechanical phase space, temperature or electrical current.
Time can be both discrete or continuous, leading to a description of the dynamics with difference or differential equations, respectively.
Because hybrid systems can be assembled from such a variety of components, it is difficult to find a common definition.
The solution approaches to problems involving hybrid systems however generally revolve around the synchronous evolution of the discrete and continuous domain, formalized with so-called hybrid automata.

For the purposes of this work, a special class of hybrid systems must be introduced, namely the linear stochastic system.


\startsubsection[title={Linear Stochastic Systems},reference=sec:theory-hybrids-lss]

    A linear stochastic system (LSS) $\mathcal{T}$ is a discrete-time continuous-space stochastic process evolving in traces $\VecX = \VecX_0 \VecX_1 ...$, governed by the evolution equation

    \placeformula[formula:lssevo]
    \startformula
        \VecX_{t+1} = \MatA \VecX_t + \MatB \VecU_t + \VecW_t \EndComma
    \stopformula

    where $\VecX_t \in X \subset \reals^n$ is the state of trace $\VecX$,
    $\VecU_t \in U \subset \reals^m$ is a control input and
    $\VecW_t \in W \subset \reals^n$ is a random perturbation, all at time $t \in \naturalnumbers_0$.
    For given dimensions $n, m \in \naturalnumbers$, $\reals^n$ is called the state space and $\reals^m$ the control space.
    $X, U, W$ are bounded sets.
    The probability distribution of the random vector is assumed to have non-zero density everywhere in $W$ for all times $t$.

    A trace evolves under the influence of matrix $\MatA \in \reals^{n \times n}$ transforming the current state, a stochastic perturbation and external control, which is exerted by a control input, projected into the state space by matrix $\MatB \in \reals^{n \times m}$.
    The control inputs are chosen every time step by a strategy $\Function{S_\mathcal{T}}{X^+}{U}$ which takes the evolution of the trace up to its latest state into account.
    To reason about properties of a trace, it is usually augmented with discrete events, e.g. entering or exiting some region of the state space.

\stopsubsection


\startsubsection[title={Related Systems}]

    The non-stochastic variant of an LSS has the dynamics

    \placeformula[formula:linearsystem]
    \startformula
        \VecX_{t+1} = \MatA \VecX_t + \MatB \VecU_t + \VecC \EndComma
    \stopformula

    where $\VecC \in \reals^n$, can be seen as a degenerate case of an LSS with $W = \Set{\VecC}$.
    These linear systems are a special case of piecewise affine systems, also known as piecewise linear systems, whose state space is partitioned into multiple disjunct regions, called modes, each with its own evolution equation

    \startformula
        \VecX_{t+1} = \MatA_l \VecX_t + \MatB_l \VecU_t + \VecW_{l,t} \EndComma
    \stopformula

    where is an index $l$ enumerating the modes.
    If a trace then evolves according the evolution equation associated with the mode it currently is in.
    A non-stochastic variant ($W_l = \Set{\VecC_l}$ for all modes $l$) of piecewise affine systems was used e.g. by \cite[Yordanov2009]. % TODO: more references
    More generally, any kind of (non-linear) dynamics can be considered with evolution equations of the form

    \startformula
        \VecX_{t+1} = F(\VecX_t, \VecU_t, \VecW_t) \EndPeriod
    \stopformula

    In (stochastic) switched systems control is not excerted by choosing the value of a control vector at each timestep, but choosing the entire form of the dynamics

    \startformula
        \VecX_{t+1} = U_t(\VecX_t, \VecW_t) \EndPeriod
    \stopformula
    
    The choice of $\Function{U_{t}}{X \times W}{X}$ is usually restricted to a (given) finite set of available dynamics between which the controller can switch at each timestep.
    This kind of system was used for example by \cite[Lahijanian2015]. % TODO: more references

    All presented systems so far have used discrete time and therefore difference equations to describe the evolution of traces.
    But the dynamics can also be entirely continuous, with traces evolving in time according to differential equations.
    Continuous-time, countinuous-space systems can still have hybrid characteristics by equiping them with discrete events or for example enabling control in a discrete manner (e.g. switched systems).
    However, continuous time is not of interest in this work and will not be discussed futher. % TODO: provide reference for continuous time systems

\stopsubsection

