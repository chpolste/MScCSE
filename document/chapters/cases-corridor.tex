Reachability/Avoidance through a corridor.

\startformula
    \VecX_{t+1} = \TwoByTwo{1}{0}{0}{1} \VecState_{t} + \TwoByTwo{1}{0}{0}{1} \VecControl_{t} + \VecRandom_{t} \EndComma
\stopformula

where $\VecState_{t} \in \StateSpace = \ClosedInterval{0}{4} \times \ClosedInterval{0}{3}$, $\VecControl_{t} \in \ControlSpace = \ClosedInterval{-0.5}{0.5}^2$ and $\VecRandom_{t} \in \RandomSpace = \ClosedInterval{-0.1}{0.1}^2$.
Objective is avoidance and reachability

\startformula
    ( \neg \pi ) \Until \varphi
\stopformula

co-safe interpretation.
Narrow passage between two rooms, challenge is the relatively precise control required to enter the corridor.

Show how layered refinement adapts to corridor due to removal of known no-states.
Compare with solution from pure robust refinement with target expansion and small state suppression.
Demonstrate variability of solution due to randomized $\RefinePos$ and how layered refinement can contribute to stabilize the variability in the refinement.

2-dimensional control space.
Compare game complexity to double integrator with 1D control space.

Problematic performance of round-robin controller which does not violate avoidance objective but does not want to enter the corridor.
Show off reordered round-robin controller for reachability that does better job of guiding the trace through the corridor.

