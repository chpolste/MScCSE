\startreusableMPgraphic{mp:problem-approach-flowchart}
    with spacing((25,10)) matrix.a(11,15);
    with shape(none) node.a[0][0](btex \ssd \strut LSS etex);
    with shape(none) node.a[0][3](btex \ssd \strut Linear Predicates etex);
    with shape(none) node.a[0][9](btex \ssd \strut Objective etex);
    with shape(circle) with size(3) node.a[1][0]("");
    with shape(circle) with size(3) node.a[1][2]("");
    with shape(circle) with size(3) node.a[1][9]("");
    with shape(boxed) with border(none) node.a[3][12]("");
    with shape(boxed) with border(none) node.a[3][12]("");
    with shape(boxed) with border(none) node.a[5][12]("");
    with fixedboxwidth(80) with fixedboxheight(30) with shape(roundfixedbox) with filling(solid) with fillingcolor(lightgray) node.a[4][1](btex \ss Abstraction etex);
    with fixedboxwidth(80) with fixedboxheight(30) with shape(roundfixedbox) with filling(solid) with fillingcolor(lightgray) node.a[4][9](btex \ss Analysis etex);
    with fixedboxwidth(80) with fixedboxheight(30) with shape(roundfixedbox) with filling(solid) with fillingcolor(lightgray) node.a[10][5](btex \ss Refinement etex);
    arrow.rt(.5, "") (a[1][0],a[4][1]) a[1][0].c..a[4][0].c;
    arrow.rt(.5, "") (a[1][2],a[4][1]) a[1][2].c..a[4][2].c;
    arrow.rt(.5, "") (a[1][9],a[4][9]) a[1][9].c..a[4][9].c;
    arrow.top(.5, btex \ssd Game Graph etex) (a[4][1],a[4][9]) a[4][1].c..a[4][9].c;
    arrow.bot(.5, btex \ssd Undecided States etex) (a[4][9],a[10][5]) a[4][9].c---a[10][9].c---a[10][5].c;
    arrow.bot(.5, btex \ssd State Space Partition etex) (a[10][5],a[4][1]) a[10][5].c---a[10][1].c---a[4][1].c;
    arrow.rt(1, btex \ssd Part of ${\InitialStates}$ etex) (a[4][9],a[3][12]) a[3][9].c---a[3][12].c;
    arrow.rt(1, btex \ssd Part of ${\StateSpace \setminus \InitialStates}$ etex) (a[4][9],a[5][12]) a[5][9].c---a[5][12].c;
\stopreusableMPgraphic

\placefigure[top][fig:problem-approach-flowchart]{
    A flowchart representation of the solution approach described in section \in[sec:problem-approach].
    $\InitialStates$ is the set of states for which a controller can guarantee satisfaction of the objective when a trace originates from within.
    Adapted from Fig.\ 1 of \cite[Svorenova2017] with inspiration from Fig.\ 1 of \cite[Lahijanian2015].
}{
    \framed[width=\textwidth,frame=off]{\reuseMPgraphic{mp:problem-approach-flowchart}}
}


The analysis and synthesis problems for the given LSS setup have been previously solved by \cite[Svorenova2017] with an iteratively refined, game-based abstraction.
Their solution procedure is depicted in \in{Figure}[fig:problem-approach-flowchart] and summarized in this section.
The individual steps are then reviewed in detail in the following chapter.

First, a 2½-player game abstraction of the LSS is constructed based on a state-space partition and its dynamics.
In the game, one player controls the evolution of a trace while the other player represents the uncertainty introduced by the abstraction model.
For computational convenience, convex polytopic partitions are chosen.
The initial state-space partition is induced by the set of linear predicates.

Next, the objective is translated to a deterministic \omega-automaton and the synchronous product of this automaton and the game graph is constructed.
The resulting game's winning condition is modelled on the one-pair Streett acceptance condition of the automaton.
The product game is then solved once for an adversarial player 2 and once for a cooperative player 2.
With both solutions, regions of the state space can be identified as parts of $\InitialStates$ or $\StateSpace \setminus \InitialStates$.
However, some regions can remain undecided because of a too coarse abstraction.
Therefore, refinement is applied to the state-space partition.
Based on the game solutions and system dynamics, a new state space partition is constructed and the procedure returns to the abstraction phase and iterates.

The procedure is constructed such that after every iteration, a partial solution to the problem is obtained \cite[authoryears][Svorenova2017].
This means that once a region of the state space has been identified as a part of $\InitialStates$ or $\StateSpace \setminus \InitialStates$, this identification is provably correct and will not change in subsequent analyses or due to refinement activity.
If the procedure terminates, the analysis problem from \in[sec:problem-statement-analysis] is solved.
However, refinement heuristics can only provide progress guarantees in some circumstances and there are systems for which this procedure will provably never terminate.

Finally, a controller is synthesized based on the game graph and the analysis results.

